\documentclass[9pt,oneside,notitlepageletterpaperopenright]{article}
% Use the standard package for accessing some math symbols:
\usepackage{latexsym}
\usepackage{endnotes}
\usepackage{hyperref}
\usepackage{hanging}
\usepackage{newcent}
\usepackage{setspace}
\usepackage{textcomp}

% Title, author, date, acknowledgements goes here:
\title{Literature in English 1}
\author{Wesley Raabe\\ your address}
\date{Month 1, 1998}


\setlength{\topmargin}{-2cm}
\setlength{\evensidemargin}{-.5cm}
\setlength{\oddsidemargin}{-.5cm}
%\setlength{\baselineskip}{20pt}
\setlength{\textwidth}{17.5cm}
\setlength{\textheight}{24cm}


\def\minivskip{\vskip 1mm}
\def\myspace{\phantom{\Biggr\|}}
\def\leavespace{\vskip 1em}


\begin{document}

\section*{ENG 39995: Digital Humanities Project
Assignment}\label{eng-39995-digital-humanities-project-assignment}

As noted in the ``Digital Humanities Projects'' section of the syllabus,
the overarching criterion for evaluating a scholarly editing project is
that ``scholarly editions make clear what they promise and keep their
promises'' (MLA Committee on Scholarly Editions 23). This
criterion---which I sometimes describe in paraphrase as ``Say what you
do, and do what you say''---is equally apt for digital humanities
projects. Again according to the ``Guidelines for Scholarly Editors,''
the end product (a scholarly edition) is evaluated according to five
criteria: 1) accuracy, 2) adequacy, 3) appropriateness, 3) consistency,
and 5) explicitness. See ``Sources for Paragraph 1'' below.

Because the MLA CSE Guidelines are intended for review of overall
project at time of its completion, a portion of my evaluation shall
consider other criteria during the process: project planning,
scheduling, and group collaboration. At the project's conclusion, I
shall review project in the following categories: introduction,
transcription, textual notes, annotative notes (optional), images, and
TEI-encoded text. I will collaborate with you on the publication of
TEI-encoded text with the Drupal plug-in TEICHI. See ``Sources for
Paragraph 2'' below.

The grade for your project will be based the following criteria:

\begin{itemize}
\itemsep1pt\parskip0pt\parsep0pt
\item
  Proposal (10\%)
\item
  Interim Project Contributions (10\%)
\item
  Biographical and Historical Introduction to Alfred Chester (10\%)
\item
  Textual Introduction (10\%)
\item
  Accuracy of Transcription (10\%)
\item
  Images (10\%): (\emph{Note}: Reduced percentage if prepared by
  Gilgenbach, with half each {[}5\% moves to Transcription and 5\% to
  Textual Notes{]})
\item
  Textual and Annotation Notes (10\%)
\item
  TEI XML-Encoded Text (10\%)
\item
  Peer Evaluation of Collaboration (10\%)
\item
  Short Reflective Project Essay (10\%)
\end{itemize}

\subsection*{Proposal}\label{proposal}

\textbf{The Proposal Draft is due on March 9.}

The proposal should be a collaborative project among all contributors. I
recommend that you write the proposal in Markdown and use GitHub to
coordinate your work. The final draft should be submitted as a print
copy and posted on the class blog. Though full sentences are important
for introduction matter, parts of the proposal may be more effective as
a series of bullet points, or a table. The proposal should be a 3--4
page document that addresses the following matters:

\begin{itemize}
\itemsep1pt\parskip0pt\parsep0pt
\item
  Draft schedule for entire project (timeline), with deadlines for
  individual tasks that contribute to the project
\item
  For each task listed on proposal, an assignee (or assignees) who are
  responsible for completing that task by the deadline. Please keep in
  mind my interim project deadlines on syllabus when assigning these
  tasks.
\item
  For each task listed on proposal, draft principles for how that task
  should be performed.
\end{itemize}

The principles and procedures for the following tasks should be
addressed in some detail on the proposal:

\textbf{Schedule and Assigned Tasks}

Please list tasks that need to be completed. Please list deadlines by
which each task will need to be completed for the project to be complete
by the end of the semester. Please list person (or persons) responsible
for each task.

Information that you will need to complete this task include the number
of letters, an estimate of the length of each letters. For proposal, the
schedule and assigned tasks should include contributions from all three
students.

This schedule should be re-evaluated over the remainder of the semester,
and the inability to meet deadlines (whether because personal matters,
whether because of unanticipated technical difficulties, or whether
because initial estimates were unreasonable) should prompt you to
reconsider range of tasks performed.

\textbf{Biographical and Historical Introduction}

Who is Alfred Chester? Who are his parents and siblings? Where did he
live? What are his dates of birth and death? What did he write? Where
are his papers? With whom and where did he live? With whom did he
correspond? Who are his well-known contemporaries? When researching
information on Chester, you should consult standard reference works for
writers, such as \emph{Dictionary of Literary Biography} and
\emph{American National Biography.} Consult the reference entries that
have entries on Chester. In addition, consult the Kent State catalog
entry and WorldCat. With basic information established, other valuable
sources would include Ancestry.com (see Kent State library list of
databases). Are Chester's publications collected? If Chester is
identified with any literary movements, can you identify details about
those with reference entries in \emph{Oxford Bibliographies}? To
identify additional references for Chester, consult Harner's
\emph{Literary Research Guide} and the American Library Association
\emph{Guide to Reference} at \url{http://guidetoreference.com}.

For proposal, provide brief bibliography of major reference sources and
plans for acquiring other biographical information. Please write a
100-word introduction and list of a minimum of 4 reference sources and a
brief note for each on what type of information that source provides.

\textbf{Textual Introduction}

For your final project draft, you should provide a textual introduction
with the following:

\begin{itemize}
\itemsep1pt\parskip0pt\parsep0pt
\item
  Explains the principles for choosing a set of documents to edit
\item
  Explains the principles for transcribing the documents
\item
  Explains the principles for proofreading the documents
\item
  Explains the principles for encoding the text
\item
  Explains the principles for annotating the text
\item
  Explains the methods for acquiring and rendering images for user
\item
  Explains the technology used for publication
\end{itemize}

For proposal, you should decide who will write the textual introduction
and what sources will be consulted to identify principles and methods.

\textbf{Transcription Methods and Principles}

Below are some guidelines for transcription:

\begin{itemize}
\itemsep1pt\parskip0pt\parsep0pt
\item
  No silent emendations during transcription.
\item
  Do not expand abbreviations during transcription.
\item
  Note alterations in hand: ink color, pencil, typewriter.
\item
  Note written-over corrections in square brackets:
  \texttt{{[}this over that{]}}
\item
  Place comments and interpolations in square brackets: mis-spelled
  words, items in need of annotation, etc.
\end{itemize}

Items in square brackets (notes to yourself) will need to be dealt with
systematically after the transcription is complete and corrected. You
may want to use regular expressions to replace square brackets in source
text with comment elements in XML. Some square brackets may need to be
encoded in XML as draft notes. Finally, if there are multiple versions
of the text (uncorrected or corrected), you may need to encode
corrections or multiple versions using witness tags and notes. The
eventual choices that you make on these matters will depend on editorial
principles chosen in textual introduction. But the most systematic way
to assess potential issues is during transcription.

How shall the project communicate to Cara Gilgenbach our needs?

Below are some principles for proofreading:

\begin{itemize}
\itemsep1pt\parskip0pt\parsep0pt
\item
  The usual guideline for scholarly editing is a total of 3 silent
  proofreading passes. A double-keying with digital file comparison is
  equivalent to two proofreading passes.
\item
  Whether proofreading by double-key file comparison, by oral reading,
  or by silent reading, print document and mark errors for correction.
\item
  In oral proofreading, read punctuation and printing marks aloud.
\item
  When you find an error in a line, mark it. Then, proofread the entire
  line again, from the beginning.
\item
  If original documents are avaialble to you, at least one proofreading
  pass should be against original document.
\end{itemize}

Please draft preliminary choices for acquiring text, with the following
items addressed:

\begin{itemize}
\itemsep1pt\parskip0pt\parsep0pt
\item
  How will you transcribe? From original documents? From photocopies?
  From digital facsimiles? (\textbf{Tip:} When transcribing, I recommend
  that you transcribe line by line and use a plain text editor set to
  UTF-8 encoding.)
\item
  Will you use optical character recognition (OCR) to acquire text?
  Which software?
\item
  Will you use oral or side-by-side proofreading (original or facsimile
  copy alongside transcription) to mark the transcription for
  correction?
\item
  If you will double-key the text and compare the two transcriptions,
  which software will you use? (\textbf{Tip:} XCode FileMerge, JUXTA,
  other?)
\item
  When you have a transcription that needs to be proofread for
  correction, how will you mark errors in the text? And how will you
  correct the errors and ensure that corrections are complete and that
  you do not introduce other errors during corection? (\textbf{Hint:}
  Keep 2 versions of file, uncorrected and corrected: after initial pass
  at corrections, compare with FileMerge to ensure all marked
  corrections are made.)
\item
  How will you name files? (\textbf{Tip}: I recommend naming each
  Chester letter by its date.)
\item
  How will you manage multiple versions of each file: 1) initial
  transcription, 2) second transcription, 3) initial OCR version, 4)
  corrected transcription. (\textbf{Tip}: Do not just copy over original
  file with corrected file. Devise a file naming convention for each
  type of file.)
\end{itemize}

\textbf{Image Acquisition and Processing}

Please address the following:

\begin{itemize}
\itemsep1pt\parskip0pt\parsep0pt
\item
  Who shall be responsible for acquiring images?
\item
  Which hardware and software packages will be used to acquire images?
  To change image format from, say, TIFF to XML?
\item
  How long will it take to acquire images? Count the number of images,
  and estimate number of minutes to acquire a single image, the entire
  process. Pull from folder, scan, save digital file, replace in folder.
\item
  Assuming approximately 1.5 hour image capture sessions with a
  10-minute setup and clean-up, how many image-capture sessions would be
  necessary to capture all images?
\item
  At archival standards, 600 dpi and 24-bit color, what is the
  approximate file size of each image? What is the amount of space that
  will be needed to store all images?
\item
  What method (software, settings) will be used to convert
  high-resolution archival TIFF images to display resolution JPEG
  images?
\item
  What file naming convention shall be used for images?
\end{itemize}

\textbf{Textual Notes and Annotation Notes}

How shall the following be recorded and/or displayed in your edition?

\begin{itemize}
\itemsep1pt\parskip0pt\parsep0pt
\item
  \textbf{An authorial error, uncorrected text displayed:} If you wish
  to display an uncorrected reading in case of an obvious error, will
  you encode both versions and display only uncorrected version? Or will
  you not mark the uncorrected version? Note choice in your textual
  introduction.
\item
  \textbf{An authorial error, editorial correction displayed:} If you
  wish to display a corrected reading in case of an obvious error, will
  you encode both versions and display only corrected version? Note
  choice in your textual introduction.
\item
  \textbf{An authorial error, both versions displayed:} Does your
  display system TEICHI offer an option to display both the uncorrected
  and corrected text, in a display option or with a note?
\item
  \textbf{An authorial abbreviation:} Do you wish to display an expanded
  reading or the original authorial abbreviation? Will you encode both
  versions and display only original abbreviation or expanded version?
  How will reader be notified of presence of abbreviation or an expanded
  abbreviation, editorial note or display option? Note choice in your
  textual introduction.
\item
  \textbf{An authorial correction or interpolation} If author corrects
  during initial transcription (overwrites or makes an editorial
  interpolation that corrects an error), will you encode both versions
  and display only original assay or corrected version? How will you
  notify reader? Editorial note or display option? Note choice in your
  textual introduction.
\item
  \textbf{An authorial revision:} If author revises, will you encode
  both versions and display only original draft or revised draft? How
  will you notify reader? Editorial note or display option? Note choice
  in your textual introduction.\\
\item
  \textbf{Multiple hands:} If there are multiple written hands on the
  document (author, author's friend, letter recipient, librarian mark),
  will you encode non-authorial hands? How will you notify reader?
  Editorial note or display option? Note choice in your textual
  introduction.
\item
  \textbf{Explanatory Annotation:} What types of annotation are needed
  to aid assumed reader? Some to consider include the following: proper
  names, geographical places, publications, unfamiliar words, foreign
  words. What you annotate in addition to these will depend on
  characteristics of individual documents. After a review of documents,
  can you identify classes of annotation that seem necessary for your
  assumed reader?
\end{itemize}

Please note that it is not necessary to encode, devise ambitious display
options, and serve all these possibilities. You can treat some or many
with editorial policies that include silent authorial correction, silent
editorial correction, silent expansion of abbreviations, and ignoring
alternate hands. If you provide, for example, a digital image, you may
choose to only provide a reading text for your user. But these choices
(which you include, which you exclude)should be addressed in your
textual introduction, regardless of what you choose. Also, you may
``encode'' but choose not to display. In that case, you may want to
allow some option for user to download encoded text.

\textbf{XML Encoding}

What procedures shall you follow when preparing XML encoding?

\begin{itemize}
\itemsep1pt\parskip0pt\parsep0pt
\item
  If initial transcription is made in plain text, how will you transfer
  corrected text into XML document?
\item
  If initial transcription includes simple marks like square brackets to
  designate items to be annotated, abbreviations, hand (typed, ink
  color), authorial corrections or interpolations, how will you encode
  those items in TEI XML, from among following choices: 1) update
  transcription encoding so encoded formally according to TEI in
  published document, 2) update transcription encoding to XML comments,
  which do not display, 3) Strip out initial encoding, and leave out of
  finished version. (\textbf{Tip:} Regular Expressions are your best bet
  here.)
\item
  What type of XML encoding does the publication environment TEICHI
  support for publication? Does it support the full range of TEI tags?
  Or only a limited set?
\item
  Is a desirable encoding practice too difficult to realize in XML?
  Should you reconsider editorial standards because of limitations of
  encoding?
\end{itemize}

\subsection*{Sources for Paragraph 1}\label{sources-for-paragraph-1}

The advice and criteria on scholarly editing are defined in greater
detail in \emph{Electronic Textual Editing} (see pgs. 23--46 in print
textbook), on the web site of the MLA CSE (see
\url{http://www.mla.org/resources/documents/rep_scholarly/cse_guidelines}),
and on the preview version of our textbook that is hosted by TEI
Consortium (see
\url{http://www.tei-c.org/About/Archive_new/ETE/Preview/\#body.1_div.2}).
I encourage you to consult any one of these versions of the ``Guidelines
for Scholarly Editors,'' when planning the project. These general
principles will be applied during my evaluation of any aspect of the
project.

\subsection*{Sources for Paragraph 2}\label{sources-for-paragraph-2}

For the portion on planning and scheduling, my guidelines are based on
the National Endowment for Humanities Scholary Editions and Translations
Grant Guidelines (see
\url{http://www.neh.gov/files/grants/scholarly-editions-dec-9-2014.pdf}).
Unlike for previous paragraph, I do not recommend that you consult that
guideline document---I only acknowledge its influence on this document.
The evaluation of queries and coordination is based on my own experience
with multiple digital projects and our class work with coordination
tools (including GitHub). Criteria for evaluation also have been
influenced by the online guidelins from Professor Ann R. Hawkins at
Texas Tech University (see
\url{http://hawkins.writingstore.com/index.php?option=com_content\&view=article\&id=133\&Itemid=105}).

\end{document}
