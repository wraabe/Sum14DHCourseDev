\documentclass[]{article}
\usepackage{lmodern}
\usepackage{amssymb,amsmath}
\usepackage{ifxetex,ifluatex}
\usepackage{fixltx2e} % provides \textsubscript
\ifnum 0\ifxetex 1\fi\ifluatex 1\fi=0 % if pdftex
  \usepackage[T1]{fontenc}
  \usepackage[utf8]{inputenc}
\else % if luatex or xelatex
  \ifxetex
    \usepackage{mathspec}
    \usepackage{xltxtra,xunicode}
  \else
    \usepackage{fontspec}
  \fi
  \defaultfontfeatures{Mapping=tex-text,Scale=MatchLowercase}
  \newcommand{\euro}{€}
\fi
% use upquote if available, for straight quotes in verbatim environments
\IfFileExists{upquote.sty}{\usepackage{upquote}}{}
% use microtype if available
\IfFileExists{microtype.sty}{\usepackage{microtype}}{}
\ifxetex
  \usepackage[setpagesize=false, % page size defined by xetex
              unicode=false, % unicode breaks when used with xetex
              xetex]{hyperref}
\else
  \usepackage[unicode=true]{hyperref}
\fi
\hypersetup{breaklinks=true,
            bookmarks=true,
            pdfauthor={Wesley Raabe},
            pdftitle={Assignment 1},
            colorlinks=true,
            citecolor=blue,
            urlcolor=blue,
            linkcolor=magenta,
            pdfborder={0 0 0}}
\urlstyle{same}  % don't use monospace font for urls
\setlength{\parindent}{0pt}
\setlength{\parskip}{6pt plus 2pt minus 1pt}
\setlength{\emergencystretch}{3em}  % prevent overfull lines
\setcounter{secnumdepth}{0}


\setlength{\topmargin}{-2cm}
\setlength{\evensidemargin}{-.5cm}
\setlength{\oddsidemargin}{-.5cm}
%\setlength{\baselineskip}{20pt}
\setlength{\textwidth}{17.5cm}
\setlength{\textheight}{24cm}



\title{Assignment 1}
\author{Wesley Raabe}
\date{July 10, 2014}

\begin{document}
\maketitle

\section{About ``Digital Humanities: What is
it?''}\label{about-digital-humanities-what-is-it}

Our first course project is a collaboratively authored document
entitled, ``A Student Project: What is Digital Humanities?'' The surface
rhetorical purpose of this assignment is to engage with one another in
regard to public definitions of digital humanities. All students will
contribute to that effort, first, by reading assigned articles and blog
posts, by writing a blog post, and by responding to one or two posts by
other students. After the initial discussion is offered and revised in
response to recommendations from other readers in the class (blog
comments), the class's second effort will engage in a collaborative
process to curate contributions and sharing the discussion with the
public.

Because the public may consist of different types of readers, we will
curate the content and present it in multiple formats, which may appeal
to different readers. Those formats are a public blog, a printed
document, and an ebook. However, to make the process more manageable, we
shall break the class into two teams, each of which will curate half of
the posts. You will both be challenged to use free open-access tools
that can make this collaborative project workable---Markdown language,
GitHub repository, Pandoc document converter, Zotero, WordPress blog
software, LaTeX, or Calibre---and you shall be encouraged to think about
the degree to which choosing a publication form (a blog, a print
document, an ebook) should shape the way a document is created and
published.

Is there such a thing as one ring to rule them all? Can students form a
temporary fellowship in which Markdown, the language for simple document
formatting, and Pandoc, the conversion tool, manage the magic ring
effectively? More prosaically, can one team of work effectively to
publish an extended collaborative document in multiple forms without
walking all over one another. Furthermore, it will be necessary for each
of us to see whether we can distinguish our own personal frustrations
with learning a new technology from the limitations that are
characteristic of the technology or the publication forms. Do some
publication technologies lend themselves to certain types of document
features (citation, links, keywords, commenting) that are more difficult
or perhaps even pointless when presenting the ``same'' content with
another publication technology? Many advocates of technological
solutions advocate a distinction between content (the ideas, the words,
rhetorical markings) and form (conventions of typography, spelling), and
this distinction is almost gospel among technology evangelists, who
argue that the essential advances in document management that undergird
technologies like Markdown and eXtensible Markup Language, which we will
look at later in the course, are unthinkable without a powerful
distinction between the essential content (which should stay the same)
and the accidental forms (which can be manipulated as needed). Is a
distinction between form (accident of being published online or in print
form) and content (essential part of words and ideas) liberating and
useful or constricting and misleading?

I should announce that the my choice to require you to use open access
tools instead of commercial software reflects multiple factors:
ideological leanings, pedagogical usefulness, and, hopefully, practical
benefit for your future employment. In the ideological sense, digital
humanities encourages an ethic of public sharing and collaboration. By
asking you to use public software and work together, I encourage you to
explore how this can be beneficial to your work. For example, when
frustrated with aspects of writing this assignment, my Twitter
colleagues Caleb McDaniel
\textless{}\url{https://twitter.com/wcaleb}\textgreater{} and Lincoln
Mullen \textless{}\url{https://twitter.com/lincolnmullen}\textgreater{}
provided useful guidance. You should turn to fellow members of your
team, to me, and to online guides also when you encounter technical
difficulties. Now, for example, you could collaborate and put together
most of same document formats with a combination of DropBox, Google
Drive, Microsoft Word, and Scrivener. But it is more in keeping with the
ethic of digital humanities to assume that the initial challenge of
learning these open access tools will prove rewarding, that you will
encounter alternate ways of thinking about what documents are. Instead
of deciding that you need to pay a software vendor, decide to learn
something new---an alternate way of working. The open access tools tend
to force you to put yourself into control of the documents and not rely
on software vendor choices. And everything is cheap, usually free.
Finally, I worked as a technical writer for over a decade, so I can with
some confidence tell you that you that your ability to adapt to
alternate corporate cultures and documentation methods has the potential
to improve your employment opportunities. You already know either
Microsoft Office or Google Docs, so this is the opportunity to expand
your personal toolkit. To know various methods of document
management---and to be able to make recommendations or adapt either as a
documentation manager or project manager---is a far more valuable skill
than knowing the one software program that a company or organizations
currently uses to produce documents.

\subsection{Shared Processes, Week 3}\label{shared-processes-week-3}

Each student's individual contribution to this assignment will consist
of the following parts:

\begin{enumerate}
\def\labelenumi{\arabic{enumi}.}
\item
  Each student will offer a formal response to one or more of the
  readings during the first two weeks of the course. The response will
  consist of a 2- to 3-page (500 words) blog post, with a thesis that
  engages one or more of the following general topics:

  \begin{itemize}
  \itemsep1pt\parskip0pt\parsep0pt
  \item
    digital humanities as a threat to literary studies (Kirsch; Marche);
  \item
    digital humanities as a set of institutional sites of study and
    professional organizations (Kirschenbaum);
  \item
    digital humanities as building things, such as computer
    applications, models, and online projects (Ramsay);
  \item
    digital humanities as sharing, developing a community (Sample);
  \item
    digital humanities as productive unease, grappling with tension
    between ideas of technological progress and the recognition that
    humanities research materials resist modeling and processing
    (Flanders).
  \end{itemize}
\item
  Each blog post must have at least one formal citation of another piece
  of writing, either from list above or from another blog post or
  article.
\item
  By Monday of week 3, each student's original blog post, which shall be
  written using Markdown language, will be posted on the class's
  Markdown-enabled blog AND shared on the class's GitHub repository.
\item
  By Wednesday of week 3, each student shall respond to two other
  students' blog posts. Each response must refer to a specific passage
  or idea in the student's post and compare or contrast with another
  idea from another student's post or from the assigned reading (i.e.,
  quote from and cite source). The response posts should be posted on
  the Markdown-enabled blog.
\end{enumerate}

\subsection{Splitting Up Labor, Week 4}\label{splitting-up-labor-week-4}

After initial posts are complete, the class shall collaborate to produce
posts in multiple forms. Editorial work shall be collaborative, and one
member of class will take responsibility for one of the following
processes: 1) print document, 2) public blog post, and 3) ebook. You are
encouraged to collaborate with other members of the class to manage or
troubleshoot any difficulties that you encounter. Whenever possible,
please work together (in IBE, in reserved library room, at same
computer, with online collaboration tool, or with professor) to achieve
solutions to problems.

\begin{enumerate}
\def\labelenumi{\arabic{enumi}.}
\itemsep1pt\parskip0pt\parsep0pt
\item
  During week 3, each student will contribute GitHub fork for team
  members' original blog posts. Based on commentary received on the
  blog, posts should be corrected and revised.. Furthermore, the
  editorial group by end of week shall organize the student
  contributions (section headings and introductions, shared
  bibliography), and shall provided editorial approve of the final
  project for publication. When the other teams (print, public blog,
  ebook) do their work during week 4, the editorial group shall remain
  responsive and accommodate revisions to GitHub repository documents to
  better serve needs of print, blog, or ebook team.
\item
  During week 4, the person responsible for print document shall use
  Pandoc or Pandoc and LaTeX to convert blog posts from its team members
  to a printed LaTeX document (report or article in PDF format) that
  gathers all student contributions and has a shared bibliography for
  all citations. The final product should be based on the most current
  and updated GitHub fork.
\item
  During week 4, the person responsible for public blog shall develop an
  independent public blog of all group posts with keywords and
  citations. The final product should be based on the most current and
  updated GitHub fork.
\item
  During week 4, the person responsible for the ebook shall use Pandoc
  or Pandoc and Calibre to generate an ebook in ePub format and
  (optional) publish the book on Google Books. The final product should
  be based on the most current and updated GitHub fork.
\end{enumerate}

At every stage, students are encouraged to raise the bar above minimal
assignment requirements. Ask yourself, for example, would post or
printed document benefit from a table of contents or index? Should blog
posts have keywords? Should contributor notes be published also? To earn
full credit for excellence, you need to raise the bar.

\emph{General Notes:} I encourage an iterative process. Test procedures
to see if they work. Try to do everything, but develop reasoned
processes to scale back ambitions if it becomes overwhelming. The final
step of the process will be a personal reflection on collaboration in
the assignment (500 words), which will be posted on the course blog. The
groups that are responsible for individual publication formats may make
recommendations to alter source documents. The editorial team is
responsible for assessing needs of other teams, coordinating any
document changes, and reviewing any proposed changes.

\subsection{Editorial Working Group}\label{editorial-working-group}

Your group is responsible for proper grammar and spelling and
consistency of citation and formatting. Revise and mark up individual
posts for revision. Maintain control over the source Markdown documents
in GitHub. Review and return posts with editorial markup by Tuesday.

I recommend marking up individual markdown documents with invisible
comments. But after individual edits and suggestions are made, notify
the writer who authored the post to review the edits. After writers make
their edits (preferable on GitHub source), it is your final review that
provides consistency of formatting style and editorial style.

You should install Zotero, the software on which you will create entries
for all cited materials
\href{{[}https://www.zotero.org/}{https://www.zotero.org/}. One of the
features of Zotero is shared public library. And I have placed the
initial references for ``What Is DH?'' assignment at the following
location: \url{https://www.zotero.org/groups/274025}. If not accessible,
send me a reminder. I will open group membership in Zotero library to
all students.

The job of the editorial working group is to ensure that all are cited
consistently and that no citations are duplicated. When the Zotero
library is consistent, other working groups will need the .bib file from
the Zotero library to create a bibliography. To export the bibliography,
highlight all entries in ``WhatisDH'' folder, right-click (PC) or
control-click (Mac) to launch menu and click ``Export Items.'' When
selecting format, choose ``BibLaTeX.'' Place the bibliography (.bib) in
the GitHub repository with markdown documents from blog.

\subsubsection{Print Working Group}\label{print-working-group}

For the print working group, you must use Pandoc and LaTeX to generate a
PDF and a printed document. The print working group shall use all
approved editorial posts to create a print document. The initial method
for creating the document shall be as follows:

\begin{enumerate}
\def\labelenumi{\arabic{enumi}.}
\item
  Install Pandoc and LaTeX for your operating system.
\item
  After the editorial working group's work is complete, use GitHub to
  download the final updates of blog posts and bibliography (but do test
  with sample date from GitHub before attempting the final version).
\item
  Review the following routines for managing elaborate Pandoc documents
  (multiple files with citations).

  \begin{itemize}
  \itemsep1pt\parskip0pt\parsep0pt
  \item
    Managing Multiple files in markdown.
    \url{https://github.com/akmassey/markdown-multiple-files-example}\\
  \item
    Adding name of the .bib file to YAML metadata block.
    \url{http://johnmacfarlane.net/pandoc/demo/example19/YAML-metadata-block.html}
  \item
    Setting up appropriate Citation Style Language in GitHub, which you
    add to Pandoc file directory.
    \url{https://github.com/citation-style-language/styles}. The Chicago
    style names begin with ``chicago-''. The Modern Language Association
    style names begin with ``modern-language-''. Place them in main
    document file folder for convenience.
  \item
    Download the \texttt{default.latex} template, and place it in the
    main document file folder for convenience.
    \url{https://github.com/jgm/pandoc-templates/blob/master/default.latex}.
  \item
    Review the \texttt{pandoc-cite-proc} extension documentation if you
    wish to generate bibliography entries through Pandoc
    \url{http://johnmacfarlane.net/pandoc/demo/example19/Citations.html}.
  \end{itemize}
\end{enumerate}

After the setup is complete, see the Pandoc documentation for
instructions on generating LaTeX document from Markdown with
bibliography. You essentially have two choices:

\begin{itemize}
\itemsep1pt\parskip0pt\parsep0pt
\item
  Use the \texttt{pandoc-cite-proc} extension to generate the
  bibliography simultaneously with one Pandoc command. If you do this,
  then you will need to alter the \texttt{default.latex} template to
  change the output format. Pandoc embeds the bibliographical
  information in the document as text (from the .bib source). I have not
  spent much effort learning how to alter the \texttt{default.latex}
  template.
\item
  Do not use the \texttt{pandoc-cite-proc} extension. Instead, you may
  edit the generated LaTeX document. After you adjust the format of the
  LaTeX document how you wish (narrower margins, perhaps) you may
  generate the bibliography with BibLaTeX (from the .bib source). To
  generate a bibliography from within LaTeX, you must run LaTeX, then
  run BibTeX, and then run LaTeX twice more. Many LaTeX editors have
  macros or commands so that you can run them from toolbar. You can also
  do it from the command line. To generate a bibliography from within
  LaTeX, you must run LaTeX, then run BibTeX, and then run LaTeX twice
  more. I can help you with this, as I know LaTeX quite well.
\end{itemize}

Below are some other recommendations recommendations to consider:

\begin{itemize}
\itemsep1pt\parskip0pt\parsep0pt
\item
  If you want to manually alter the LaTeX document, one dissatisfaction
  that new users often feel is with margins. For information on page
  layout, see \url{http://en.wikibooks.org/wiki/LaTeX/Page_Layout}.
\item
  After Pandoc generates your LaTeX document, you may want to consider
  uploading the document to an shared LaTeX editor, on which you can
  collaborate for your LaTeX design. The two that I would recommend you
  consider are \url{http://www.authorea.com} and
  \url{https://www.sharelatex.com/}. Do note that default for both these
  tools is to only allow one collaborator at a time for free accounts,
  though Authorea says it allows two.
\end{itemize}

\subsubsection{Public Blog Working
Group}\label{public-blog-working-group}

For the public blog working group, you must set up a WordPress
installation with a Markdown plugin. I am using Reclaim Hosting, and we
can use the shared class blog. The person who is managing the blog
should contact me, and we will work together on the setup.

\begin{enumerate}
\def\labelenumi{\arabic{enumi}.}
\itemsep1pt\parskip0pt\parsep0pt
\item
  (Administrator): Install WordPress on the domain,
  \url{http://portal.reclaimhosting.com/knowledgebase.php?action=displayarticle\&id=2}.
\item
  (Administrator): In consultation with fellow members of your working
  group, create a URL and title for blog that offers approximate
  representation of assignment question (that is, no
  ``adorablekittens.com'').
\item
  (Administrator): Invite other students to your blog as administrators.
  On Users menu, click ``Add Users,'' and invite other class members via
  their official Kent State address. For Role, select ``Administrator.''
\end{enumerate}

Students who are invited in Step 3 are set up as administrators of the
WordPress account. Subsequent steps may be performed by anyone in work
group who has an Administrator account in WordPress, so all subsequent
steps should be performed collaboratively.

\begin{enumerate}
\def\labelenumi{\arabic{enumi}.}
\itemsep1pt\parskip0pt\parsep0pt
\item
  When logged into WordPress as an Administrator, select Plugins menu
  and search for ``JetPack.'' Install it.\\
\item
  After Jetpack appears in your list of plugins, click ``Activate.''
\item
  On the Jetpack menu item, click ``Get Started'' and ``Authorize
  Jetpack.''
\item
  Create a test post with Markdown language (you may need to log out and
  log back into the blog after JetPack is installed). For details on the
  flavor of Markdown that is supported, see
  \textless{}\url{http://jetpack.me/support/markdown/}\textgreater{}.
\item
  When GitHub repository is updated with revised blog posts, develop a
  procedure to publish posts on your blog. Please attempt Markdown, but
  you have the option to decide against Markdown should it prove
  unworkable (uninstall JetPack).
\end{enumerate}

Below are some design recommendations to consider:

\begin{itemize}
\itemsep1pt\parskip0pt\parsep0pt
\item
  What should home page of blog look like? Should certain posts be more
  prominent than others--such as most recent first? Or should you select
  a design that places equal weight regardless of when post published?
  Review difference between behavior of ``page'' and ``post''.
\item
  Should documentation (works cited) be associated with individual
  posts, or should entire blog have a shared bibliography? Which makes
  more sense to user, Chicago style notes and bibliography or MLA-style
  parenthetical references? Can you use Pandoc to generate bibliography
  or notes? Is Markdown the best way to format posts, or could you use
  Pandoc to generate another format, like HTML, that would work better?
\item
  For student posts that refer to other student posts on the same blog,
  can you devise a standard system for cross-referencing between posts,
  such as a naming convention for blogs that can be used also in
  cross-references?
\item
  Could your blog be better with information in the About section or
  other widgets? Might you promote blog on Twitter or Facebook?
\item
  Would tags or keywords offer a better means of sorting posts into
  groups?
\item
  Would blog posts be better served with individual bibliographies for
  each post or a shared bibliography for entire blog.
\end{itemize}

\subsubsection{EBook Working Group}\label{ebook-working-group}

As you will read on
\textless{}\url{http://chronicle.com/blogs/profhacker/make-your-own-e-books-with-pandoc/39067}\textgreater{}
from Lincoln Mullen, creating an eBook is ``almost trivially easy.''
Mullen, I should note, does not include a bibliography in his example.
Therefore, as the ebook working group, you have the special
responsibility of improving the quality of the final product beyond the
source documents.

Below are the basic steps that Mullen recommends, revised to emphasize
our class documents.

\begin{enumerate}
\def\labelenumi{\arabic{enumi}.}
\itemsep1pt\parskip0pt\parsep0pt
\item
  Download the updated GitHub versions of our class blog posts.
\item
  Review the following routines for managing elaborate Pandoc documents
  (multiple files with citations): managing Multiple files in markdown.
  \textless{}\url{https://github.com/akmassey/markdown-multiple-files-example}\textgreater{}.
\item
  Create a title page named \texttt{title.txt} and a copyright page
  named \texttt{metadata.xml}.
\item
  Review the Pandoc options to handle bibliography entries and citation
  style.
\end{enumerate}

\begin{itemize}
\itemsep1pt\parskip0pt\parsep0pt
\item
  Review the \texttt{pandoc-cite-proc} extension documentation to
  generate bibliography entries through Pandoc
  \url{http://johnmacfarlane.net/pandoc/demo/example19/Citations.html}.

  \begin{itemize}
  \itemsep1pt\parskip0pt\parsep0pt
  \item
    Review adding name of the .bib file to YAML metadata block.
    \textless{}\url{http://johnmacfarlane.net/pandoc/demo/example19/YAML-metadata-block.html}\textgreater{}.
  \item
    Review setting up appropriate Citation Style Language in GitHub:
    place \texttt{.csl} file in Pandoc file directory.
    \textless{}\url{https://github.com/citation-style-language/styles}\textgreater{}.
    The Chicago style names begin with ``chicago-''. The Modern Language
    Association style names begin with ``modern-language-''.
  \end{itemize}
\end{itemize}

\begin{enumerate}
\def\labelenumi{\arabic{enumi}.}
\setcounter{enumi}{4}
\itemsep1pt\parskip0pt\parsep0pt
\item
  Using Pandoc, stitch all the document parts together, including
  citation entries with the \texttt{pandoc-cite-proc}.
\item
  Inspect the book in Calibre.
\item
  Publish the ebook on a free service, such as Google Books. See
  \textless{}\url{https://www.lib.umn.edu/faq/22714}\textgreater{}
\end{enumerate}

A helpful tutorial on ebook options with Pandoc and Calibre is available
at
\textless{}\url{http://puppetlabs.com/blog/automated-ebook-generation-convert-markdown-epub-mobi-pandoc-kindlegen}\textgreater{}.
However, you should carefully consult the Pandoc documentation as well,
which lists all options. See the basic tutorial at
\textless{}\url{http://johnmacfarlane.net/pandoc/epub.html}\textgreater{}
and the advanced list of options in the user's guide
\url{http://johnmacfarlane.net/pandoc/README.html}\textgreater{}

Below are some design recommendations to consider:

\begin{itemize}
\itemsep1pt\parskip0pt\parsep0pt
\item
  Should your book have embedded fonts?
\item
  Should your book have a cover page and a TOC?
\item
  Is the default style acceptable, or should you consider a .css file to
  make the format more pleasing?
\end{itemize}

A series of more technical methods for creating ebooks is available from
IBM on its developer network at
\textless{}\url{http://www.ibm.com/developerworks/xml/tutorials/x-epubtut/index.html?ca=drs-}\textgreater{}.
Though I recommend that you use Pandoc, the IBM documentation may help
you to understand other customization options.

\section*{Bibliography}\label{bibliography}
\addcontentsline{toc}{section}{Bibliography}

Flanders, Julia. ``The Productive Unease of 21st-Century Digital
Scholarship.'' \emph{Digital Humanities Quarterly} 3.3 (2009): n. pag.
Web. 9 July 2014.

Kirsch, Adam. ``Technology Is Taking over English Departments.''
\emph{The New Republic} (2014): n. pag. Web. 9 July 2014.

Kirschenbaum, Matthew. ``What Is Digital Humanities and What's It Doing
in English Departments?'' \emph{ADE Bulletin} 150.2010 (2010): 1--7.
Web. 27 Oct. 2013.

Marche, Stephen. ``Literature Is Not Data: Against Digital Humanities.''
\emph{The Los Angeles Review of Books} (2014): n. pag. Web. 9 July 2014.

Ramsay, Stephen. ``On Building.'' \emph{Stephen Ramsay} Jan. 2011. Web.
26 Oct. 2013.

Sample, Mark. ``The Digital Humanities Is Not About Building, It's About
Sharing.'' \emph{Sample Reality} May 2011. Web. 26 Oct. 2013.

\end{document}
