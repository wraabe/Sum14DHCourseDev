\documentclass[]{article}
\usepackage{lmodern}
\usepackage{amssymb,amsmath}
\usepackage{ifxetex,ifluatex}
\usepackage{fixltx2e} % provides \textsubscript
\ifnum 0\ifxetex 1\fi\ifluatex 1\fi=0 % if pdftex
  \usepackage[T1]{fontenc}
  \usepackage[utf8]{inputenc}
\else % if luatex or xelatex
  \ifxetex
    \usepackage{mathspec}
    \usepackage{xltxtra,xunicode}
  \else
    \usepackage{fontspec}
  \fi
  \defaultfontfeatures{Mapping=tex-text,Scale=MatchLowercase}
  \newcommand{\euro}{€}
\fi
% use upquote if available, for straight quotes in verbatim environments
\IfFileExists{upquote.sty}{\usepackage{upquote}}{}
% use microtype if available
\IfFileExists{microtype.sty}{\usepackage{microtype}}{}
\ifxetex
  \usepackage[setpagesize=false, % page size defined by xetex
              unicode=false, % unicode breaks when used with xetex
              xetex]{hyperref}
\else
  \usepackage[unicode=true]{hyperref}
\fi
\hypersetup{breaklinks=true,
            bookmarks=true,
            pdfauthor={Wesley Raabe},
            pdftitle={Assignment 1},
            colorlinks=true,
            citecolor=blue,
            urlcolor=blue,
            linkcolor=magenta,
            pdfborder={0 0 0}}
\urlstyle{same}  % don't use monospace font for urls
\setlength{\parindent}{0pt}
\setlength{\parskip}{6pt plus 2pt minus 1pt}
\setlength{\emergencystretch}{3em}  % prevent overfull lines
\setcounter{secnumdepth}{0}

\title{Assignment 1}
\author{Wesley Raabe}
\date{July 10, 2014}

\begin{document}
\maketitle

\section{Digital Humanities: What is
it?}\label{digital-humanities-what-is-it}

Our first course project is a collaboratively authored document
entitled, ``A Student Project: What is Digital Humanities?'' Each
student's individual contribution to this assignment will consist of the
following parts: 1. Each student will offer a formal response to one or
more of the readings during the first two weeks of the course. The
response will consist of a 2- to 3-page (500--700 words) blog post, with
a thesis that engages one or more of the following general topics: -
digital humanities as a threat to literary studies (see Kirsch; also
Marche). - digital humanities as a set of institutional sites of study
and professional organizations (see Kirschenbaum) - digital humanities
as building things, such as computer applications, models, and online
projects (see Ramsay) - digital humanities as sharing, developing a
community (see Sample) - digital humanities as productive unease,
grappling with tension between ideas of technological progress and the
recognition that humanities research materials resist modeling and
processing (see Flanders) 2. Each blog post must have at least one
formal citation of another piece of writing, either from list above or
from another blog post or article. 3. By Monday of week 3, each
student's original blog post, which shall be written using Markdown
language, will be posted on the class's Markdown-enabled blog AND shared
on the class's GitHub repository. 4. By Wednesday of week 3, each
student shall respond, at least twice, to two other students' blog
posts. Each response must refer to a specific passage or idea in the
student's post and compare or contrast with another idea from another
student's post or from the assigned reading (i.e., quote from and cite
source). The response posts should be posted on the Markdown-enabled
blog.

After initial posts are complete, the class shall be split into two
12-person teams, and each team shall be split into 4 3-person groups.
Each team shall have the following four groups: 1) editorial working
group, 2) print working group, 3) public blog working group, and 4)
ebook working group. For all of the remainder of the work, you are
encouraged to collaborate with other members of your team (inside your
group and outside), and you are encouraged to collaborate with the
corresponding group on the other team to manage or troubleshoot any
difficulties that you encounter. Whenever possible, please work together
(in reserved library room, at same computer, with online collaboration
tool) to achieve solutions to problems.

\begin{enumerate}
\def\labelenumi{\arabic{enumi}.}
\setcounter{enumi}{4}
\itemsep1pt\parskip0pt\parsep0pt
\item
  During week 3, the both editorial groups (one for each team) shall
  create a GitHub fork for its 12 team member's original blog posts.
  They shall, based on commentary received on the blog, recommend
  corrections and revisions to posts. Furthermore, the editorial group
  by end of week shall organize the student contributions (section
  headings and introductions, shared bibliography), and shall provided
  editorial approve of the final project for publication. When the other
  teams do their work during week 4, the editorial group shall remain
  responsive and accommodate revisions to GitHub repository documents to
  better serve needs of print, blog, or ebook team.
\item
  During week 4, both print working group shall use pandoc to convert
  blog posts from its team members to a printed LaTeX document (report
  or article in PDF format) that gathers all student contributions
  (including those by editorial team) and has a shared bibliography for
  all citations. The final product should be based on the most current
  and updated GitHub fork.
\item
  During week 4, both public blog working group shall develop an
  independent public blog of all group posts with keywords and
  citations. The final product should be based on the most current and
  updated GitHub fork.
\item
  During week 4, both ebook groups shall use Calibre to generate an
  ebook in ePub format and publish the book on Google Books. The final
  product should be based on the most current and updated GitHub fork.
\end{enumerate}

At every stage, students are encouraged to raise the bar above minimal
assignment requirements. Ask yourself, for example, would post or
printed document benefit from a table of contents or index? Should blog
posts have keywords? Should contributor notes be published also?

Final notes: I encourage an iterative process. Test things to see if
they work. Try to do everything, but developed reasoned processes to
scale back ambitions if it becomes overwhelming. The final step of the
process will be a personal reflection on collaboration in the assignment
(500 words), which will be posted on the course blog.

\section*{References}\label{references}
\addcontentsline{toc}{section}{References}

Flanders, Julia. ``The Productive Unease of 21st-Century Digital
Scholarship.'' \emph{Digital Humanities Quarterly} 3.3 (2009): n. pag.
Web. 9 July 2014.

Kirsch, Adam. ``Technology Is Taking over English Departments.''
\emph{The New Republic} (2014): n. pag. Web. 9 July 2014.

Kirschenbaum, Matthew. ``What Is Digital Humanities and What's It Doing
in English Departments?'' \emph{ADE Bulletin} 150.2010 (2010): 1--7.
Web. 27 Oct. 2013.

Marche, Stephen. ``Literature Is Not Data: Against Digital Humanities.''
\emph{The Los Angeles Review of Books} (2014): n. pag. Web. 9 July 2014.

Ramsay, Stephen. ``On Building.'' \emph{Stephen Ramsay} Jan. 2011. Web.
26 Oct. 2013.

Sample, Mark. ``The Digital Humanities Is Not About Building, It's About
Sharing.'' \emph{Sample Reality} May 2011. Web. 26 Oct. 2013.

\end{document}
