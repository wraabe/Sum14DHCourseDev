\documentclass[]{article}
\usepackage{lmodern}
\usepackage{amssymb,amsmath}
\usepackage{ifxetex,ifluatex}
\usepackage{fixltx2e} % provides \textsubscript
\ifnum 0\ifxetex 1\fi\ifluatex 1\fi=0 % if pdftex
  \usepackage[T1]{fontenc}
  \usepackage[utf8]{inputenc}
\else % if luatex or xelatex
  \ifxetex
    \usepackage{mathspec}
    \usepackage{xltxtra,xunicode}
  \else
    \usepackage{fontspec}
  \fi
  \defaultfontfeatures{Mapping=tex-text,Scale=MatchLowercase}
  \newcommand{\euro}{€}
\fi
% use upquote if available, for straight quotes in verbatim environments
\IfFileExists{upquote.sty}{\usepackage{upquote}}{}
% use microtype if available
\IfFileExists{microtype.sty}{\usepackage{microtype}}{}
\usepackage{longtable,booktabs}
\ifxetex
  \usepackage[setpagesize=false, % page size defined by xetex
              unicode=false, % unicode breaks when used with xetex
              xetex]{hyperref}
\else
  \usepackage[unicode=true]{hyperref}
\fi
\hypersetup{breaklinks=true,
            bookmarks=true,
            pdfauthor={},
            pdftitle={},
            colorlinks=true,
            citecolor=blue,
            urlcolor=blue,
            linkcolor=magenta,
            pdfborder={0 0 0}}
\urlstyle{same}  % don't use monospace font for urls
\setlength{\parindent}{0pt}
\setlength{\parskip}{6pt plus 2pt minus 1pt}
\setlength{\emergencystretch}{3em}  % prevent overfull lines
\setcounter{secnumdepth}{0}


\begin{document}

\section{Title of Course (CRN XXXXX)}\label{title-of-course-crn-xxxxx}

\begin{longtable}[c]{@{}ll@{}}
\toprule\addlinespace
Wesley Raabe & Fall 2015: ENG XXXXX-XXX (CRN XXXXX)
\\\addlinespace
Contact Info: & KSU Email (preferred): wraabe@kent.edu
\\\addlinespace
& Twitter: @wraabe
\\\addlinespace
& By phone during office hours: messages checked
\\\addlinespace
& ONLY during office hours
\\\addlinespace
Office Hours & SFH 205c (Ph. 672-2092):
\\\addlinespace
& Library 920 (Ph. 672-1723):
\\\addlinespace
& By appointment (at agreed time, with 4-hr.
\\\addlinespace
& notice to cancel)
\\\addlinespace
\bottomrule
\end{longtable}

\subsection{Notices}\label{notices}

\begin{itemize}
\itemsep1pt\parskip0pt\parsep0pt
\item
  The prerequisite for this course is either College Writing I (ENG

  \begin{enumerate}
  \def\labelenumi{\arabic{enumi})}
  \setcounter{enumi}{11010}
  \itemsep1pt\parskip0pt\parsep0pt
  \item
    or Honors Colloquium I (HONR 10197).
  \end{enumerate}
\item
  Consult the registrar calendar for each semester's add/drop and
  withdrawal (no grade) date, which may vary among courses.
\item
  If you are not officially registered by add/drop deadlines, you will
  not receive credit or a grade for the course. Confirm enrollment by
  checking your class schedule in FlashLine. Errors must be corrected
  prior to the add/drop deadline.
\end{itemize}

\subsection{Goals}\label{goals}

The goals for students in ``ENG 39395, ST: Intro to Digital Humanities''
are the following:

\begin{enumerate}
\def\labelenumi{\arabic{enumi}.}
\itemsep1pt\parskip0pt\parsep0pt
\item
  To acquire basic digital literacy skills (text and image acquisition,
  text encoding, image processing, text processing) that undergird
  web-based technology
\item
  To become cognizant of the challenges of reproducing cultural
  artifacts and to become aware that some benefits of digitization, such
  as access, depend on types of translation that include the risk of
  damaging original artifacts or of not fully representing some
  qualities of original documents;
\item
  To develop project management and collaboration skills that are
  required for significant web-based projects;
\item
  To develop interpersonal skills and communication techniques for
  addressing systematically and collaboratively the challenges that
  impede progress in group-based technology projects;
\item
  To achieve awareness of the responsibilities that developers of public
  projects have for crediting sources and for ensuring access by diverse
  audiences;
\item
  To develop a professional public identity that is associated with a
  web project and with associated class materials (blog posts, tweets)
  in publicly accessible online sites and forums.
\end{enumerate}

\subsection{Course Materials}\label{course-materials}

TBD

\subsection{Grading}\label{grading}

\begin{longtable}[c]{@{}ll@{}}
\toprule\addlinespace
20\% & Blogs, Assignments, In-Class Participation, and Quizzes
\\\addlinespace
20\% & Paper 1 at 4--5 pgs. with proposal and final draft
\\\addlinespace
30\% & Paper 2 at 7--8 pgs. with proposal, workshop, and final draft
\\\addlinespace
30\% & Exams
\\\addlinespace
\bottomrule
\end{longtable}

I grade and return all on-time assignments (quizzes, assignments, exams,
papers) within a week. When blog assignments are staggered (e.g., 4
students blog for each class period), I grade all blog posts within one
week after a ``set'' is complete. I allow myself one extended grading
period (extra week) for a quiz, blog set, or exam. If you miss class,
please contact me about work returned during previous class. And please
double-check grade entries on Blackboard. \emph{Caution:} As a formal
policy, if you earn a failing grade (below D- or 60\%) on two or more
papers or exams, you will automatically fail the course.

\section{Accessibility Statement}\label{accessibility-statement}

My aim is for course content to be available to all students. Students
who have a documented disability may need reasonable accommodations to
participate fully in this class. Even if you do not have a documented
disability, some materials that I provide may present challenges. For
example, I tend to rely on handouts or posted instructions to spell out
detailed requirements on assignments. If that means for absorbing
information is challenging to you, you are welcome to stop by my office
hours to discuss. Many of the basic university services that are
available to all students in this class---office hours, library
reference desk, writing center, departmental advising, psychological
services---are available to you on an as-needed basis without formal
documentation.

In the case of a formally designated ``documented disability,''
adjustments that alter course policies or procedures to make the course
more accessible to one student may result in different course policies
for different students---and that's fine. I and other professors have
varied policies in different classes according to number of students,
pedagogical aim of class, etc. So the legally defined standard of
``reasonable accommodation'' is a sensible burden for a professor to
assume in order to ensure the greater value of accessibility---because
the burden that a professor assumes to provide alternate options for
accessibility is no greater than what students bear when professors have
different policies. However, for you to receive an accommodation---for
me to alter general syllabus policy on your individual
behalf---university policy requires that you complete the paperwork to
verify that you have a ``documented disability.'' And you must complete
the paperwork at the start of the semester to verify your eligibility.
Therefore, to ensure that you receive the accommodation to which you
have a right, \textbf{you must first verify your eligibility for these
through Student Accessibility Services} (contact 330-672-3391 or visit
\textless{}\url{www.kent.edu/sas}\textgreater{} for more information on
registration procedures). Consult legalistic details for all university
policies at
\textless{}\url{http://www.kent.edu/policyreg/index.cfm}\textgreater{}.

\subsection{Blog Posts and Formal
Assignments}\label{blog-posts-and-formal-assignments}

You will contribute 4 to 5 blog posts of approximately 300 words to the
class blog. A blog post is your opportunity to build on in-class
discussion, to select IDs and Quotations for exams, and to contribute
your own observations. While blog posts need not observe the studied
formality of a paper, the blog post should exhibit proper spelling and
punctuation, its claims about the text should be supported by textual
citation as evidence, and it should conclude with a bibliographical
entry in a close approximation of MLA style. If you are assigned to
contribute a blog post, please publish the post within 24 hours of the
end of class. A handout will list requirements and suggestions for blog
posts. Blog posts can complement active in-class participation, and
ambitious blog posts can compensate entirely for minimal oral
participation in class. Blog assignments are submitted only in
electronic format.

At the start of the semester, I will allow full credit for blog posts up
to one class session late and will advise in class with reminders on
what a complete blog post requires: visit me during office hours if you
are having trouble. For your second and subsequent posts, I will expect
the posts to appear on time, within 24 hours of end of class. I grade
blog posts either at the end of the week or after the entire set is
complete. For late posts, I deduct a letter grade. Posts more than one
week late earn no credit.

Formal assignments (papers and assignments) are listed on syllabus and
posted on Blackboard. All formal assignments must be submitted both as
\textbf{print copies} and as \textbf{electronic copies} on Blackboard.
You may submit a word processor document in any of the following forms:
Word DOC or DOCX, Acrobat PDF, GoogleDoc, or Open Document. If
submitting a Google Doc, you may link to KSU Google Drive so long as you
grant me access to the document and do not alter it after the due date.
Presentations may be submitted as PowerPoint Document or Shows (PPT,
PPTX) or as Google Presentations or Prezi presentations. Paper-style
assignments must also be submitted as printed documents.

\textbf{Formal assignments and papers submitted electronically via other
means (such as email) will not be accepted as on time nor will they be
accepted for credit. Submissions in non-designated proprietary formats
(including Apple Pages) will not be accepted for credit. You may request
permission to submit papers in alternate electronic format: if I have or
can locate non-proprietary free software to read the document, I will
accept them, provided you have requested and received approval before
submission. Corrupt file formats (invalid extensions, etc.) shall be
construed as missed assignments. }

\subsubsection{In-Class Participation}\label{in-class-participation}

Participation during class-wide or small-group discussion is preferred.
A smaller number of high-quality contributions is valued more than
contributions of great quantity. A handout on the art of Sprezzatura
will list suggestions for developing high-quality in-class
contributions, which is also the model for informal blog writing.
Attendance and active in-class participation can complement (but not
substitute for) blog posts and assignments and quizzes.

\subsubsection{In-Class Quizzes}\label{in-class-quizzes}

Short in-class quizzes (4--5 questions) may include questions to review
the reading and the lecture or discussion or activity from the previous
class. Quizzes will typically have at least one question on reading due
for the current class. The quiz will allow you to demonstrate that you
recognize characters, place names, plot events, and brief quotations
from reading. Spelling and quotation forms will always be from the
required book. If you miss a quiz because you are absent or late, you
will receive a grade of ``0'' on the quiz. In general, quizzes, which
are part of in-class participation, may not be made up if you are absent
or late, but see ``Late Work and Extended Absences'' below. I am not a
big fan of quizzes and will only use them if class participation shows
that a significant number of students are choosing to skip reading.

\subsection{Absences and Disruptions}\label{absences-and-disruptions}

You are permitted to miss the equivalent of up to 2 class sessions on
non-exam days with no formal penalty to your participation grade for the
absences, though you cannot make up in-class grades or activities. For
no more than 2 absences, your ``in-class participation'' portion of your
participation grade will range from A to B+. For 3 or 4 absences, the
``in-class participation'' portion of your participation grade will be
calculated at B- or C+. For 5 or 6 absences, C to D-. For 7 or more, a
``0.''

I will distinguish between excused and unexcused absences. To grant an
excused absence, I expect formal notification (email or voice mail
message) as soon as practicable. In the case of family matters or
serious illness, please send me a brief message as soon as practical.
For scheduled university activities, contact me BEFORE the expected
absence. So that I can arrange make-up exams or quizzes, please provide
a one-week notice prior to the absence. I respect your privacy. You need
not provide documentation in forms of doctor's excuse, death
certificate, etc. An email message in the following form qualifies an
absence as excused: ``I need to attend to a health matter,'' or ``I had
to attend to a family matter.'' Please also notify me when you expect to
return to class. I expect you to contact me at the earliest convenient
time: do not wait until day of your return to class.

\textbf{Extended Absences:} If you suffer an extended health matter or
family crisis---you miss more than a week---you may make up one quiz or
turn in one blog or assignment late without penalty. You will earn full
credit for the make-up. One set of extended absence dates (up to two
weeks) for a serious matter can be worked around during the first 10
weeks of the semester. During late-semester, extended absences for a
qualifying reason (death in family, illness) may qualify you to seek an
incomplete. If I do not receive reasonably regular communication about a
matter that you are attempting to address (at least every two weeks), I
will assume that you intend either to withdraw, request incomplete, or
to face grade consequence for excessive absences. Because of past
experience with students who request extraordinary assistance from
professor in their effort to catch up after missing several classes, I
will only provide catch-up assistance after you have returned to class
and successully completed at least one missed assignment satisfactorily.
Serious health or family catastrophes (more than 3 weeks) may qualify
you to have a semester expunged from your record. For such matters, you
should contact your academic advisor or the student ombuds office, which
will coordinate with your professors.

Keep disruptions to a minimum. Before class begins, silence or turn off
electronic devices (pager, phone, etc.). Conversations unrelated to
class should be held outside of class, and minimize communication (talk,
or text) that distracts you or others from class. Arrive in class on
time, and do not leave early. If you arrive more than ten minutes late
or leave before class is dismissed, expect to be counted absent. To
consult with the instructor, send an email, drop by during office hours,
or schedule an appointment.

\subsection{Maintaining Communication}\label{maintaining-communication}

Regardless of whether absences are excused or unexcused, you are
responsible for checking on Blackboard or class blog and to contact
classmates or group members to identify what you missed. You are also
advised to stop by during office hours or to contact me by email to
confirm what you missed. If you schedule an office visit during my
normal hours but are unable to attend, please notify me at least 4 hours
before the scheduled visit. Office visits during my ``by appointment''
hours should always be scheduled. If you schedule a by-appointment
office visit outside of my usual hours and miss the scheduled
appointment, I will count it as an absence.

In the case of an extended series of absences or an unexplained absence
on a major paper or project or exam date, you are required to initiate a
formal contact with the professor (email, office visit) to reinstate
yourself in the class. Any of the following three events demand that you
contact me: missing more than two classes in a row, missing an exam, or
missing a paper due date and the following class. If you have not
formally dropped and wish to continue in course, an email of explanation
and an office visit are required within two days after returning to
class. If during early semester you miss more than three classes in a
row or if you miss class at a major due date (paper, exam) with no
contact, I will file an ``early alert'' on the campus notification
system. If you miss multiple classes or major due date late in the
semester (weeks 10--15), I will contact you once via email. If you do
not respond promptly (within 48 hours), I will assume that you intend to
drop.

\subsection{Summary}\label{summary}

To maintain yourself in good class standing, it is not acceptable to
skip more than two classes in a row. To restore your standing after
missing three or more classes, I expect to return to class \textbf{and}
set up an office hour appointment. Sporadic good-faith efforts (a paper
in my mail box or on Blackboard, a cryptic email) may demonstrate that
you have a functioning conscience, but possession of a conscience is not
a substitute for class attendance and for submitting assignments and for
taking exams on proper due date.

\section{Papers}\label{papers}

Papers must \emph{always} be submitted in print and electronic form. To
earn full credit, follow all conventions of academic prose and format.
In general I assume the following matters are understood as expectations
for academic papers, but you should review and highlight anything that
departs from your previous practices on papers.

\begin{itemize}
\item
  Papers must have appropriate format for titles (centered, no extra
  space), first-page headings (your name, date, my name, name of class
  and assignment), page numbers, appropriate font (11-pt. Times Roman or
  similar), 1-inch margins, and line spacing. For a sample MLA Style
  paper, see a handbook or the Purdue OWL site. Papers with comically
  exaggerated font size, line space, or margins to lengthen or shorten
  will be returned without credit.
\item
  A paper in standard format, when one allows for difference between one
  or two extended block quotations and all full-length prose lines, has
  about 400 to 425 words per page. Because of heading matter, the first
  page will have fewer words, about 350 to 375: a 4-page paper has 1,550
  to 1,650 words, and a 6-page paper has 2,350 to 2,500 words.
  Generally, I assign papers with flexible length, ``4 to 5'' or ``6 to
  7'' pages. Therefore, based on word count math (5 characters is a
  ``word''), a flexible cushion is built into assignment: ``4 to 5''
  pages may be read as ``1,550 to 2,075'' words. To qualify for full
  credit, an ``A,'' your paper should not depart from these norms by
  more than 10 percent.
\item
  Guidelines on length may seem arbitrary, but you should change your
  way of thinking about that: editors and publishers always have length
  guidelines. The time that you spend revising to ensure your paper
  falls into appropriate length, if you exercise good planning and
  self-discipline so as to demand productive work from yourself, is some
  of the most difficult but important work of writing for a designated
  audience. Tell-tale signs of excessive attention to formatting
  (instead of revision) include the following: fewer than 23 or more
  than 25 lines on a full page; a 0.75 or 1.5-in. page margin; a font at
  a peculiar size like 10.3- or 11.8-pt. or a sans-serif face. Block
  quotes seem especially to invite creativity in the formatting vein, so
  observe following guide: no extra padding of 1-in. left indent, no
  right-margin indent; no 3-line or 8-plus line quotes, and no single
  spacing or extra line space preceding or following. I worked as a
  university press typesetter, typically receive well over 1,000
  manuscript pages per semester, and have access to your electronic
  submissions, so don't waste an hour on formatting cleverness to try to
  sneak something by.
\item
  Use MLA parenthetical references for quotations and paraphrases. At
  end of paper, include works cited list. I do not require a separate
  page for works cited list. If you can save a page, you may print part
  of works cited list on bottom of last text page (I accept that. Some
  professors may not). If the author of a quoted or paraphrased passage
  is unambiguous (i.e., mentioned in sentence, same as previous, primary
  work under discussion), do not repeat author's name in parenthetical
  notation.
\item
  The proposal draft is required. A final draft will only be accepted
  for credit if the proposal draft has been completed.
\item
  You may only submit one paper or proposal late. The late paper
  submissions at any stage (proposal or final) will incur a permanent
  deduction of one letter grade on the overall paper. A second paper or
  proposal submitted late will be assigned a grade of ``0.''
\item
  \textbf{1st and 3rd Person} The judicious use of the first-person
  pronoun ``I'' is acceptable. You can avoid its use in formal writing
  as 3rd-person writing carries with it the assumption that the writer
  holds a critical view or offers an observation. Brief 1st-person
  impressions are permissible in formal writing in my academic
  disciplines (literary and cultural studies), but other professional
  disciplines, such as so-called hard sciences, vary on attitude twoard
  1st-person remarks. On matter of 3rd-person critical voice, its use is
  not an excuse to bury your source. Statements about text and its
  cultural contexts or history of critical reading should be attributed
  to external sources, even if the source is something the professor
  said in class, is included in anthology introduction, or is posted on
  Wikipedia. In other words, the use of 3rd-person as your well-earned
  voice of critical authority (because you have done research) does not
  relieve responsibility to note sources for facts.
\item
  You are permitted to revise Paper 1 to improve the grade. Paper 2
  cannot be revised, but it is prepared in stages. The deadlines for
  interim stages (proposal, etc.) are actual paper deadlines with a
  consequence for missing the deadline. The purpose is not to be
  punitive but to ensure that you progress in multiple stages, the best
  recipe for ensuring that you write a stronger paper. You are welcome
  to send me a note with questions or to share drafts during office
  hours. But I will not pre-grade multi-page drafts by email, and I am
  only willing to approximate grade if office hour visit to discuss is
  more than 24 hours before the assignment is due. I will answer short
  email queries promptly, but I can offer only one or two comments by
  email on drafts up to 2 pages. If you wish for extended comments at a
  full-draft stage, an office-hour visit is required. The check-up draft
  (when requested) is not ``graded for content'' nor does missing it
  cause a paper grade deduction. It is a participation grade to ensure
  that you continue to make progress on the longer paper.
\end{itemize}

\subsection{Group Presentations}\label{group-presentations}

For a group presentation (3 or 4 students presenting a series of slides,
but see full guidelines) you will choose an author and work from the
anthology that is \textbf{not} assigned on the syllabus. You will
present a 10-minute presentation (with slides) that explains the
significance of the author and work how and why your chosen work could
fit within the parameters of the current course survey. Consider the
following questions: When and where was work written and/or published
(must match basic course parameters of geographical area, historical
period, language---to extent that course is defined by such parameters
in catalog)? What is the nature (genre) of the work? What strikes you as
important about the work and the writer? How is the work similar to
and/or different from (thematically, historically, generically) other
works that we have read (or will read) in class? The straightforward
option is to select for your presentation an author and work that is
included in the course-assigned anthology but is not assigned on the
syllabus (to check whether an author is assigned, search PDF syllabus
for names).

\subsection{Exams}\label{exams}

Two exams (IDs, quotations, essays) count toward 30\% of your grade.
Essay questions will be provided before the exam: one or two questions
may be take-home, and one question will be answered in class. If you
must take an exam at an alternate time (i.e., excused absence for
illness, bereavement, or university activity), the substitute exam may
be identical to or different from the in-class exam, at my discretion.

When an exam includes a take-home portion, a computer printed copy---not
an email message or a handwritten copy---is due in class. If you do not
have the printed copy at start of class, you can drop printed exam by
the envelope on my Satterfield 205c office within three hours after the
end of the exam. But you will receive a 20\% deduction on that portion
of exam. If a handwritten answer is submitted for a take-home, you will
receive a 20\% deduction on that portion of the exam. These two
deductions (late and handwritten take-home) are cumulative.

\subsection{Cheating and Plagiarism}\label{cheating-and-plagiarism}

By second week of class, I will post a Blackboard assignment in which
you affirm your familiarity with the university's cheating and
plagiarism policy and in which sanctions for cheating and plagiarism are
described. You must complete the assignment before you can earn credit
for class submissions.

For a violation on a minor assignment---if you cheat or plagiarize on a
quiz, take-home assignment, or blog assignment---you will receive a
permanent zero on the assignment, one which will be calculated with the
final average even if another higher grade is dropped. For cases of
possibly inadvertent misrepresentations (citation omitted, quotation
presented as paraphrase), you will be reminded of the importance by a
deduction of one letter grade to the assignment. A second violation on a
minor assignment will be treated as a serious violation.

The following violations are treated as serious violations. If you cheat
or plagiarize on an exam or paper (proposal or draft or final) or if you
submit falsified information to avoid penalties for late submission, you
must submit a full non-plagiarized version capable of earning a grade of
``B.'' But the grade you receive will be a ``0.'' If you fail to fulfill
the make-up requirement, you will automatically fail the course. For
cases of possibly inadvertent misrepresentations (citation page not
printed, quotation presented as paraphrase), you earn a permanent
deduction of one letter grade for the assignment.

I have generally found that detection of a single incident of plagiarism
requires further investigation of previous assignments. If I detect
plagiarism on one assignment, you will be asked to withdraw previous
assignments (grade changed to ``0'') or resubmit them for review.

For one serious violation or two minor violations, I will forward the
evidence to the department chair, have the charge added to your record
with the college, recommend further judicial sanction, and pursue the
case during the appeals process. As plagiarism accusation procedures
require an opportunity for student to offer defense, plagiarism
accusations on final exam or final paper may result in a grade of
incomplete until procedure can be completed during the following
semester.

\textbf{Note}: The university faculty senate has recommended an option
called plagiarism school for the first incident in which a student is
accused of plagiarism. If this course is your first incident, I will
recommend you to ``plagiarism school,'' which will be required. If you
have previously been accused of plagiarism or have attended plagiarism
school before, you are not eligible for it and shall face consequences
above.

\textbf{How Not to Plagiarize: } Amanda French has offered helpful
advice on impermissible copying, especially actions that constitute
plagiarism and copyright violation.
\textless{}\url{http://digitalpast002.onmason.com/syllabus/}\textgreater{}:

\begin{quote}
If you are copying and pasting text that someone else wrote, you might
be plagiarizing. Pasted or manually retyped text is not plagiarized only
when all of the following three conditions are true: 1) the pasted text
is surrounded by quotation marks or set off as a block quote, and 2) the
pasted text is attributed in your text to its author and its source
(e.g., ``As Jane Smith writes on her blog . . .''), and 3) the pasted
text is cited in a footnote, endnote, and/or a bibliography (e.g.,
``Smith, Jane. Smith Stuff. Blog. Available
\textless{}\url{http://smithstuff.wordpress.com}\textgreater{} Accessed
August 1, 2012.'') Conventions for copying and pasting computer code are
less strict, but even when you copy and paste code, if you can identify
the actual individual who wrote the code, you should give the coder's
name and the source of the code in a code comment. If you find and use
images, audio, or video on the web, you should also cite the creator (if
known) and the source (at the very least) of that media file, usually in
a caption as well as in a footnote, endnote, or bibliography. Note that
reproducing someone else's text, image, audio, or video file in full on
your own public website may constitute copyright infringement, even with
proper attribution.
\end{quote}

That everyone violates formal copyright now or that techvangelists or
corporate shills on Twitter or YouTube or Facebook or Instagram or
Pandora or Google endorse a culture of free sharing of copyrighted
content is not sufficient for you to escape the consequences of
plagiarism within this class. Times and laws change, but my demand that
you hold yourself to a high standard for ethical behavior is fully
within the realm of course policy. I am not qualified to give legal
advice on copyright, but I can advise sensible self-protection. When you
post material on a public web site, due diligence will help you defend
yourself against claims of copyright infringement. To exercise due
diligence, see Cornell University's ``Copyright Term and the Public
Domain in the United States''
\textless{}\url{http://copyright.cornell.edu/resources/publicdomain.cfm}\textgreater{}.
Thoreau, who called for civil disobedience, spent the night in jail. If
your violation of copyright is principled, I assume that failing a
college course assignment is a reasonable opportunity to test whether
you are truly devoted to your principles. If your copyright violations
are clear and in wanton disregard to guidelines, you will be assigned a
failing grade on assignment.

\subsection{Course Material Copyright}\label{course-material-copyright}

The university counsel (attorney's office) has notified professors that
students are selling course materials (presentations, handouts, notes,
exams, etc.) to an internet company. The company re-sells those
materials to subscribers. Selling course materials violates a
professor's copyright: the company is re-selling stolen intellectual
property. Course materials that I create and display or distribute to
students (unless they are owned by someone else and distributed under
fair use guidelines) are my intellectual property. Likewise, were I to
sell your work on a term paper web site, I would be violating your
copyright.

However, my course materials build on the work of other scholars.
Therefore, I claim what is known as an Attribution-NonCommercial License
(CC By-NC). See
\textless{}\url{http://creativecommons.org/licenses/}\textgreater{} for
details. In sum, you have permission to remix, tweak, or build upon my
work (for example, as a school lesson plan), but you must also release
your new remixed work (if it is substantially similar content) in
noncommercial form. If you create a derivative work (that is, you cite
me when creating something new, but yours is a substantially different
work), you do have permission to license your own work on a commercial
basis.

Please note that my course material copyright differs from standard
syllabus boilerplate that the university counsel recommends. Unless
another professor offers materials under a Creative Commons license, the
usual copyright rules apply for material from that professor.

\subsection{Credits}\label{credits}

Credit to other syllabi.

\section{Course Schedule}\label{course-schedule}

\subsection{Week 1: You do what?}\label{week-1-you-do-what}

Readings on what digital humanities, by scholars and by outside critics.

\begin{longtable}[c]{@{}ll@{}}
\toprule\addlinespace
\begin{minipage}[b]{0.18\columnwidth}\raggedright
Readings:
\end{minipage} & \begin{minipage}[b]{0.76\columnwidth}\raggedright
X X Marche, Stephen, ``Literature is not Data: Against Digital
Humanities'' X
\url{http://lareviewofbooks.org/essay/literature-is-not-data-against-digital-humanities}
X Kirsch, Adam, ``Technology Is Taking Over English Departments: The
false promise of the digital humanities'' X
\url{http://www.newrepublic.com/article/117428/limits-digital-humanities-adam-kirsch}
X X X X What is Digital Humanities: A Student Debate: Build a Blog and
eBook X Sullivan, Ian, ``Innovation in practice,'' Software Freedom Law
Center" X
\url{https://www.softwarefreedom.org/blog/2014/apr/11/innovation-in-practice/}
X Activity 1: Introduce Command Line X Activity 2: Introduce Plain Text
Editors
\end{minipage}
\\\addlinespace
\midrule\endhead
\begin{minipage}[t]{0.18\columnwidth}\raggedright
\#\#\# Week 1 Lecture Notes - Digital humanities i - betrayal of higher
- wrapped up with on - way to save money - making students mo - Digital
humanities i - creating scholarly - Scholar strategy i - Passing fad -
Unavoidable for sc - Lacks institutiona
\end{minipage} & \begin{minipage}[t]{0.76\columnwidth}\raggedright
n public calling line education (MOOCS, etc.)
\end{minipage}
\\\addlinespace
\begin{minipage}[t]{0.18\columnwidth}\raggedright
\% Test OxGarage Conversi \#\#\# Week 1 Goals
\end{minipage} & \begin{minipage}[t]{0.76\columnwidth}\raggedright
on on TEI-c site
\end{minipage}
\\\addlinespace
\begin{minipage}[t]{0.18\columnwidth}\raggedright
1. To learn open-source transforming them in and for publishing s and in
online forms 2. To develop interpers (Google Drive, DropB and
collaboratively 3. To develop a profess a web project and wi in publicly
accessib 4. To achieve a high le two of the following A) project
managemen B) installation and C) proficiency with D) proficiency with E)
automation of tex editors and regul F) acquiring and man G) Achieving
accurac representations o with OCR or trans H) hosting content m (Omeka,
Drupal, W 5. To become aware of t artifacts and benefi as access,
damaging qualities of origina 6. To achieve awareness public projects
have by diverse audiences
\end{minipage} & \begin{minipage}[t]{0.76\columnwidth}\raggedright
technologies for writing texts and to multiple formats (Pandoc and
Markdown) tudent-authored texts in print (LaTeX, PDF) (blogs, ebooks);
onal skills and communication techniques ox, GitHub) for addressing
systematically the challenges of group-based technology projects; ional
public identity that is associated with thclass materials (blog posts,
tweets) le online sites and forums. vel of proficiency in at least
processes or technologies: t and collaboration, use of open-source
software, LaTeX and bibliographical management XML and TEI encoding, t
manipulation with plain text ar expressions, aging digital images of
cultural objects y in text-based f literary or historical texts cription
and proofreading anagement systems on web servers ordPress) he balance
between challenges of reproducing cultural ts of digitization, on
matters such original artificats, or not fully representing l documents;
of the responsibilities that developers of for crediting sources and for
ensuring access .
\end{minipage}
\\\addlinespace
\begin{minipage}[t]{0.18\columnwidth}\raggedright
\#\# Week 2: Who are you?
\end{minipage} & \begin{minipage}[t]{0.76\columnwidth}\raggedright
\end{minipage}
\\\addlinespace
\begin{minipage}[t]{0.18\columnwidth}\raggedright
Readings on what digital barbarians at the gates
\end{minipage} & \begin{minipage}[t]{0.76\columnwidth}\raggedright
humanities, by scholars and by outside critics, narrative and the
destruction of the human.
\end{minipage}
\\\addlinespace
\begin{minipage}[t]{0.18\columnwidth}\raggedright
Stephen Ramsay, ``Toward
\end{minipage} & \begin{minipage}[t]{0.76\columnwidth}\raggedright
an Algorithmic Criticism"
\url{http://dho.ie/sites/default/files/Toward_an_Algorithmic_Criticism.pdf}
\end{minipage}
\\\addlinespace
\bottomrule
\end{longtable}

\begin{verbatim}
                    X
\end{verbatim}

Readings: X Matthew Kirschenbam, ``What is Digital Humanities and What's
It Doing in English Departments?'' X
\url{http://mkirschenbaum.files.wordpress.com/2011/01/kirschenbaum_ade150.pdf}
X Julia Flanders, ``The productive unease of 21st-century digital
scholarship
\url{http://www.digitalhumanities.org/dhq/vol/3/3/000055/000055.html} X
Stephen Ramsay,''On Building"
\url{http://stephenramsay.us/text/2011/01/11/on-building/} X Mark
Sample, ``The digital humanities is not about building, it's about
sharing''
\url{http://www.samplereality.com/2011/05/25/the-digital-humanities-is-not-about-building-its-about-sharing/}
X Assignment: X Activity 1: Install Pandoc and LaTeX X Activity 2:
Introduce Zotero and Bib files X Activity 3: Convert Sample Markdown
File w/ bib to Word RTF and LaTeX X Activity 4: Write blog post with
bibliographical reference on what DH is ----------------
-------------------------------------------------------------------------------------------------------

\subsubsection{Week 2 Lecture Notes}\label{week-2-lecture-notes}

\begin{itemize}
\itemsep1pt\parskip0pt\parsep0pt
\item
  Digital humanities in English Departments

  \begin{itemize}
  \itemsep1pt\parskip0pt\parsep0pt
  \item
    Making people nervous about technology competence
  \item
    Way to blend literary scholarship and rhetoric and composition
  \item
    Archival and editorial work
  \end{itemize}
\item
  Digital humanities in larger culture

  \begin{itemize}
  \item
    Online access as community good
  \item
    Open-source software
  \item
  \end{itemize}
\end{itemize}

\subsubsection{Week 2 Goals}\label{week-2-goals}

\begin{enumerate}
\def\labelenumi{\arabic{enumi}.}
\itemsep1pt\parskip0pt\parsep0pt
\item
  To learn open-source technologies for writing texts and transforming
  them into multiple formats (Pandoc and Markdown) and for publishing
  student-authored texts in print (LaTeX, PDF) and in online forms
  (blogs, ebooks);
\item
  To develop interpersonal skills and communication techniques (Google
  Drive, DropBox, GitHub) for addressing systematically and
  collaboratively the challenges of group-based technology projects;
\item
  To develop a professional public identity that is associated with a
  web project and withclass materials (blog posts, tweets) in publicly
  accessible online sites and forums.
\item
  To achieve a high level of proficiency in at least two of the
  following processes or technologies:

  \begin{enumerate}
  \def\labelenumii{\Alph{enumii})}
  \itemsep1pt\parskip0pt\parsep0pt
  \item
    project management and collaboration,
  \item
    installation and use of open-source software,
  \item
    proficiency with LaTeX and bibliographical management
  \item
    proficiency with XML and TEI encoding,
  \item
    automation of text manipulation with plain text editors and regular
    expressions,
  \item
    acquiring and managing digital images of cultural objects
  \item
    Achieving accuracy in text-based representations of literary or
    historical texts with OCR or transcription and proofreading
  \item
    hosting content management systems on web servers (Omeka, Drupal,
    WordPress)
  \end{enumerate}
\item
  To become aware of the balance between challenges of reproducing
  cultural artifacts and benefits of digitization, on matters such as
  access, damaging original artificats, or not fully representing
  qualities of original documents;
\item
  To achieve awareness of the responsibilities that developers of public
  projects have for crediting sources and for ensuring access by diverse
  audiences.
\end{enumerate}

\subsection{Week 3: What is text,
really?}\label{week-3-what-is-text-really}

\begin{longtable}[c]{@{}ll@{}}
\toprule\addlinespace
& X
\\\addlinespace
Readings: & X
\\\addlinespace
& X
\\\addlinespace
& X
\\\addlinespace
& X
\\\addlinespace
& X ``Build a Digital Book with ePub''
\\\addlinespace
& X
\url{http://www.ibm.com/developerworks/xml/tutorials/x-epubtut/index.html?ca=drs-}
\\\addlinespace
& X
\\\addlinespace
& X
\\\addlinespace
Assignment: & X Identify 3 Collaborators
\\\addlinespace
& X Install DropBox and GitHub
\\\addlinespace
& X Install Calibre
\\\addlinespace
\bottomrule
\end{longtable}

\subsubsection{Week 3 Goals}\label{week-3-goals}

\begin{enumerate}
\def\labelenumi{\arabic{enumi}.}
\itemsep1pt\parskip0pt\parsep0pt
\item
  To learn open-source technologies for writing texts and transforming
  them into multiple formats (Pandoc and Markdown) and for publishing
  student-authored texts in print (LaTeX, PDF) and in online forms
  (blogs, ebooks);
\item
  To develop interpersonal skills and communication techniques (Google
  Drive, DropBox, GitHub) for addressing systematically and
  collaboratively the challenges of group-based technology projects;
\item
  To develop a professional public identity that is associated with a
  web project and withclass materials (blog posts, tweets) in publicly
  accessible online sites and forums.
\item
  To achieve a high level of proficiency in at least two of the
  following processes or technologies:

  \begin{enumerate}
  \def\labelenumii{\Alph{enumii})}
  \itemsep1pt\parskip0pt\parsep0pt
  \item
    project management and collaboration,
  \item
    installation and use of open-source software,
  \item
    proficiency with LaTeX and bibliographical management
  \item
    proficiency with XML and TEI encoding,
  \item
    automation of text manipulation with plain text editors and regular
    expressions,
  \item
    acquiring and managing digital images of cultural objects
  \item
    Achieving accuracy in text-based representations of literary or
    historical texts with OCR or transcription and proofreading
  \item
    hosting content management systems on web servers (Omeka, Drupal,
    WordPress)
  \end{enumerate}
\item
  To become aware of the balance between challenges of reproducing
  cultural artifacts and benefits of digitization, on matters such as
  access, damaging original artificats, or not fully representing
  qualities of original documents;
\item
  To achieve awareness of the responsibilities that developers of public
  projects have for crediting sources and for ensuring access by diverse
  audiences.
\end{enumerate}

\subsection{Week 4: Who am I?}\label{week-4-who-am-i}

\begin{longtable}[c]{@{}ll@{}}
\toprule\addlinespace
& X
\\\addlinespace
Readings: & X
\\\addlinespace
& X
\\\addlinespace
& X
\\\addlinespace
& X
\\\addlinespace
& X
\\\addlinespace
& X
\\\addlinespace
& X
\\\addlinespace
Assignment: & X Post MarkDown source and LaTeX source to GitHub, and
post link to ePub eBook on DropBox
\\\addlinespace
& X Submit LaTeX-generated print copy and PDF
\\\addlinespace
& X Create Twitter account (class or personal) and designate hash tag
\\\addlinespace
\bottomrule
\end{longtable}

\subsubsection{Week 4 Goals}\label{week-4-goals}

\begin{enumerate}
\def\labelenumi{\arabic{enumi}.}
\itemsep1pt\parskip0pt\parsep0pt
\item
  To learn open-source technologies for writing texts and transforming
  them into multiple formats (Pandoc and Markdown) and for publishing
  student-authored texts in print (LaTeX, PDF) and in online forms
  (blogs, ebooks);
\item
  To develop interpersonal skills and communication techniques (Google
  Drive, DropBox, GitHub) for addressing systematically and
  collaboratively the challenges of group-based technology projects;
\item
  To develop a professional public identity that is associated with a
  web project and withclass materials (blog posts, tweets) in publicly
  accessible online sites and forums.
\item
  To achieve a high level of proficiency in at least two of the
  following processes or technologies:

  \begin{enumerate}
  \def\labelenumii{\Alph{enumii})}
  \itemsep1pt\parskip0pt\parsep0pt
  \item
    project management and collaboration,
  \item
    installation and use of open-source software,
  \item
    proficiency with LaTeX and bibliographical management
  \item
    proficiency with XML and TEI encoding,
  \item
    automation of text manipulation with plain text editors and regular
    expressions,
  \item
    acquiring and managing digital images of cultural objects
  \item
    Achieving accuracy in text-based representations of literary or
    historical texts with OCR or transcription and proofreading
  \item
    hosting content management systems on web servers (Omeka, Drupal,
    WordPress)
  \end{enumerate}
\item
  To become aware of the balance between challenges of reproducing
  cultural artifacts and benefits of digitization, on matters such as
  access, damaging original artificats, or not fully representing
  qualities of original documents;
\item
  To achieve awareness of the responsibilities that developers of public
  projects have for crediting sources and for ensuring access by diverse
  audiences.
\end{enumerate}

\subsection{Week 5: Searching for lost
arks}\label{week-5-searching-for-lost-arks}

\begin{longtable}[c]{@{}ll@{}}
\toprule\addlinespace
& X
\\\addlinespace
Readings: & X
\\\addlinespace
& X
\\\addlinespace
& X
\\\addlinespace
& X
\\\addlinespace
& X
\\\addlinespace
& X
\\\addlinespace
& X
\\\addlinespace
Assignment: & X Visit library special collections
\\\addlinespace
& X Set up Reclaim Hosting and Install Omeka and WordPress
\\\addlinespace
\bottomrule
\end{longtable}

\subsubsection{Week 5 Goals}\label{week-5-goals}

\begin{enumerate}
\def\labelenumi{\arabic{enumi}.}
\itemsep1pt\parskip0pt\parsep0pt
\item
  To learn open-source technologies for writing texts and transforming
  them into multiple formats (Pandoc and Markdown) and for publishing
  student-authored texts in print (LaTeX, PDF) and in online forms
  (blogs, ebooks);
\item
  To develop interpersonal skills and communication techniques (Google
  Drive, DropBox, GitHub) for addressing systematically and
  collaboratively the challenges of group-based technology projects;
\item
  To develop a professional public identity that is associated with a
  web project and withclass materials (blog posts, tweets) in publicly
  accessible online sites and forums.
\item
  To achieve a high level of proficiency in at least two of the
  following processes or technologies:

  \begin{enumerate}
  \def\labelenumii{\Alph{enumii})}
  \itemsep1pt\parskip0pt\parsep0pt
  \item
    project management and collaboration,
  \item
    installation and use of open-source software,
  \item
    proficiency with LaTeX and bibliographical management
  \item
    proficiency with XML and TEI encoding,
  \item
    automation of text manipulation with plain text editors and regular
    expressions,
  \item
    acquiring and managing digital images of cultural objects
  \item
    Achieving accuracy in text-based representations of literary or
    historical texts with OCR or transcription and proofreading
  \item
    hosting content management systems on web servers (Omeka, Drupal,
    WordPress)
  \end{enumerate}
\item
  To become aware of the balance between challenges of reproducing
  cultural artifacts and benefits of digitization, on matters such as
  access, damaging original artificats, or not fully representing
  qualities of original documents;
\item
  To achieve awareness of the responsibilities that developers of public
  projects have for crediting sources and for ensuring access by diverse
  audiences.
\end{enumerate}

\subsection{Week 6: Back to the archive
again}\label{week-6-back-to-the-archive-again}

\begin{longtable}[c]{@{}ll@{}}
\toprule\addlinespace
& X
\\\addlinespace
Readings: & X
\\\addlinespace
& X
\\\addlinespace
& X Daniel J. Cohen and Roy Rosenszwig, ``How to Make Text Digital:
Scanning, OCR, and Typing,'' \emph{Digital History,}
\\\addlinespace
& X \url{http://chnm.gmu.edu/digitalhistory/digitizing/4.php}
\\\addlinespace
& X Daniel J. Cohen and Roy Rosenszwig, ``Digital Images'' \emph{Digital
History,}
\\\addlinespace
& X \url{http://chnm.gmu.edu/digitalhistory/digitizing/5.php}
\\\addlinespace
& X
\\\addlinespace
Assignment: & X Transcribe Text
\\\addlinespace
& X Install Tesseract and OCR print text
\\\addlinespace
& X
\\\addlinespace
\bottomrule
\end{longtable}

\subsubsection{Week 6 Goals}\label{week-6-goals}

\begin{enumerate}
\def\labelenumi{\arabic{enumi}.}
\itemsep1pt\parskip0pt\parsep0pt
\item
  To learn open-source technologies for writing texts and transforming
  them into multiple formats (Pandoc and Markdown) and for publishing
  student-authored texts in print (LaTeX, PDF) and in online forms
  (blogs, ebooks);
\item
  To develop interpersonal skills and communication techniques (Google
  Drive, DropBox, GitHub) for addressing systematically and
  collaboratively the challenges of group-based technology projects;
\item
  To develop a professional public identity that is associated with a
  web project and withclass materials (blog posts, tweets) in publicly
  accessible online sites and forums.
\item
  To achieve a high level of proficiency in at least two of the
  following processes or technologies:

  \begin{enumerate}
  \def\labelenumii{\Alph{enumii})}
  \itemsep1pt\parskip0pt\parsep0pt
  \item
    project management and collaboration,
  \item
    installation and use of open-source software,
  \item
    proficiency with LaTeX and bibliographical management
  \item
    proficiency with XML and TEI encoding,
  \item
    automation of text manipulation with plain text editors and regular
    expressions,
  \item
    acquiring and managing digital images of cultural objects
  \item
    Achieving accuracy in text-based representations of literary or
    historical texts with OCR or transcription and proofreading
  \item
    hosting content management systems on web servers (Omeka, Drupal,
    WordPress)
  \end{enumerate}
\item
  To become aware of the balance between challenges of reproducing
  cultural artifacts and benefits of digitization, on matters such as
  access, damaging original artificats, or not fully representing
  qualities of original documents;
\item
  To achieve awareness of the responsibilities that developers of public
  projects have for crediting sources and for ensuring access by diverse
  audiences.
\end{enumerate}

\subsection{Week 7: Pictures are worth a 1000
words}\label{week-7-pictures-are-worth-a-1000-words}

\begin{longtable}[c]{@{}ll@{}}
\toprule\addlinespace
\begin{minipage}[t]{0.18\columnwidth}\raggedright
Readings:
\end{minipage} & \begin{minipage}[t]{0.76\columnwidth}\raggedright
X X
\end{minipage}
\\\addlinespace
\begin{minipage}[t]{0.18\columnwidth}\raggedright
\end{minipage} & \begin{minipage}[t]{0.76\columnwidth}\raggedright
X Klenczon \& Rygiel, ``Librarian Cornered by Images, or How to Index
Visual Resources'' \emph{Cataloging \& Classification Quarterly} X 52:1
(2014): 1-21. X X X X X Compare Transcribed and OCR Text in JUXTA X
Orally proofread transcribed text against original document X Introduce
REGEX X Read Introduction to XML
\end{minipage}
\\\addlinespace
\bottomrule
\end{longtable}

\subsubsection{Week 7 Goals}\label{week-7-goals}

\begin{enumerate}
\def\labelenumi{\arabic{enumi}.}
\itemsep1pt\parskip0pt\parsep0pt
\item
  To learn open-source technologies for writing texts and transforming
  them into multiple formats (Pandoc and Markdown) and for publishing
  student-authored texts in print (LaTeX, PDF) and in online forms
  (blogs, ebooks);
\item
  To develop interpersonal skills and communication techniques (Google
  Drive, DropBox, GitHub) for addressing systematically and
  collaboratively the challenges of group-based technology projects;
\item
  To develop a professional public identity that is associated with a
  web project and withclass materials (blog posts, tweets) in publicly
  accessible online sites and forums.
\item
  To achieve a high level of proficiency in at least two of the
  following processes or technologies:

  \begin{enumerate}
  \def\labelenumii{\Alph{enumii})}
  \itemsep1pt\parskip0pt\parsep0pt
  \item
    project management and collaboration,
  \item
    installation and use of open-source software,
  \item
    proficiency with LaTeX and bibliographical management
  \item
    proficiency with XML and TEI encoding,
  \item
    automation of text manipulation with plain text editors and regular
    expressions,
  \item
    acquiring and managing digital images of cultural objects
  \item
    Achieving accuracy in text-based representations of literary or
    historical texts with OCR or transcription and proofreading
  \item
    hosting content management systems on web servers (Omeka, Drupal,
    WordPress)
  \end{enumerate}
\item
  To become aware of the balance between challenges of reproducing
  cultural artifacts and benefits of digitization, on matters such as
  access, damaging original artificats, or not fully representing
  qualities of original documents;
\item
  To achieve awareness of the responsibilities that developers of public
  projects have for crediting sources and for ensuring access by diverse
  audiences.
\end{enumerate}

\subsection{Week 8: Text pictures are
words}\label{week-8-text-pictures-are-words}

\begin{longtable}[c]{@{}ll@{}}
\toprule\addlinespace
& X
\\\addlinespace
Readings: & X
\\\addlinespace
& X
\\\addlinespace
& X
\\\addlinespace
& X Kline, Mary-Jo and Susan Holbrook Perdue, \emph{A Guide to
Documentary Editing}
\\\addlinespace
& X Sections from ``Transcribing the Source Text'': Sections I--IV and V
A:1, Handwritten Source, and V B:1, Correspondence
\\\addlinespace
& X
(http://gde.upress.virginia.edu/04-gde.html\#h2.5){[}http://gde.upress.virginia.edu/04-gde.html\#h2.5{]}
\\\addlinespace
& X
\\\addlinespace
& X
\\\addlinespace
& X
\\\addlinespace
Assignment: & X Introduce XML, TEILite, XSLT, and Saxon
\\\addlinespace
& X Install Drupal and TEIChI
\\\addlinespace
& X Install Omeka and
\\\addlinespace
\bottomrule
\end{longtable}

\subsubsection{Week 8 Goals}\label{week-8-goals}

\begin{enumerate}
\def\labelenumi{\arabic{enumi}.}
\itemsep1pt\parskip0pt\parsep0pt
\item
  To learn open-source technologies for writing texts and transforming
  them into multiple formats (Pandoc and Markdown) and for publishing
  student-authored texts in print (LaTeX, PDF) and in online forms
  (blogs, ebooks);
\item
  To develop interpersonal skills and communication techniques (Google
  Drive, DropBox, GitHub) for addressing systematically and
  collaboratively the challenges of group-based technology projects;
\item
  To develop a professional public identity that is associated with a
  web project and withclass materials (blog posts, tweets) in publicly
  accessible online sites and forums.
\item
  To achieve a high level of proficiency in at least two of the
  following processes or technologies:

  \begin{enumerate}
  \def\labelenumii{\Alph{enumii})}
  \itemsep1pt\parskip0pt\parsep0pt
  \item
    project management and collaboration,
  \item
    installation and use of open-source software,
  \item
    proficiency with LaTeX and bibliographical management
  \item
    proficiency with XML and TEI encoding,
  \item
    automation of text manipulation with plain text editors and regular
    expressions,
  \item
    acquiring and managing digital images of cultural objects
  \item
    Achieving accuracy in text-based representations of literary or
    historical texts with OCR or transcription and proofreading
  \item
    hosting content management systems on web servers (Omeka, Drupal,
    WordPress)
  \end{enumerate}
\item
  To become aware of the balance between challenges of reproducing
  cultural artifacts and benefits of digitization, on matters such as
  access, damaging original artificats, or not fully representing
  qualities of original documents;
\item
  To achieve awareness of the responsibilities that developers of public
  projects have for crediting sources and for ensuring access by diverse
  audiences.
\end{enumerate}

\subsection{Week 9: What is text,
really?}\label{week-9-what-is-text-really}

\begin{longtable}[c]{@{}ll@{}}
\toprule\addlinespace
& X
\\\addlinespace
Readings: & X
\\\addlinespace
& X
\\\addlinespace
& X Daniel J. Cohen and
\\\addlinespace
& X
(http://chnm.gmu.edu/digitalhistory/digitizing/3.php){[}http://chnm.gmu.edu/digitalhistory/digitizing/3.php{]}
\\\addlinespace
& X
\\\addlinespace
& X
\\\addlinespace
& X
\\\addlinespace
Assignment: & X Propose Project Plan
\\\addlinespace
& X Install Drupal and TEIChI \emph{or} Install Omeka and Scripto
\emph{or} Install Version Machine
\\\addlinespace
& X
\\\addlinespace
\bottomrule
\end{longtable}

\subsubsection{Week 9 Goals}\label{week-9-goals}

\begin{enumerate}
\def\labelenumi{\arabic{enumi}.}
\itemsep1pt\parskip0pt\parsep0pt
\item
  To learn open-source technologies for writing texts and transforming
  them into multiple formats (Pandoc and Markdown) and for publishing
  student-authored texts in print (LaTeX, PDF) and in online forms
  (blogs, ebooks);
\item
  To develop interpersonal skills and communication techniques (Google
  Drive, DropBox, GitHub) for addressing systematically and
  collaboratively the challenges of group-based technology projects;
\item
  To develop a professional public identity that is associated with a
  web project and withclass materials (blog posts, tweets) in publicly
  accessible online sites and forums.
\item
  To achieve a high level of proficiency in at least two of the
  following processes or technologies:

  \begin{enumerate}
  \def\labelenumii{\Alph{enumii})}
  \itemsep1pt\parskip0pt\parsep0pt
  \item
    project management and collaboration,
  \item
    installation and use of open-source software,
  \item
    proficiency with LaTeX and bibliographical management
  \item
    proficiency with XML and TEI encoding,
  \item
    automation of text manipulation with plain text editors and regular
    expressions,
  \item
    acquiring and managing digital images of cultural objects
  \item
    Achieving accuracy in text-based representations of literary or
    historical texts with OCR or transcription and proofreading
  \item
    hosting content management systems on web servers (Omeka, Drupal,
    WordPress)
  \end{enumerate}
\item
  To become aware of the balance between challenges of reproducing
  cultural artifacts and benefits of digitization, on matters such as
  access, damaging original artificats, or not fully representing
  qualities of original documents;
\item
  To achieve awareness of the responsibilities that developers of public
  projects have for crediting sources and for ensuring access by diverse
  audiences.
\end{enumerate}

\subsection{Week 10: All hang together, or all hang
separately}\label{week-10-all-hang-together-or-all-hang-separately}

\begin{longtable}[c]{@{}ll@{}}
\toprule\addlinespace
& X
\\\addlinespace
Readings: & X
\\\addlinespace
& X
\\\addlinespace
& X
\\\addlinespace
& X
\\\addlinespace
& X
\\\addlinespace
& X
\\\addlinespace
& X
\\\addlinespace
Assignment: & X
\\\addlinespace
& X
\\\addlinespace
& X
\\\addlinespace
\bottomrule
\end{longtable}

\subsubsection{Week 10 Goals: Pick a site, any
site.}\label{week-10-goals-pick-a-site-any-site.}

\begin{enumerate}
\def\labelenumi{\arabic{enumi}.}
\itemsep1pt\parskip0pt\parsep0pt
\item
  To learn open-source technologies for writing texts and transforming
  them into multiple formats (Pandoc and Markdown) and for publishing
  student-authored texts in print (LaTeX, PDF) and in online forms
  (blogs, ebooks);
\item
  To develop interpersonal skills and communication techniques (Google
  Drive, DropBox, GitHub) for addressing systematically and
  collaboratively the challenges of group-based technology projects;
\item
  To develop a professional public identity that is associated with a
  web project and withclass materials (blog posts, tweets) in publicly
  accessible online sites and forums.
\item
  To achieve a high level of proficiency in at least two of the
  following processes or technologies:

  \begin{enumerate}
  \def\labelenumii{\Alph{enumii})}
  \itemsep1pt\parskip0pt\parsep0pt
  \item
    project management and collaboration,
  \item
    installation and use of open-source software,
  \item
    proficiency with LaTeX and bibliographical management
  \item
    proficiency with XML and TEI encoding,
  \item
    automation of text manipulation with plain text editors and regular
    expressions,
  \item
    acquiring and managing digital images of cultural objects
  \item
    Achieving accuracy in text-based representations of literary or
    historical texts with OCR or transcription and proofreading
  \item
    hosting content management systems on web servers (Omeka, Drupal,
    WordPress)
  \end{enumerate}
\item
  To become aware of the balance between challenges of reproducing
  cultural artifacts and benefits of digitization, on matters such as
  access, damaging original artificats, or not fully representing
  qualities of original documents;
\item
  To achieve awareness of the responsibilities that developers of public
  projects have for crediting sources and for ensuring access by diverse
  audiences.
\end{enumerate}

\subsection{Week 11: Gathering our
wares}\label{week-11-gathering-our-wares}

\begin{longtable}[c]{@{}ll@{}}
\toprule\addlinespace
& X
\\\addlinespace
Readings: & X
\\\addlinespace
& X
\\\addlinespace
& X
\\\addlinespace
& X
\\\addlinespace
& X
\\\addlinespace
& X
\\\addlinespace
& X
\\\addlinespace
Assignment: & X
\\\addlinespace
& X
\\\addlinespace
& X
\\\addlinespace
\bottomrule
\end{longtable}

\subsubsection{Week 11 Goals}\label{week-11-goals}

\begin{enumerate}
\def\labelenumi{\arabic{enumi}.}
\itemsep1pt\parskip0pt\parsep0pt
\item
  To learn open-source technologies for writing texts and transforming
  them into multiple formats (Pandoc and Markdown) and for publishing
  student-authored texts in print (LaTeX, PDF) and in online forms
  (blogs, ebooks);
\item
  To develop interpersonal skills and communication techniques (Google
  Drive, DropBox, GitHub) for addressing systematically and
  collaboratively the challenges of group-based technology projects;
\item
  To develop a professional public identity that is associated with a
  web project and withclass materials (blog posts, tweets) in publicly
  accessible online sites and forums.
\item
  To achieve a high level of proficiency in at least two of the
  following processes or technologies:

  \begin{enumerate}
  \def\labelenumii{\Alph{enumii})}
  \itemsep1pt\parskip0pt\parsep0pt
  \item
    project management and collaboration,
  \item
    installation and use of open-source software,
  \item
    proficiency with LaTeX and bibliographical management
  \item
    proficiency with XML and TEI encoding,
  \item
    automation of text manipulation with plain text editors and regular
    expressions,
  \item
    acquiring and managing digital images of cultural objects
  \item
    Achieving accuracy in text-based representations of literary or
    historical texts with OCR or transcription and proofreading
  \item
    hosting content management systems on web servers (Omeka, Drupal,
    WordPress)
  \end{enumerate}
\item
  To become aware of the balance between challenges of reproducing
  cultural artifacts and benefits of digitization, on matters such as
  access, damaging original artificats, or not fully representing
  qualities of original documents;
\item
  To achieve awareness of the responsibilities that developers of public
  projects have for crediting sources and for ensuring access by diverse
  audiences.
\end{enumerate}

\subsection{Week 12: If we build it, will they
come?}\label{week-12-if-we-build-it-will-they-come}

\begin{longtable}[c]{@{}ll@{}}
\toprule\addlinespace
& X
\\\addlinespace
Readings: & X
\\\addlinespace
& X
\\\addlinespace
& X
\\\addlinespace
& X
\\\addlinespace
& X
\\\addlinespace
& X
\\\addlinespace
& X
\\\addlinespace
Assignment: & X
\\\addlinespace
& X
\\\addlinespace
& X
\\\addlinespace
\bottomrule
\end{longtable}

\subsubsection{Week 12 Goals}\label{week-12-goals}

\begin{enumerate}
\def\labelenumi{\arabic{enumi}.}
\itemsep1pt\parskip0pt\parsep0pt
\item
  To learn open-source technologies for writing texts and transforming
  them into multiple formats (Pandoc and Markdown) and for publishing
  student-authored texts in print (LaTeX, PDF) and in online forms
  (blogs, ebooks);
\item
  To develop interpersonal skills and communication techniques (Google
  Drive, DropBox, GitHub) for addressing systematically and
  collaboratively the challenges of group-based technology projects;
\item
  To develop a professional public identity that is associated with a
  web project and withclass materials (blog posts, tweets) in publicly
  accessible online sites and forums.
\item
  To achieve a high level of proficiency in at least two of the
  following processes or technologies:

  \begin{enumerate}
  \def\labelenumii{\Alph{enumii})}
  \itemsep1pt\parskip0pt\parsep0pt
  \item
    project management and collaboration,
  \item
    installation and use of open-source software,
  \item
    proficiency with LaTeX and bibliographical management
  \item
    proficiency with XML and TEI encoding,
  \item
    automation of text manipulation with plain text editors and regular
    expressions,
  \item
    acquiring and managing digital images of cultural objects
  \item
    Achieving accuracy in text-based representations of literary or
    historical texts with OCR or transcription and proofreading
  \item
    hosting content management systems on web servers (Omeka, Drupal,
    WordPress)
  \end{enumerate}
\item
  To become aware of the balance between challenges of reproducing
  cultural artifacts and benefits of digitization, on matters such as
  access, damaging original artificats, or not fully representing
  qualities of original documents;
\item
  To achieve awareness of the responsibilities that developers of public
  projects have for crediting sources and for ensuring access by diverse
  audiences.
\end{enumerate}

\subsection{Week 13: Who is our
audience?}\label{week-13-who-is-our-audience}

\begin{longtable}[c]{@{}ll@{}}
\toprule\addlinespace
& X
\\\addlinespace
Readings: & X
\\\addlinespace
& X
\\\addlinespace
& X
\\\addlinespace
& X
\\\addlinespace
& X
\\\addlinespace
& X
\\\addlinespace
& X
\\\addlinespace
Assignment: & X
\\\addlinespace
& X
\\\addlinespace
& X
\\\addlinespace
\bottomrule
\end{longtable}

\subsubsection{Week 13 Goals}\label{week-13-goals}

\begin{enumerate}
\def\labelenumi{\arabic{enumi}.}
\itemsep1pt\parskip0pt\parsep0pt
\item
  To learn open-source technologies for writing texts and transforming
  them into multiple formats (Pandoc and Markdown) and for publishing
  student-authored texts in print (LaTeX, PDF) and in online forms
  (blogs, ebooks);
\item
  To develop interpersonal skills and communication techniques (Google
  Drive, DropBox, GitHub) for addressing systematically and
  collaboratively the challenges of group-based technology projects;
\item
  To develop a professional public identity that is associated with a
  web project and withclass materials (blog posts, tweets) in publicly
  accessible online sites and forums.
\item
  To achieve a high level of proficiency in at least two of the
  following processes or technologies:

  \begin{enumerate}
  \def\labelenumii{\Alph{enumii})}
  \itemsep1pt\parskip0pt\parsep0pt
  \item
    project management and collaboration,
  \item
    installation and use of open-source software,
  \item
    proficiency with LaTeX and bibliographical management
  \item
    proficiency with XML and TEI encoding,
  \item
    automation of text manipulation with plain text editors and regular
    expressions,
  \item
    acquiring and managing digital images of cultural objects
  \item
    Achieving accuracy in text-based representations of literary or
    historical texts with OCR or transcription and proofreading
  \item
    hosting content management systems on web servers (Omeka, Drupal,
    WordPress)
  \end{enumerate}
\item
  To become aware of the balance between challenges of reproducing
  cultural artifacts and benefits of digitization, on matters such as
  access, damaging original artificats, or not fully representing
  qualities of original documents;
\item
  To achieve awareness of the responsibilities that developers of public
  projects have for crediting sources and for ensuring access by diverse
  audiences.
\end{enumerate}

\subsection{Week 14: Is this legal?}\label{week-14-is-this-legal}

\begin{longtable}[c]{@{}ll@{}}
\toprule\addlinespace
& X
\\\addlinespace
Readings: & X
\\\addlinespace
& X
\\\addlinespace
& X
\\\addlinespace
& X
\\\addlinespace
& X
\\\addlinespace
& X
\\\addlinespace
& X
\\\addlinespace
Assignment: & X
\\\addlinespace
& X
\\\addlinespace
& X
\\\addlinespace
\bottomrule
\end{longtable}

\subsubsection{Week 14 Goals}\label{week-14-goals}

\begin{enumerate}
\def\labelenumi{\arabic{enumi}.}
\itemsep1pt\parskip0pt\parsep0pt
\item
  To learn open-source technologies for writing texts and transforming
  them into multiple formats (Pandoc and Markdown) and for publishing
  student-authored texts in print (LaTeX, PDF) and in online forms
  (blogs, ebooks);
\item
  To develop interpersonal skills and communication techniques (Google
  Drive, DropBox, GitHub) for addressing systematically and
  collaboratively the challenges of group-based technology projects;
\item
  To develop a professional public identity that is associated with a
  web project and withclass materials (blog posts, tweets) in publicly
  accessible online sites and forums.
\item
  To achieve a high level of proficiency in at least two of the
  following processes or technologies:

  \begin{enumerate}
  \def\labelenumii{\Alph{enumii})}
  \itemsep1pt\parskip0pt\parsep0pt
  \item
    project management and collaboration,
  \item
    installation and use of open-source software,
  \item
    proficiency with LaTeX and bibliographical management
  \item
    proficiency with XML and TEI encoding,
  \item
    automation of text manipulation with plain text editors and regular
    expressions,
  \item
    acquiring and managing digital images of cultural objects
  \item
    Achieving accuracy in text-based representations of literary or
    historical texts with OCR or transcription and proofreading
  \item
    hosting content management systems on web servers (Omeka, Drupal,
    WordPress)
  \end{enumerate}
\item
  To become aware of the balance between challenges of reproducing
  cultural artifacts and benefits of digitization, on matters such as
  access, damaging original artificats, or not fully representing
  qualities of original documents;
\item
  To achieve awareness of the responsibilities that developers of public
  projects have for crediting sources and for ensuring access by diverse
  audiences.
\end{enumerate}

\subsection{Week 15: Sharing our wares}\label{week-15-sharing-our-wares}

\begin{longtable}[c]{@{}ll@{}}
\toprule\addlinespace
& X
\\\addlinespace
Readings: & X
\\\addlinespace
& X
\\\addlinespace
& X
\\\addlinespace
& X
\\\addlinespace
& X
\\\addlinespace
& X
\\\addlinespace
& X
\\\addlinespace
Assignment: & X
\\\addlinespace
& X
\\\addlinespace
& X
\\\addlinespace
\bottomrule
\end{longtable}

\subsubsection{Week 15 Goals}\label{week-15-goals}

\begin{enumerate}
\def\labelenumi{\arabic{enumi}.}
\itemsep1pt\parskip0pt\parsep0pt
\item
  To learn open-source technologies for writing texts and transforming
  them into multiple formats (Pandoc and Markdown) and for publishing
  student-authored texts in print (LaTeX, PDF) and in online forms
  (blogs, ebooks);
\item
  To develop interpersonal skills and communication techniques (Google
  Drive, DropBox, GitHub) for addressing systematically and
  collaboratively the challenges of group-based technology projects;
\item
  To develop a professional public identity that is associated with a
  web project and withclass materials (blog posts, tweets) in publicly
  accessible online sites and forums.
\item
  To achieve a high level of proficiency in at least two of the
  following processes or technologies:

  \begin{enumerate}
  \def\labelenumii{\Alph{enumii})}
  \itemsep1pt\parskip0pt\parsep0pt
  \item
    project management and collaboration,
  \item
    installation and use of open-source software,
  \item
    proficiency with LaTeX and bibliographical management
  \item
    proficiency with XML and TEI encoding,
  \item
    automation of text manipulation with plain text editors and regular
    expressions,
  \item
    acquiring and managing digital images of cultural objects
  \item
    Achieving accuracy in text-based representations of literary or
    historical texts with OCR or transcription and proofreading
  \item
    hosting content management systems on web servers (Omeka, Drupal,
    WordPress)
  \end{enumerate}
\item
  To become aware of the balance between challenges of reproducing
  cultural artifacts and benefits of digitization, on matters such as
  access, damaging original artificats, or not fully representing
  qualities of original documents;
\item
  To achieve awareness of the responsibilities that developers of public
  projects have for crediting sources and for ensuring access by diverse
  audiences.
\end{enumerate}

\end{document}
