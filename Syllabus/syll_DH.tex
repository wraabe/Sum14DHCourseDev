\documentclass[]{article}
\usepackage{lmodern}
\usepackage{amssymb,amsmath}
\usepackage{ifxetex,ifluatex}
\usepackage{fixltx2e} % provides \textsubscript
\ifnum 0\ifxetex 1\fi\ifluatex 1\fi=0 % if pdftex
  \usepackage[T1]{fontenc}
  \usepackage[utf8]{inputenc}
\else % if luatex or xelatex
  \ifxetex
    \usepackage{mathspec}
    \usepackage{xltxtra,xunicode}
  \else
    \usepackage{fontspec}
  \fi
  \defaultfontfeatures{Mapping=tex-text,Scale=MatchLowercase}
  \newcommand{\euro}{€}
\fi
% use upquote if available, for straight quotes in verbatim environments
\IfFileExists{upquote.sty}{\usepackage{upquote}}{}
% use microtype if available
\IfFileExists{microtype.sty}{\usepackage{microtype}}{}
\ifxetex
  \usepackage[setpagesize=false, % page size defined by xetex
              unicode=false, % unicode breaks when used with xetex
              xetex]{hyperref}
\else
  \usepackage[unicode=true]{hyperref}
\fi
\hypersetup{breaklinks=true,
            bookmarks=true,
            pdfauthor={},
            pdftitle={},
            colorlinks=true,
            citecolor=blue,
            urlcolor=blue,
            linkcolor=magenta,
            pdfborder={0 0 0}}
\urlstyle{same}  % don't use monospace font for urls
\setlength{\parindent}{0pt}
\setlength{\parskip}{6pt plus 2pt minus 1pt}
\setlength{\emergencystretch}{3em}  % prevent overfull lines
\setcounter{secnumdepth}{0}


\begin{document}

\section{Course Schedule}\label{course-schedule}

\subsection{Week 1: You do what?}\label{week-1-you-do-what}

Readings on what digital humanities, by scholars and by outside critics:
barbarians at the gates narrative and the destruction of the human.

\subsubsection{Readings:}\label{readings}

\begin{itemize}
\itemsep1pt\parskip0pt\parsep0pt
\item
  Marche, Stephen, ``Literature is not Data: Against Digital
  Humanities''
  \url{http://lareviewofbooks.org/essay/literature-is-not-data-against-digital-humanities}
\item
  Kirsch, Adam, ``Technology Is Taking Over English Departments: The
  false promise of the digital humanities''
  \url{http://www.newrepublic.com/article/117428/limits-digital-humanities-adam-kirsch}
\end{itemize}

\subsubsection{Assignment:}\label{assignment}

\begin{itemize}
\itemsep1pt\parskip0pt\parsep0pt
\item
  ``What is Digital Humanities: A Student Debate``: Build a Blog, Print
  Document, and eBook
\end{itemize}

\subsubsection{In-Class Activities}\label{in-class-activities}

\begin{itemize}
\itemsep1pt\parskip0pt\parsep0pt
\item
  OS Administrator and File Extensions
\item
  Introduce Command Line
\item
  Introduce Plain Text Editors
\end{itemize}

\subsection{Week 2: What does a digital humanist
do?}\label{week-2-what-does-a-digital-humanist-do}

Discussions about what digital humanists do by digital humanists.

\subsubsection{Readings:}\label{readings-1}

\begin{itemize}
\itemsep1pt\parskip0pt\parsep0pt
\item
  Matthew Kirschenbam, ``What is Digital Humanities and What's It Doing
  in English Departments?''
  \url{http://mkirschenbaum.files.wordpress.com/2011/01/kirschenbaum_ade150.pdf}
\item
  Julia Flanders, ``The productive unease of 21st-century digital
  scholarship''
  \url{http://www.digitalhumanities.org/dhq/vol/3/3/000055/000055.html}
\item
  Stephen Ramsay, ``On Building''
  \url{http://stephenramsay.us/text/2011/01/11/on-building/}
\item
  Mark Sample, ``The digital humanities is not about building, it's
  about sharing'' \url{http://bit.ly/1kLZ8XW}
\end{itemize}

\subsubsection{Assignment:}\label{assignment-1}

\begin{itemize}
\itemsep1pt\parskip0pt\parsep0pt
\item
  Activity 1: Install Pandoc.
\item
  Activity 2: Convert Sample Markdown File w/ bib to Word RTF and HTML
\item
  Activity 3: Write blog post with bibliographical reference on what DH
  is
\end{itemize}

\subsection{Week 3: Form and content: What's the
Difference?}\label{week-3-form-and-content-whats-the-difference}

\subsubsection{Readings:}\label{readings-2}

\begin{itemize}
\itemsep1pt\parskip0pt\parsep0pt
\item
  ``Single-Source Publishing,'' according to Wikipedia
  \url{http://en.wikipedia.org/wiki/Single_source_publishing}
\item
  ``Document Management System,'' according to Wikipedia
  \url{http://en.wikipedia.org/wiki/Document_management_system}
\item
  Sullivan, Ian, ``Innovation in practice,'' Software Freedom Law Center
  \url{https://www.softwarefreedom.org/blog/2014/apr/11/innovation-in-practice/}
\end{itemize}

\subsubsection{Assignment:}\label{assignment-2}

Activity 1: Install GitHub and DropBox and submit Markdown-format Blog
Post

\begin{itemize}
\item
  Activity 2: Install Zotero, and export BibLaTeX (*.bib) files
\item
  Activity 3: Install (as necessary) LaTeX or Calibre, or set up Reclaim
  Hosting site with WordPress
\end{itemize}

\subsection{Week 4: Publishing Class Single-Source Discussion of ``What
is
DH''?}\label{week-4-publishing-class-single-source-discussion-of-what-is-dh}

\subsubsection{Assignments:}\label{assignments}

\begin{itemize}
\itemsep1pt\parskip0pt\parsep0pt
\item
  All: Post revised blog Markdown source to GitHub
\item
  Assignment 1 Print Working Group: Submit LaTeX-generated print copy
  and PDF to Team GitHub Repository
\item
  Assignment 1 Public Blog Working Group: Submit link to blog home page
  on Team GitHub Repository
\item
  Assignment 1 eBook Working Group: Submit link to eBook on Team GitHub
  Repository
\item
  All Students: Post Assignment 1 Reflection to class blog, 500 words
\end{itemize}

\subsection{Week 5: Getting to the
Text}\label{week-5-getting-to-the-text}

\subsubsection{Readings:}\label{readings-3}

\begin{itemize}
\itemsep1pt\parskip0pt\parsep0pt
\item
  Alcott, \emph{Hospital Sketches} \emph{Boston Commonwealth,} 22 May
  1863
\item
  Alcott, \emph{Hospital Sketches} \emph{Boston Commonwealth,} 29 May
  1863
\item
  Alcott, \emph{Hospital Sketches} \emph{Boston Commonwealth,} 12 June
  1863
\item
  Alcott, \emph{Hospital Sketches} \emph{Boston Commonwealth,} 26 June
  1863
\item
  Alcott, \emph{Hospital Sketches} (Boston, Redpath, 1863),
  (https://archive.org/details/hospitalsketches00alcorich){[}https://archive.org/details/hospitalsketches00alcorich{]}
\end{itemize}

\subsubsection{Assignment:}\label{assignment-3}

\begin{itemize}
\itemsep1pt\parskip0pt\parsep0pt
\item
  Visit library special collections
\item
  Assign Groups: Set up Reclaim Hosting and Install Omeka, WordPress and
  Drupal
\item
  Create Twitter account (class or personal) and designate hash tag
\end{itemize}

\subsection{Week 6: Back to the archive
again}\label{week-6-back-to-the-archive-again}

\subsubsection{Readings:}\label{readings-4}

\begin{itemize}
\itemsep1pt\parskip0pt\parsep0pt
\item
  Kline, Mary-Jo and Susan Holbrook Perdue, ``{[}Section{]} A:
  Establishing the Editorial Texts''
  \url{http://gde.upress.virginia.edu/06-gde.html}
\item
  Daniel J. Cohen and Roy Rosenszwig, ``How to Make Text Digital:
  Scanning, OCR, and Typing,'' \emph{Digital History,}
  \url{http://chnm.gmu.edu/digitalhistory/digitizing/4.php}
\end{itemize}

\subsubsection{Assignment:}\label{assignment-4}

\begin{itemize}
\itemsep1pt\parskip0pt\parsep0pt
\item
  Transcribe \emph{Hospital Sketches} (Group Project)
\item
  Install Tesseract for OCR test
\end{itemize}

\subsection{Week 7:}\label{week-7}

\subsubsection{Readings:}\label{readings-5}

\begin{itemize}
\itemsep1pt\parskip0pt\parsep0pt
\item
  Renear, Alan. ``Text Encoding.'' \emph{A Companion to the Digital
  Humanities}
\item
  Turkel, William J. ``Doing OCR Using Command Line in UNIX''
  \url{http://williamjturkel.net/2013/07/06/doing-ocr-using-command-line-tools-in-linux/}
\item
  Daniel J. Cohen and Roy Rosenszwig, ``Digital Images'' \emph{Digital
  History,} \url{http://chnm.gmu.edu/digitalhistory/digitizing/5.php}
\end{itemize}

\subsubsection{Assignment:}\label{assignment-5}

\begin{itemize}
\itemsep1pt\parskip0pt\parsep0pt
\item
  Compare Transcribed and OCR Text in JUXTA\\
\item
  Orally proofread transcribed text against original document
\item
  Introduce REGEX
\item
  Read Introduction to XML
\end{itemize}

\subsection{Week 8: Text pictures are
words}\label{week-8-text-pictures-are-words}

\subsubsection{Readings:}\label{readings-6}

\begin{itemize}
\item
\item
\item
\item
\end{itemize}

\subsubsection{Assignment:}\label{assignment-6}

\begin{itemize}
\itemsep1pt\parskip0pt\parsep0pt
\item
  Introduce XML, TEILite, XSLT, and Saxon
\item
  Install Versioning Machine
\end{itemize}

\subsection{Week 9: What is text,
really?}\label{week-9-what-is-text-really}

\subsubsection{Readings:}\label{readings-7}

\begin{itemize}
\item
\item
\item
\item
\item
  Daniel J. Cohen and Roy Rosenzweig, ``To Mark Up, Or Not To Mark Up''
  \emph{Becoming Digital}
  (http://chnm.gmu.edu/digitalhistory/digitizing/3.php){[}http://chnm.gmu.edu/digitalhistory/digitizing/3.php{]}
\end{itemize}

\subsubsection{Assignment:}\label{assignment-7}

\begin{itemize}
\itemsep1pt\parskip0pt\parsep0pt
\item
  Propose Project Plan
\item
  Install Drupal and TEIChI \emph{or} Install Omeka and TEILite
\end{itemize}

\subsection{Week 10: All hang together, or all hang
separately}\label{week-10-all-hang-together-or-all-hang-separately}

\begin{itemize}
\item
\item
\item
\end{itemize}

\subsubsection{Readings:}\label{readings-8}

\begin{itemize}
\item
\item
\item
\end{itemize}

\subsubsection{Assignment:}\label{assignment-8}

\begin{itemize}
\item
\item
\item
\end{itemize}

\subsection{Week 11: Gathering our
wares}\label{week-11-gathering-our-wares}

\begin{itemize}
\item
\item
\item
\end{itemize}

\subsubsection{Readings:}\label{readings-9}

\begin{itemize}
\item
\item
\item
\end{itemize}

\subsubsection{Assignment:}\label{assignment-9}

\begin{itemize}
\item
\item
\item
\end{itemize}

\subsection{Week 12: If we build it, will they
come?}\label{week-12-if-we-build-it-will-they-come}

\begin{itemize}
\item
\item
\item
\end{itemize}

\subsubsection{Readings:}\label{readings-10}

\begin{itemize}
\item
\item
\item
\end{itemize}

\subsubsection{Assignment:}\label{assignment-10}

\begin{itemize}
\item
\item
\item
\end{itemize}

\subsection{Week 13: Who is our
audience?}\label{week-13-who-is-our-audience}

\begin{itemize}
\item
\item
\item
\end{itemize}

\subsubsection{Readings:}\label{readings-11}

\begin{itemize}
\item
\item
\item
\end{itemize}

\subsubsection{Assignment:}\label{assignment-11}

\begin{itemize}
\item
\item
\item
\end{itemize}

\subsection{Week 14: Is this legal?}\label{week-14-is-this-legal}

\begin{itemize}
\item
\item
\item
\end{itemize}

\subsubsection{Readings:}\label{readings-12}

\begin{itemize}
\item
\item
\item
\end{itemize}

\subsubsection{Assignment:}\label{assignment-12}

\begin{itemize}
\item
\item
\item
\end{itemize}

\subsection{Week 15: Sharing our wares}\label{week-15-sharing-our-wares}

\begin{itemize}
\item
\item
\item
\end{itemize}

\subsubsection{Readings:}\label{readings-13}

\begin{itemize}
\item
\item
\item
\end{itemize}

\subsubsection{Assignment:}\label{assignment-13}

\begin{itemize}
\item
\item
\item
\end{itemize}

\end{document}
