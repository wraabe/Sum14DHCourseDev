\documentclass[]{article}
\usepackage{lmodern}
\usepackage{amssymb,amsmath}
\usepackage{ifxetex,ifluatex}
\usepackage{fixltx2e} % provides \textsubscript
\ifnum 0\ifxetex 1\fi\ifluatex 1\fi=0 % if pdftex
  \usepackage[T1]{fontenc}
  \usepackage[utf8]{inputenc}
\else % if luatex or xelatex
  \ifxetex
    \usepackage{mathspec}
    \usepackage{xltxtra,xunicode}
  \else
    \usepackage{fontspec}
  \fi
  \defaultfontfeatures{Mapping=tex-text,Scale=MatchLowercase}
  \newcommand{\euro}{€}
\fi
% use upquote if available, for straight quotes in verbatim environments
\IfFileExists{upquote.sty}{\usepackage{upquote}}{}
% use microtype if available
\IfFileExists{microtype.sty}{\usepackage{microtype}}{}
\usepackage{longtable,booktabs}
\ifxetex
  \usepackage[setpagesize=false, % page size defined by xetex
              unicode=false, % unicode breaks when used with xetex
              xetex]{hyperref}
\else
  \usepackage[unicode=true]{hyperref}
\fi
\hypersetup{breaklinks=true,
            bookmarks=true,
            pdfauthor={},
            pdftitle={},
            colorlinks=true,
            citecolor=blue,
            urlcolor=blue,
            linkcolor=magenta,
            pdfborder={0 0 0}}
\urlstyle{same}  % don't use monospace font for urls
\setlength{\parindent}{0pt}
\setlength{\parskip}{6pt plus 2pt minus 1pt}
\setlength{\emergencystretch}{3em}  % prevent overfull lines
\setcounter{secnumdepth}{0}


\setlength{\topmargin}{-2cm}
\setlength{\evensidemargin}{-.5cm}
\setlength{\oddsidemargin}{-.5cm}
%\setlength{\baselineskip}{20pt}
\setlength{\textwidth}{17.5cm}
\setlength{\textheight}{24cm}


\begin{document}

\section{ENG 39995: Special Topics: Digital
Humanities}\label{eng-39995-special-topics-digital-humanities}

\begin{longtable}[c]{@{}ll@{}}
\toprule\addlinespace
Wesley Raabe & Spring 2015: ENG 39995-001 (CRN 21059)
\\\addlinespace
& Library 317: MW 12:30 pm--01:45 pm
\\\addlinespace
Contact Info: & KSU Email (preferred): wraabe@kent.edu
\\\addlinespace
& Twitter: @wraabe
\\\addlinespace
Office Hours: & By phone during office hours: messages checked ONLY
during office hours
\\\addlinespace
& SFH 205c (Ph. 672-2092): T 9:45--10:45 am
\\\addlinespace
& Library 920 (Ph. 672-1723): M 2:00--2:45 pm; W 8:00--10:00 am
\\\addlinespace
& By appointment (at agreed time---give 4-hr. notice to cancel)
\\\addlinespace
\bottomrule
\end{longtable}

\subsection{Notices}\label{notices}

\begin{itemize}
\itemsep1pt\parskip0pt\parsep0pt
\item
  The prerequisite for this course is either College Writing II (ENG
  21011) or Honors Colloquium II (HONR 20197).
\item
  Consult the registrar calendar for each semester's add/drop and
  withdrawal (no grade) date, which may vary among courses.
\item
  If you are not officially registered by add/drop deadlines, you will
  not receive credit or a grade for the course. Confirm enrollment by
  checking your class schedule in FlashLine. Errors must be corrected
  prior to the add/drop deadline.
\end{itemize}

\subsection{Goals}\label{goals}

The goals for student learning in ``ENG 39395, ST: Intro to Digital
Humanities'' are the following:

\begin{enumerate}
\def\labelenumi{\arabic{enumi}.}
\itemsep1pt\parskip0pt\parsep0pt
\item
  To acquire basic digital literacy skills (text and image acquisition,
  text encoding, image processing, text processing) that undergird
  web-based technology;
\item
  To become cognizant of the challenges of reproducing cultural
  artifacts and to become aware that some benefits of digitization, may
  risk damaging original artifacts or not fully representing some
  qualities of original documents;
\item
  To develop project management and collaboration skills that are
  required for significant web-based projects;
\item
  To develop interpersonal skills and communication techniques for
  addressing systematically and collaboratively the challenges that
  impede progress in group-based technology projects;
\item
  To achieve awareness of the responsibilities that developers of public
  projects have for crediting sources and for ensuring access by diverse
  audiences;
\item
  To develop a professional public identity that is associated with a
  web project and with associated class materials (blog posts and
  tweets) in publicly accessible online sites and forums.
\end{enumerate}

\subsection{Course Materials}\label{course-materials}

\begin{itemize}
\itemsep1pt\parskip0pt\parsep0pt
\item
  \emph{A Companion to Digital Humanities.} Ed. Susan Schreibman, Ray
  Siemens, and John Unsworth. New York: Blackwell, 2004.
  \textless{}\url{http://www.digitalhumanities.org/companion/}\textgreater{}.
\item
  \emph{Electronic Textual Editing.} Ed. Lou Burnard, Katherine O'Brien
  O'Keefe, and John Unsworth. New York: MLA, 2006.
  \textless{}\url{http://www.tei-c.org/About/Archive_new/ETE/Preview/}\textgreater{}.
\end{itemize}

\subsection{Grading}\label{grading}

\begin{longtable}[c]{@{}cl@{}}
\toprule\addlinespace
30\% & Blogs and Participation
\\\addlinespace
20\% & Paper
\\\addlinespace
20\% & Interim Project Stages
\\\addlinespace
30\% & Final Group Project
\\\addlinespace
\bottomrule
\end{longtable}

I grade all on-time assignments (blogs, papers, projects) within a week.
I allow myself one extended grading period (extra week) for a major
paper or project. If you miss class, please contact me about work
returned during previous class. And please double-check grade entries on
Blackboard. \emph{Caution:} As a formal policy, if you earn a failing
grade (below D- or 60\%) on major project and paper, you will
automatically fail the course.

\subsection{Accessibility Statement}\label{accessibility-statement}

My aim is for course content to be available to all students. Students
who have a documented disability may need reasonable accommodations to
participate fully in this class. Even if you do not have a documented
disability, some materials that I provide may present challenges. Many
of the basic university services that are available to all students in
this class---office hours, library reference desk, writing center,
departmental advising, psychological services---are available to you on
an as-needed basis without formal documentation.

In the case of a formally designated ``documented disability,''
alterations of course policies or procedures to make the course more
accessible to one student may result in different course policies for
different students---and that's fine. I and other professors have varied
policies in different classes according to number of students,
pedagogical aim of class, etc. So the legally defined standard of
``reasonable accommodation'' is a sensible burden for a professor to
assume in order to ensure the greater value of accessibility---because
the burden that a professor assumes to provide alternate options for
accessibility is no greater than what students bear when professors have
different policies. However, to receive an accommodation---to customize
class syllabus policy on your individual behalf---university policy
requires that you complete the paperwork to verify that you have a
``documented disability.'' You must complete the paperwork at the start
of the semester to verify eligibility. To ensure that you receive the
accommodation to which you have a right, \textbf{you must first verify
your eligibility for these through Student Accessibility Services}
(contact 330-672-3391 or visit
\textless{}\url{www.kent.edu/sas}\textgreater{} for more information on
registration procedures). Consult legalistic details for all university
policies at
\textless{}\url{http://www.kent.edu/policyreg/index.cfm}\textgreater{}.

\subsection{Blog Posts}\label{blog-posts}

You will contribute 4 to blog posts of approximately 500--750 words to
the class blog. Each blog will have a prompt. A blog post is an
opportunity to build on in-class discussion and to contribute your own
observations, and I recommend for full credit that you engage in
additional research or reading. Blog posts need not observe the studied
formality of a paper, but you should should exhibit proper spelling and
punctuation, and MLA citation style. Blog assignments are submitted only
in electronic format: no paper copy is required.

\subsection{Absences and Disruptions}\label{absences-and-disruptions}

I will distinguish between excused and unexcused absences. To grant an
excused absence, I expect formal notification (email or voice mail
message) as soon as practicable. In the case of family matters or
serious illness, please send me a brief message as soon as practical.
For scheduled university activities, contact me {before} the expected
absence. So that I can arrange make-up exams or quizzes, please provide
a one-week notice prior to the absence. I respect your privacy. You need
not provide documentation in forms of doctor's excuse, death
certificate, etc. An email message in the following form qualifies an
absence as excused: ``I need to attend to a health matter,'' or ``I had
to attend to a family matter.'' Please also notify me when you expect to
return to class. I expect you to contact me at the earliest convenient
time: do not wait until day of your return to class.

You are permitted to miss the equivalent of up to 2 class sessions on
non-exam days with no formal penalty to your participation grade for the
absences, though you cannot make up in-class grades or activities. For
no more than 2 absences, your ``in-class'' portion of your participation
grade will range from A to B+. For 3 or 4 absences, the in-class portion
of your participation grade will be calculated at B to C+. For 5 or 6
absences, C to D-. For 7 or more, a ``0.''

\textbf{Extended Absences:} If you suffer an extended health matter or
family crisis---you miss more than a week---you can earn full credit for
a make-up. One set of extended absence dates (up to two weeks) for a
serious matter can be worked around during the first 10 weeks of the
semester. During late-semester, extended absences for a qualifying
reason (death in family, illness) may qualify you to seek an incomplete.
If I do not receive reasonably regular communication about a matter that
causes you to be absent from class (at least every two weeks), I will
assume that you intend either to withdraw or to face grade consequence
for excessive absences. Because of past experience with students who
request extraordinary assistance from professor in their effort to catch
up after missing several classes, I will only provide catch-up
assistance after you have returned to class, have completed at least one
missed assignment satisfactorily, and have made an office visit. Serious
health or family catastrophes (more than 2 weeks) may qualify you to
have a semester expunged from your record. Such matters or coordinated
by your academic advisor or the student ombuds office, which will
contact your professors.

Keep disruptions to a minimum. Before class begins, silence or turn off
electronic devices (pager, phone, etc.). Conversations unrelated to
class should be held outside of class, and minimize communication (talk,
or text) that distracts you or others from class. Arrive in class on
time, and do not leave early. If you arrive more than ten minutes late
or leave before class is dismissed, expect to be counted absent. To
consult with the instructor, send an email, drop by during office hours,
or schedule an appointment.

\subsection{Maintaining Communication}\label{maintaining-communication}

Formal policies on absences are not a substitute for maintaining
communication: a students choice to drop communication is more
disruptive than class absences. If you communicate, I can plan for your
return to class. Regardless of whether absences are excused or
unexcused, you are responsible for checking class web site and (for
group projects) classmates to identify what you missed. It is more
important to visit during office hours or to contact me by email to
confirm what you missed. Office visits during ``by appointment'' hours
(which means I have not said I will be in the office) should always be
scheduled. If you schedule a by-appointment office visit outside of my
usual hours and miss the scheduled appointment, I will count it as an
absence.

In the case of an extended series of absences or an unexplained absence
on a major paper or project or exam date, you are required to initiate a
formal contact with the professor (email, office visit) to reinstate
yourself in the class. Any of the following three events demand that you
contact me: missing more than two classes in a row, missing an exam, or
missing a paper due date and the following class. If you have not
formally dropped and wish to continue in course, an email of explanation
and an office visit are required within two days after returning to
class. If during early semester you miss more than three classes in a
row or if you miss class at a major due date (paper, exam) with no
contact, I will file an ``early alert'' on the campus notification
system. If you miss multiple classes or major due date late in the
semester (weeks 10--15), I will contact you once via email. If you do
not respond promptly (within 48 hours), I will assume that you intend to
drop.

\subsection{Papers}\label{papers}

Papers must \emph{always} be submitted in print and electronic form. To
earn full credit, follow all conventions of academic prose and format.
In general I assume the following matters are understood as expectations
for academic papers, but you should review and note for yourself
anything that departs from your previous practices on papers.

\begin{itemize}
\item
  Papers must have appropriate format for titles (centered, no extra
  space), first-page headings (your name, date, my name, name of class
  and assignment), page numbers, appropriate font (11-pt. Times Roman or
  similar), 1-inch margins, and line spacing. For a sample MLA Style
  paper, see a handbook or the Purdue OWL site. I will also accept
  papers in Chicago or APA style, but I prefer the MLA-style first-page
  header rather than a cover sheet. Papers with comically exaggerated
  font size, line space, or margins to lengthen or shorten will be
  returned without credit.
\item
  A paper in standard format---when one allows for difference between
  one or two extended block quotations and all full-length prose
  lines---has about 400 to 425 words per page. Because of heading
  matter, the first page will have fewer words, about 350 to 375 words:
  a 4-page paper has 1,550 to 1,650 words, and a 6-page paper has 2,350
  to 2,500 words. Generally, based on word count math (5 characters is a
  ``word''), a flexible cushion is built into most assignment: ``4 to
  5'' pages may be read as ``1,550 to 2,075'' words. To qualify for full
  credit, an ``A,'' your paper should not depart from these norms by
  more than 10 percent.
\item
  Guidelines on length may seem arbitrary, but length requirements are
  an inescapable fact about most published writing. The time that you
  spend revising to ensure that your paper falls into appropriate
  length, if you exercise good planning and self-discipline so as to
  demand productive work from yourself, is how you appeal to a
  designated audience. In college papers, some tell-tale signs of
  excessive attention to formatting (instead of revision) include the
  following: fewer than 23 or more than 25 lines on a full page; a 0.75-
  or 1.25-in. page margin; a font at a peculiar size like 10.3- or
  11.8-pt. or a sans-serif face. Block quotes seem especially to invite
  creative formatting, so place no extra padding beyond proper
  block-quote indent of 1 in., do not indent right margin, have a
  minimum of 3 lines and do not use 8-plus line quotes, and do not use
  single spacing or extra line before or after block quotes. I worked as
  a university press typesetter, typically receive well over 1,000
  manuscript pages per semester, have access to electronic submissions,
  and generally have a ruler on my desk with which to measure margins,
  so don't spend an hour on formatting cleverness to sneak something by.
\item
  Formal papers are listed on syllabus and posted on Blackboard or
  course web site. All formal assignments must be submitted both as
  \textbf{print copies} and as \textbf{electronic copies} on Blackboard.
  Formal papers submitted electronically via other means (such as email)
  will not be accepted as on time nor will they be accepted for credit.
  Submissions in non-designated proprietary formats (including Apple
  Pages) will not be accepted for credit. You may request permission to
  submit papers in alternate electronic format: if I have or can locate
  non-proprietary free software to read the document, I will accept
  them, provided you have requested and received approval before
  submission. Corrupt file formats (invalid extensions, etc.) shall be
  construed as missed assignments.
\item
  Use MLA parenthetical references for quotations and for paraphrases.
  At end of paper, include works cited list. I do not require a separate
  page for works cited list. If you can save a page, you may print part
  of works cited list on bottom of last text page (I accept that. Some
  professors may not). If the author of a quoted or paraphrased passage
  is unambiguous (i.e., mentioned in sentence, same as previous, primary
  work under discussion), omit author's name in parenthetical notation.
\item
  The proposal draft is required. A final draft will only be accepted
  for credit if the proposal draft has been completed.
\item
  You may submit one paper or proposal up to one week late. The late
  paper submissions at any stage (proposal or final) will incur a
  permanent deduction of one letter grade on the overall paper. For a
  second late submission, I will assign a grade of ``0.''
\item
  \textbf{1st and 3rd Person} The judicious use of the first-person
  pronoun ``I'' is acceptable. You can avoid its use in formal writing
  as 3rd-person writing carries with it the assumption that the writer
  holds a critical view or offers an observation. Brief 1st-person
  impressions are permissible in formal writing in my academic
  disciplines (literary and cultural studies), but other professional
  disciplines, such as sciences, vary on attitude toward 1st-person
  remarks. On matter of 3rd-person critical voice, its use is not an
  excuse to bury your source. Statements about text and its cultural
  contexts or history of critical reading should be attributed to
  external sources, even if the source is something the professor said
  in class, is included in anthology introduction, or is posted on
  Wikipedia. In other words, the use of 3rd-person as your well-earned
  voice of critical authority (because you have done research) does not
  relieve responsibility to note sources for facts.
\item
  You are permitted to revise Paper 1 to improve the grade. I welcome
  brief email notes with questions and anything from notes to full
  drafts during office hours. I will not pre-grade multi-page drafts by
  email. I will answer short email queries promptly, but I can offer
  only one or two comments by email on drafts up to 2 pages. If you wish
  for extended comments at a full-draft stage, an office-hour visit is
  required. I am only willing to offer an approximate grade if you bring
  a complete draft to office hours 24 hours before the assignment is
  due. The check-up draft (when requested) is not ``graded for content''
  nor does missing it cause a grade deduction. It is a participation
  grade to ensure that you continue to make progress on a longer paper.
\end{itemize}

\subsection{Digital Humanities
Projects}\label{digital-humanities-projects}

According to the Modern Language Association Committee on Scholarly
Editions, ``scholarly editions make clear what they promise and keep
their promises'' (\emph{Electronic Textual Editing} 23). The same advice
is fitting for electronic projects. Five criteria are used to evaluate
scholarly editions: accuracy, adequacy, appropriateness, consistency,
and explicitness (23). I will aid you in choosing a project that is
``adequate'' and believed to be manageable over the course of the last
several weeks of the semester. By regular communication and with my
assistance (during review of proposals and office visits for interim
project stages), we will ensure that you develop reasonable claims about
your intended project and are confident that you can live up to those
claims. Because the amount of work that is necessary to complete a
project is not easily predictable (short of doing a representative
sample of the project), I do count on you to help me refine expectations
over the course of the project. At the end of the term, when I review
your project systematically, I expect to see a substantial amount of
work, an explicit accounting of procedures that were used to establish
accuracy and consistency in your work, and a clear acknowledgment if
portions of anticipated work remain undone.

\textbf{Note: I will provide more extensive guidance in project
assignment.}

\subsection{Cheating and Plagiarism}\label{cheating-and-plagiarism}

By second week of class, I will post a Blackboard assignment in which
you affirm your familiarity with the university's cheating and
plagiarism policy and in which sanctions for cheating and plagiarism are
described. You must complete the assignment before you can earn credit
for class submissions.

\textbf{How Not to Plagiarize:} The scholar Amanda French has offered
helpful advice on impermissible copying, especially actions that
constitute plagiarism and copyright violation, at
\textless{}\url{http://digitalpast002.onmason.com/syllabus/}\textgreater{}:

\begin{quote}
If you are copying and pasting text that someone else wrote, you might
be plagiarizing. Pasted or manually retyped text is not plagiarized only
when all of the following three conditions are true: 1) the pasted text
is surrounded by quotation marks or set off as a block quote, and 2) the
pasted text is attributed in your text to its author and its source
(e.g., ``As Jane Smith writes on her blog . . .''), and 3) the pasted
text is cited in a footnote, endnote, and/or a bibliography (e.g.,
``Smith, Jane. Smith Stuff. Blog. Available
\textless{}\url{http://dummyaddress.wordpress.com}\textgreater{}
Accessed August 1, 2012.''). Conventions for copying and pasting
computer code are less strict, but even when you copy and paste code, if
you can identify the actual individual who wrote the code, you should
give the coder's name and the source of the code in a code comment. If
you find and use images, audio, or video on the web, you should also
cite the creator (if known) and the source (at the very least) of that
media file, usually in a caption as well as in a footnote, endnote, or
bibliography. Note that reproducing someone else's text, image, audio,
or video file in full on your own public website may constitute
copyright infringement, even with proper attribution.
\end{quote}

That everyone steals code and violates formal copyright now or that
tech-evangelists or corporate shills on Twitter or YouTube---or Facebook
or Instagram or Pandora or Google, etc.---endorse a culture of free
sharing of copyrighted content, is not sufficient for you to escape the
consequences of plagiarism within this class. Times and laws change, but
my demand that you hold yourself to a high standard for ethical behavior
is part of course policy. I am not qualified to give legal advice on
copyright, but I can advise sensible self-protection. When you post
material on a public web site, due diligence will help you defend
yourself against claims of copyright infringement. To exercise due
diligence, see Cornell University's ``Copyright Term and the Public
Domain in the United States''
\textless{}\url{http://copyright.cornell.edu/resources/publicdomain.cfm}\textgreater{}.
Henry David Thoreau, who called for civil disobedience, spent the night
in jail. If your violation of copyright is principled, indicate your
devotion to the principle by accepting without complaint the failing
grade that I will assign.

\subsection{Course Material Copyright}\label{course-material-copyright}

The university counsel (attorney's office) has notified professors that
students are selling course materials (presentations, handouts, notes,
exams, etc.) to an Internet company, which re-sells those materials to
subscribers. Selling course materials violates a professor's copyright:
the company is re-selling stolen intellectual property. Course materials
that I create and display or distribute to students (unless they are
owned by someone else and distributed under fair use guidelines) are my
intellectual property. Likewise, were I to sell your work on a term
paper web site, I would be violating your copyright.

However, my course materials build on the work of other scholars.
Therefore, I claim what is known as an Attribution-NonCommercial License
(CC By-NC). See
\textless{}\url{http://creativecommons.org/licenses/}\textgreater{} for
details. In sum, you have permission to remix, tweak, or build upon my
work (for example, as a school lesson plan), but you must also release
your new remixed work (if it is substantially similar content) in
noncommercial form. If you create a derivative work (that is, you cite
me when creating something new, but yours is a substantially different
work), you do have permission to license your own work on a commercial
basis.

Please note that my course material copyright differs from standard
syllabus boilerplate that the university counsel recommends. Unless
another professor offers materials under a Creative Commons license, the
usual copyright rules apply for material from that professor.

\section{Course Schedule}\label{course-schedule}

\subsection{Week 1: Digital humanities in popular
culture}\label{week-1-digital-humanities-in-popular-culture}

\subsubsection{In-Class Activities (Jan.
12)}\label{in-class-activities-jan.-12}

\begin{itemize}
\itemsep1pt\parskip0pt\parsep0pt
\item
  Syllabus Overview and Class Introduction
\end{itemize}

\subsubsection{Readings (Jan. 14):}\label{readings-jan.-14}

\begin{itemize}
\itemsep1pt\parskip0pt\parsep0pt
\item
  Marche, Stephen, ``Literature is not Data: Against Digital
  Humanities.'' \emph{Los Angeles Review of Books}
  \textless{}\url{http://lareviewofbooks.org/essay/literature-is-not-data-against-digital-humanities}\textgreater{}.
\item
  Kirsch, Adam, ``Technology Is Taking Over English Departments.''
  \emph{The New Republic}
  \textless{}\url{http://www.newrepublic.com/article/117428/limits-digital-humanities-adam-kirsch}\textgreater{}.
\item
  Susan Hockey, ``The History of Humanities Computing.'' \emph{A
  Companion to Digital Humanities.} 3--19.
\end{itemize}

\subsubsection{In-Class Activities (Jan.
14)}\label{in-class-activities-jan.-14}

\begin{itemize}
\itemsep1pt\parskip0pt\parsep0pt
\item
  OS Administration and File Extensions
\item
  Introduce Command Line and Plain Text Editors. Multi-Purpose:
  NotePad++ (Win), TextWrangler (Mac), SublimeText (Mac and Win).
  MarkDown Only: Mou (Mac) and WriteMonkey (Win). Robust: Vim (Mac or
  Win) or emacs (Mac or Win).
\item
  Introduce Assignment (Due Jan. 28) ``A Student Project: What is
  Digital Humanities?''
\end{itemize}

\subsection{Week 2: What do digital humanists
do?}\label{week-2-what-do-digital-humanists-do}

\subsubsection{Readings (Jan. 21):}\label{readings-jan.-21}

\begin{itemize}
\itemsep1pt\parskip0pt\parsep0pt
\item
  Kirschenbaum, Matthew. ``What is Digital Humanities and What's It
  Doing in English Departments?''
  \textless{}\url{http://mkirschenbaum.files.wordpress.com/2011/01/kirschenbaum_ade150.pdf}\textgreater{}.
\item
  Flanders, Julia. ``The productive unease of 21st-century digital
  scholarship.''
  \textless{}\url{http://www.digitalhumanities.org/dhq/vol/3/3/000055/000055.html}\textgreater{}.
\item
  Ramsay, Stephen. ``On Building.'' \emph{Stephen Ramsay.}
  \textless{}\url{http://stephenramsay.us/text/2011/01/11/on-building/}
\item
  Sample, Mark. ``The digital humanities is not about building, it's
  about sharing.'' \textless{}\url{http://bit.ly/1kLZ8XW}\textgreater{}.
\end{itemize}

\subsubsection{In-Class Activities:}\label{in-class-activities}

\begin{itemize}
\itemsep1pt\parskip0pt\parsep0pt
\item
  Install Pandoc and convert sample Markdown File with a bibliography to
  Word RTF and to HTML
\item
  Write blog post with bibliographical reference on what DH is
\end{itemize}

\subsection{Week 3: Form and content: Publishing Single-Source
Discussion of ``What is
DH''?}\label{week-3-form-and-content-publishing-single-source-discussion-of-what-is-dh}

\subsubsection{Readings (Jan. 26):}\label{readings-jan.-26}

\begin{itemize}
\itemsep1pt\parskip0pt\parsep0pt
\item
  ``Single-Source Publishing.'' Wikipedia.
  \textless{}\url{http://en.wikipedia.org/wiki/Single_source_publishing}
\item
  Sullivan, Ian. ``Innovation in practice.'' Software Freedom Law
  Center.
  \textless{}\url{https://www.softwarefreedom.org/blog/2014/apr/11/innovation-in-practice/}\textgreater{}.
\end{itemize}

\subsubsection{In-Class Activities:}\label{in-class-activities-1}

\begin{itemize}
\itemsep1pt\parskip0pt\parsep0pt
\item
  Install GitHub and create shared repository for Markdown-format Blog
  Post. See Konrad M. Lawson, ``Getting Started with a GitHub
  Repository.'' \emph{Profhacker---Chronicle of Higher Education} (15
  Mar. 2013).
  \textless{}\url{http://chronicle.com/blogs/profhacker/getting-started-with-a-github-repository/47393}\textgreater{}
\item
  Install Zotero \textless{}\url{https://www.zotero.org/}\textgreater{}
  and a ``connector'' (if you prefer standalone and browser), add
  references, and export BibLaTeX (*.bib) files to GitHub repository.
\item
  Install (as necessary, that is, if you will take responsibility) LaTeX
  for print
  documents---\textless{}\url{https://www.tug.org/texlive/acquire-netinstall.html}\textgreater{}
  or \textless{}\url{https://tug.org/mactex/}\textgreater{}---or Calibre
  \textless{}\url{http://calibre-ebook.com/}\textgreater{} for ebooks.
\end{itemize}

\subsubsection{Readings (Jan. 28)}\label{readings-jan.-28}

\begin{itemize}
\itemsep1pt\parskip0pt\parsep0pt
\item
  Laue, Andrea. ``How the Computer Works.'' \emph{A Companion to Digital
  Humanities.} 145--160.
\item
  Sperberg-McQueen, C. M. ``Classification and its Structures'' \emph{A
  Companion to Digital Humanities.} 161--176.
\item
  Borges, Jorge Luis. ``Funes, the Memorious.'' Blackboard.
\item
  Reference Reading for Assignment: Dennis Tenen and Grant Wythoff,
  ``Sustainable Authorship in Plain Text using Pandoc and Markdown.''
  \emph{Programming Historian.}
  \textless{}\url{http://programminghistorian.org/lessons/sustainable-authorship-in-plain-text-using-pandoc-and-markdown}\textgreater{}.
\end{itemize}

\subsubsection{Assignment Submission}\label{assignment-submission}

\begin{itemize}
\itemsep1pt\parskip0pt\parsep0pt
\item
  By Monday (Jan. 26), post draft Markdown source to GitHub
\item
  By Wednesday (Jan. 28), review edits and update
\item
  By Friday (Jan. 30), submit LaTeX-generated PDF to GitHub Repository;
  post MarkDown source to class blog; Upload eBook with all posts and
  submit link to eBook on GitHub Repository
\item
  By Monday (Feb. 2) Submit blog reflection to class blog, 500--750
  words
\end{itemize}

\subsection{Week 4: Acquiring Texts}\label{week-4-acquiring-texts}

\subsubsection{Readings (Feb. 2):}\label{readings-feb.-2}

\begin{itemize}
\itemsep1pt\parskip0pt\parsep0pt
\item
  Berman, Ruth. ``\,`Spirituous Consolation': Alcott's Jokes on Drinking
  and Religion.'' \emph{Children's Literature in Education} 39:3 (2008):
  169--185.
  \textless{}\url{http://search.ebscohost.com/login.aspx?direct=true\&db=aph\&AN=32679769\&site=eds-live}\textgreater{}
\item
  Cohen, Daniel J. and Roy Rosenszwig, ``How to Make Text Digital:
  Scanning, OCR, and Typing.'' \emph{Digital History}
  \textless{}\url{http://chnm.gmu.edu/digitalhistory/digitizing/4.php}\textgreater{}.
\end{itemize}

\subsubsection{In-Class Activity (Feb.
2):}\label{in-class-activity-feb.-2}

\begin{itemize}
\itemsep1pt\parskip0pt\parsep0pt
\item
  Introduce guidelines for acquiring text from primary sources.
\item
  Create Twitter account (class or personal) and designate hash tag
\end{itemize}

\subsubsection{Readings (Feb. 4):}\label{readings-feb.-4}

\begin{itemize}
\itemsep1pt\parskip0pt\parsep0pt
\item
  Duggan, Hoyt N. and Eileen Gifford Fenton, ``Effective methods of
  producing machine-readable text from manuscript and print sources.''
  \emph{Electronic Textual Editing} 241--253.

  \begin{itemize}
  \itemsep1pt\parskip0pt\parsep0pt
  \item
    Turkel, William J. ``Doing OCR Using Command Line in UNIX.''
    \emph{William J. Turkel.} \textless{}
    \url{http://williamjturkel.net/2013/07/06/doing-ocr-using-command-line-tools-in-linux/}\textgreater{}.
  \end{itemize}
\item
  Alcott, \emph{Hospital Sketches} (Boston: Redpath, 1863),
  \url{https://archive.org/details/hospitalsketches00alcorich}\textgreater{}.
  (scans posted to Blackboard, a ``browsing'' would be more a more
  accurate description than a ``reading'' for this and next three
  texts.)
\item
  Alcott, \emph{Hospital Sketches and Camp and Fireside Stories}
  (Boston: Roberts Brothers, 1868).
  \textless{}\url{https://archive.org/details/hospitalsketche00alco}\textgreater{}
\item
  Alcott, Louisa May. \emph{Hospital Sketches} In \emph{Boston
  Commonwealth} 22 May 1863, 29 May 1863, 12 June 1863, and 26 June
  1863.
\end{itemize}

\subsubsection{In-Class Activity (Feb.
4):}\label{in-class-activity-feb.-4}

\begin{itemize}
\itemsep1pt\parskip0pt\parsep0pt
\item
  Introduce open-source OCR tool Tesseract, and plan OCR of
  corresponding sections of \emph{Hospital Sketches} in \emph{Hospital
  Sketches} in \emph{Boston Commonwealth}.
\item
  Plan type transcription of corresponding section of \emph{Hospital
  Sketches} in \emph{Boston Commonwealth}.
\end{itemize}

\subsubsection{Assignment (Due Feb. 10)}\label{assignment-due-feb.-10}

\begin{itemize}
\itemsep1pt\parskip0pt\parsep0pt
\item
  Post acquired texts for a section from two version of the
  \emph{Hospital Sketches} (Redpath 1863 and Roberts Brothers 1868) to
  GitHub project.
\item
  Post OCR-acquired text and transcribed text from same section of
  newspaper (\emph{Boston Commonwealth}) version of \emph{Hospital
  Sketches} to GitHub project.
\end{itemize}

\subsection{Week 5: To the Archive}\label{week-5-to-the-archive}

\subsubsection{Readings (Feb. 9):}\label{readings-feb.-9}

\begin{itemize}
\itemsep1pt\parskip0pt\parsep0pt
\item
  Daniel J. Cohen and Roy Rosenszweig, ``Digital Images'' \emph{Digital
  History,}
  \textless{}\url{http://chnm.gmu.edu/digitalhistory/digitizing/5.php}\textgreater{}
\item
  KSU Finding Aid for Alfred Chester Papers,
  \textless{}\url{http://www.library.kent.edu/alfred-chester-papers}\textgreater{}
\item
  KSU Finding Aid for Thornton Wilder Papers,
  \textless{}\url{http://speccoll.library.kent.edu/theater/wilder.html}\textgreater{}.
\end{itemize}

\subsubsection{In-Class Activity (Feb.
9):}\label{in-class-activity-feb.-9}

\begin{itemize}
\itemsep1pt\parskip0pt\parsep0pt
\item
  Visit library special collections to review Alfred Chester and
  Thornton Wilder Papers and to consider means to acquire images and
  transcribed texts.
\end{itemize}

\subsubsection{Assignment (Due Feb. 9)}\label{assignment-due-feb.-9}

\begin{itemize}
\itemsep1pt\parskip0pt\parsep0pt
\item
  Paper 1 Proposal Due
\end{itemize}

\subsubsection{Readings (Feb. 11):}\label{readings-feb.-11}

\begin{itemize}
\itemsep1pt\parskip0pt\parsep0pt
\item
  Kline, Mary-Jo and Susan Holbrook Perdue, ``{[}Section{]} A:
  Establishing the Editorial Texts.''
  \textless{}\url{http://gde.upress.virginia.edu/06-gde.html}\textgreater{}.
\item
  Knox, Doug. ``Understanding Regular Expressions.'' \emph{Programming
  Historian.}
  \textless{}\url{http://programminghistorian.org/lessons/understanding-regular-expressions}\textgreater{}.
\item
  Ramsay, Stephen, ``Using Regular Expressions.'' \emph{Electronic Text
  Center: University of Virginia} (Blackboard).
  \textless{}\url{http://solaris-8.tripod.com/regexp.pdf}\textgreater{}.
\end{itemize}

\subsubsection{In-Class Activity (Feb.
11):}\label{in-class-activity-feb.-11}

\begin{itemize}
\itemsep1pt\parskip0pt\parsep0pt
\item
  Compare transcribed and OCR Text with multiple means (Juxta and file
  comparison), and identify patterns of differences for correction.
\item
  Review regular expression capabilities in editors.
\end{itemize}

\subsubsection{Assignment (Due Feb. 11)}\label{assignment-due-feb.-11}

\begin{itemize}
\itemsep1pt\parskip0pt\parsep0pt
\item
  Blog Post (500--750 words) on challenges of acquiring and correcting
  texts
\end{itemize}

\subsubsection{Assignment (due Feb. 16):}\label{assignment-due-feb.-16}

\begin{itemize}
\itemsep1pt\parskip0pt\parsep0pt
\item
  Correct all three versions of the \emph{Boston Commonwealth} text (or
  impose editorial consistency) using all four methods: silent
  proofreading with marking, file comparison, regular expressions, and
  oral proofreading.
\end{itemize}

\subsection{Week 7: Text and Encoding}\label{week-7-text-and-encoding}

\subsubsection{Readings (Feb. 16):}\label{readings-feb.-16}

\begin{itemize}
\itemsep1pt\parskip0pt\parsep0pt
\item
  Renear, Alan. ``Text Encoding.'' \emph{A Companion to the Digital
  Humanities} 218--239.
\item
  Birnbaum, David J. ``What is XML and why should humanists care? An
  even gentler introduction to XML.'' \emph{Digital Humanities.}
  \textless{}\url{http://dh.obdurodon.org/what-is-xml.xhtml}\textgreater{}.
\item
  TEI Consortium. ``A Gentle Introduction to XML.''
  \textless{}\url{http://www.tei-c.org/release/doc/tei-p5-doc/en/html/SG.html}\textgreater{}.
\end{itemize}

\subsubsection{In-Class Activity (Feb.
16):}\label{in-class-activity-feb.-16}

\begin{itemize}
\itemsep1pt\parskip0pt\parsep0pt
\item
  Introduce oXygen
\end{itemize}

\subsubsection{Readings (Feb. 18):}\label{readings-feb.-18}

\begin{itemize}
\itemsep1pt\parskip0pt\parsep0pt
\item
  ``Introduction.'' \emph{TEI by Example.}
  \textless{}\url{http://teibyexample.org/examples/TBED00v00.htm}\textgreater{}.
\item
  ``Common Structure and Elements.'' \emph{TEI by Example.}
  \textless{}\url{http://teibyexample.org/modules/TBED01v00.htm}\textgreater{}.
\end{itemize}

\subsubsection{In-Class Activity (Feb.
18):}\label{in-class-activity-feb.-18}

\begin{itemize}
\itemsep1pt\parskip0pt\parsep0pt
\item
  Model encoding a poem and a prose work to conform to TEI in oXygen
\end{itemize}

\subsubsection{Assignment:}\label{assignment}

\begin{itemize}
\itemsep1pt\parskip0pt\parsep0pt
\item
  Paper 1 Due
\end{itemize}

\subsection{Week 8: What is Text,
Really?}\label{week-8-what-is-text-really}

\subsubsection{Readings (Feb. 23):}\label{readings-feb.-23}

\begin{itemize}
\itemsep1pt\parskip0pt\parsep0pt
\item
  DeRose, S. J., Durand, D. G., Mylonas, E., and Renear A. H. (1990),
  ``What is Text, Really?'' \emph{* Journal of Computing in Higher
  Education} 1.2: 3--26. Blackboard.
\item
  Pitti, Daniel V. ``Designing Sustainable Projects and Publications.''
  \emph{Companion to Digital Humanities.} 471--487.
\end{itemize}

\subsubsection{Assignment (Feb. 23):}\label{assignment-feb.-23}

\begin{itemize}
\itemsep1pt\parskip0pt\parsep0pt
\item
  Blog Post (500--750 words) on encoding documents with TEI and two
  sample XML documents, prose and poetry.
\end{itemize}

\subsubsection{Readings (Feb. 25):}\label{readings-feb.-25}

\begin{itemize}
\itemsep1pt\parskip0pt\parsep0pt
\item
  Committee on Scholarly Editions. ``Guidelines for Editors of Scholarly
  Editions.'' \emph{Electronic Textual Editing.} 23--46.
\item
  Smith, Martha Nell. ``Scholarly Editing.'' \emph{Companion to Digital
  Humanities.} 306--322.
\end{itemize}

\subsection{Week 9: Acquiring Images}\label{week-9-acquiring-images}

\subsubsection{Readings (Mar. 2):}\label{readings-mar.-2}

\begin{itemize}
\itemsep1pt\parskip0pt\parsep0pt
\item
  Minnesota Digital Library. ``Quick Reference Guide: Digital Imaging
  Best Practices.''
  \textless{}\url{http://www.mndigital.org/digitizing/standards/guide.php}\textgreater{}.
\item
  University Library, University of Illinois, Urbana-Champaign. ``3.0
  Best Practices for Creating Digital Images.''
  \textless{}\url{http://www.library.illinois.edu/dcc/bestpractices/chapter_03_creatingdigitalimages.html\#textdoc}\textgreater{}.
\end{itemize}

\subsubsection{In-Class Activity:}\label{in-class-activity}

\begin{itemize}
\itemsep1pt\parskip0pt\parsep0pt
\item
  Introduce Scanning
\end{itemize}

\subsubsection{Readings (Mar. 4):}\label{readings-mar.-4}

\begin{itemize}
\itemsep1pt\parskip0pt\parsep0pt
\item
  Deegan, Marilyn and Simon Tanner. ``Conversion of Primary Sources.''
  \emph{Companion to Digital Humanities.} 488--504.
\item
  Rosenberg, Bob. ``Documentary Editing.'' \emph{Electronic Textual
  Editing.} 92--104.
\end{itemize}

\subsubsection{Assignment (Mar. 4):}\label{assignment-mar.-4}

\begin{itemize}
\itemsep1pt\parskip0pt\parsep0pt
\item
  Propose Final Project Plan
\end{itemize}

\subsection{Week 10: Encoding Text, or
Not}\label{week-10-encoding-text-or-not}

\subsubsection{Readings (Mar. 9):}\label{readings-mar.-9}

\begin{itemize}
\itemsep1pt\parskip0pt\parsep0pt
\item
  Cohen, Daniel J. and Roy Rosenzweig, ``To Mark Up, Or Not To Mark
  Up.'' \emph{Becoming Digital}
  \textless{}\url{http://chnm.gmu.edu/digitalhistory/digitizing/3.php}\textgreater{}.
\item
  McGann, Jerome. ``Marking Texts of Many Dimensions.'' \emph{Companion
  to Digital Humanities} 198--217.
\end{itemize}

\subsubsection{Readings (Mar. 11):}\label{readings-mar.-11}

\begin{itemize}
\itemsep1pt\parskip0pt\parsep0pt
\item
  Lavagnino, John. ``When Not to Use TEI.'' \emph{Electronic Textual
  Editing.} 334--338.
\item
  Durusau, Patrick. ``Why and How to Document Your Markup Choices.''
  \emph{Electronic Textual Editing} 299--309.
\end{itemize}

\subsubsection{Assignment:}\label{assignment-1}

\begin{itemize}
\itemsep1pt\parskip0pt\parsep0pt
\item
  Project Meeting with Professor
\end{itemize}

\subsection{Week 11: TeiLite and TEI by
Example}\label{week-11-teilite-and-tei-by-example}

\subsubsection{Readings (Mar. 16):}\label{readings-mar.-16}

\begin{itemize}
\itemsep1pt\parskip0pt\parsep0pt
\item
  ``TEI Header.'' \emph{TEI by Example.}
  \textless{}\url{http://teibyexample.org/examples/TBED02v00.htm}\textgreater{}.
\item
  ``Prose.'' \emph{TEI by Example.}
  \textless{}\url{http://teibyexample.org/examples/TBED03v00.htm}\textgreater{}.
\end{itemize}

\subsubsection{Readings (Mar. 18):}\label{readings-mar.-18}

\begin{itemize}
\itemsep1pt\parskip0pt\parsep0pt
\item
  ``Primary Sources.'' \emph{TEI by Example.}
  \textless{}\url{http://teibyexample.org/examples/TBED06v00.htm}\textgreater{}.

  \begin{itemize}
  \itemsep1pt\parskip0pt\parsep0pt
  \item
    ``TEILite.''
    \textless{}\url{http://www.tei-c.org/Guidelines/Customization/Lite/}\textgreater{}.
  \end{itemize}
\end{itemize}

\subsubsection{Assignment (Due Mar. 30):}\label{assignment-due-mar.-30}

\begin{itemize}
\itemsep1pt\parskip0pt\parsep0pt
\item
  Submit project letter that validates to TEILite Schema
\end{itemize}

\subsection{Week 12: Digital Publication
Options}\label{week-12-digital-publication-options}

\subsubsection{Readings (Mar. 30):}\label{readings-mar.-30}

\begin{itemize}
\itemsep1pt\parskip0pt\parsep0pt
\item
  Pape, Sebastian, Christof Schöch, and Lutz Wegner, ``TEICHI and the
  Tools Paradox.''
  \textless{}\url{http://jtei.revues.org/432}\textgreater{}
\item
  ``Documentation'' and ``Downloads.'' TEICHI.
  \textless{}\url{http://www.teichi.org/}\textgreater{}
\item
  ``Documentation: What Is Omeka?''
  \textless{}\url{http://omeka.org/}\textgreater{}.
\item
  Kirschenbaum, Matthew. ``\,`So the Colors Cover the Wires?':
  Interface, Aesthetics, and Usability.''
\end{itemize}

\subsubsection{In-Class Activities:}\label{in-class-activities-2}

\begin{itemize}
\itemsep1pt\parskip0pt\parsep0pt
\item
  Introduce Reclaim Hosting with TEICHI on Drupal
\item
  Introduce Omeka on Reclaim Hosting
\end{itemize}

\subsubsection{Readings (Apr. 1):}\label{readings-apr.-1}

\begin{itemize}
\itemsep1pt\parskip0pt\parsep0pt
\item
  Wittern, Christian. ``Writing Systems and Character Representation.''
  \emph{Electronic Textual Editing.} 291--298.
\end{itemize}

\subsubsection{Assignment:}\label{assignment-2}

\begin{itemize}
\itemsep1pt\parskip0pt\parsep0pt
\item
  Blog Post (500--750 words) on project status
\item
  Submit project letter to Drupal Instance
\end{itemize}

\subsection{Week 13: Who is our
audience?}\label{week-13-who-is-our-audience}

\subsubsection{Readings (Apr. 6):}\label{readings-apr.-6}

\begin{itemize}
\itemsep1pt\parskip0pt\parsep0pt
\item
  Deegan, Marilyn. ``Collection and Preservation of an Electronic
  Edition.'' \emph{Electronic Textual Editing.} 358--370.
\item
  Smith, Abby. ``Preservation.'' \emph{Companion to Digital Humanities.}
  576--591.
\end{itemize}

\subsubsection{Assignment:}\label{assignment-3}

\begin{itemize}
\itemsep1pt\parskip0pt\parsep0pt
\item
  Project Work
\end{itemize}

\subsection{Week 14 and Week 15 (Apr. 20--Apr. Apr.
27)}\label{week-14-and-week-15-apr.-20apr.-apr.-27}

\subsubsection{Assignment:}\label{assignment-4}

\begin{itemize}
\itemsep1pt\parskip0pt\parsep0pt
\item
  Project Work
\end{itemize}

\subsection{Week 15:}\label{week-15}

\subsubsection{Assignment (Apr. 29):}\label{assignment-apr.-29}

\begin{itemize}
\itemsep1pt\parskip0pt\parsep0pt
\item
  Present Project in Class
\end{itemize}

\subsection{Finals Week}\label{finals-week}

\begin{itemize}
\itemsep1pt\parskip0pt\parsep0pt
\item
  Submit Project Assignment Source Files and Documentation
\end{itemize}

\end{document}
