\documentclass[]{article}
\usepackage{lmodern}
\usepackage{amssymb,amsmath}
\usepackage{ifxetex,ifluatex}
\usepackage{fixltx2e} % provides \textsubscript
\ifnum 0\ifxetex 1\fi\ifluatex 1\fi=0 % if pdftex
  \usepackage[T1]{fontenc}
  \usepackage[utf8]{inputenc}
\else % if luatex or xelatex
  \ifxetex
    \usepackage{mathspec}
    \usepackage{xltxtra,xunicode}
  \else
    \usepackage{fontspec}
  \fi
  \defaultfontfeatures{Mapping=tex-text,Scale=MatchLowercase}
  \newcommand{\euro}{€}
\fi
% use upquote if available, for straight quotes in verbatim environments
\IfFileExists{upquote.sty}{\usepackage{upquote}}{}
% use microtype if available
\IfFileExists{microtype.sty}{\usepackage{microtype}}{}
\usepackage{longtable,booktabs}
\ifxetex
  \usepackage[setpagesize=false, % page size defined by xetex
              unicode=false, % unicode breaks when used with xetex
              xetex]{hyperref}
\else
  \usepackage[unicode=true]{hyperref}
\fi
\hypersetup{breaklinks=true,
            bookmarks=true,
            pdfauthor={},
            pdftitle={},
            colorlinks=true,
            citecolor=blue,
            urlcolor=blue,
            linkcolor=magenta,
            pdfborder={0 0 0}}
\urlstyle{same}  % don't use monospace font for urls
\setlength{\parindent}{0pt}
\setlength{\parskip}{6pt plus 2pt minus 1pt}
\setlength{\emergencystretch}{3em}  % prevent overfull lines
\setcounter{secnumdepth}{0}



\setlength{\topmargin}{-2cm}
\setlength{\evensidemargin}{-.5cm}
\setlength{\oddsidemargin}{-.5cm}
%\setlength{\baselineskip}{20pt}
\setlength{\textwidth}{17.5cm}
\setlength{\textheight}{24cm}

\begin{document}

\section{ENG 39995: Special Topics: Digital
Humanities}\label{eng-39995-special-topics-digital-humanities}

\begin{longtable}[c]{@{}ll@{}}
\toprule\addlinespace
Wesley Raabe & Spring 2015: ENG 39995-001 (CRN 21059)
\\\addlinespace
& Library 317: MW 12:30 pm--01:45 pm
\\\addlinespace
Contact Info: & KSU Email (preferred): wraabe@kent.edu
\\\addlinespace
& Twitter: @wraabe
\\\addlinespace
Office Hours: & By phone during office hours: messages checked ONLY
during office hours
\\\addlinespace
& SFH 205c (Ph. 672-2092): T 9:45--10:45 am
\\\addlinespace
& Library 920 (Ph. 672-1723): M 2:00--2:45 pm; W 8:00--10:00 am
\\\addlinespace
& By appointment (at agreed time---give 4-hr. notice to cancel)
\\\addlinespace
\bottomrule
\end{longtable}

\subsection{Notices}\label{notices}

\begin{itemize}
\itemsep1pt\parskip0pt\parsep0pt
\item
  The prerequisite for this course is either College Writing II (ENG
  21011) or Honors Colloquium II (HONR 20197).
\item
  Consult the registrar calendar for each semester's add/drop and
  withdrawal (no grade) date, which may vary among courses.
\item
  If you are not officially registered by add/drop deadlines, you will
  not receive credit or a grade for the course. Confirm enrollment by
  checking your class schedule in FlashLine. Errors must be corrected
  prior to the add/drop deadline.
\end{itemize}

\subsection{Goals}\label{goals}

The goals for student learning in ``ENG 39395, ST: Intro to Digital
Humanities'' are the following:

\begin{enumerate}
\def\labelenumi{\arabic{enumi}.}
\itemsep1pt\parskip0pt\parsep0pt
\item
  To acquire basic digital literacy skills (text and image acquisition,
  text encoding, image processing, text processing) that undergird
  web-based technology
\item
  To become cognizant of the challenges of reproducing cultural
  artifacts and to become aware that some benefits of digitization, such
  as access, depend on types of translation that include the risk of
  damaging original artifacts or of not fully representing some
  qualities of original documents;
\item
  To develop project management and collaboration skills that are
  required for significant web-based projects;
\item
  To develop interpersonal skills and communication techniques for
  addressing systematically and collaboratively the challenges that
  impede progress in group-based technology projects;
\item
  To achieve awareness of the responsibilities that developers of public
  projects have for crediting sources and for ensuring access by diverse
  audiences;
\item
  To develop a professional public identity that is associated with a
  web project and with associated class materials (blog posts, tweets)
  in publicly accessible online sites and forums.
\end{enumerate}

\subsection{Course Materials}\label{course-materials}

\begin{itemize}
\itemsep1pt\parskip0pt\parsep0pt
\item
  \emph{A Companion to Digital Humanities.} Ed. Susan Schreibman, Ray
  Siemens, and John Unsworth. New York: Blackwell, 2004.
  \url{http://www.digitalhumanities.org/companion/}.
\item
  \emph{Electronic Textual Editing.} Ed. Lou Burnard, Katherine O'Brien
  O'Keefe, and John Unsworth. New York: MLA, 2006.
  \url{http://www.tei-c.org/About/Archive_new/ETE/Preview/}
\end{itemize}

\subsection{Grading}\label{grading}

\begin{longtable}[c]{@{}ll@{}}
\toprule\addlinespace
20\% & Blogs and Participation
\\\addlinespace
10\% & Build Personal Website
\\\addlinespace
20\% & Paper
\\\addlinespace
20\% & Interim Project Steps
\\\addlinespace
30\% & Final Group Project
\\\addlinespace
\bottomrule
\end{longtable}

I grade all on-time assignments (papers, projects) within a week. When
blog assignments are staggered, I grade all blog posts within one week
after a ``set'' is complete. I allow myself one extended grading period
(extra week) for a major project. If you miss class, please contact me
about work returned during previous class. And please double-check grade
entries on Blackboard. \emph{Caution:} As a formal policy, if you earn a
failing grade (below D- or 60\%) two major projects or papers, you will
automatically fail the course.

\subsection{Accessibility Statement}\label{accessibility-statement}

My aim is for course content to be available to all students. Students
who have a documented disability may need reasonable accommodations to
participate fully in this class. Even if you do not have a documented
disability, some materials that I provide may present challenges. For
example, I tend to rely on handouts or posted instructions to spell out
detailed requirements on assignments. If that means for absorbing
information is challenging to you, you are welcome to stop by my office
hours to discuss. Many of the basic university services that are
available to all students in this class---office hours, library
reference desk, writing center, departmental advising, psychological
services---are available to you on an as-needed basis without formal
documentation.

In the case of a formally designated ``documented disability,''
adjustments that alter course policies or procedures to make the course
more accessible to one student may result in different course policies
for different students---and that's fine. I and other professors have
varied policies in different classes according to number of students,
pedagogical aim of class, etc. So the legally defined standard of
``reasonable accommodation'' is a sensible burden for a professor to
assume in order to ensure the greater value of accessibility---because
the burden that a professor assumes to provide alternate options for
accessibility is no greater than what students bear when professors have
different policies. However, for you to receive an accommodation---for
me to alter general syllabus policy on your individual
behalf---university policy requires that you complete the paperwork to
verify that you have a ``documented disability.'' And you must complete
the paperwork at the start of the semester to verify your eligibility.
Therefore, to ensure that you receive the accommodation to which you
have a right, \textbf{you must first verify your eligibility for these
through Student Accessibility Services} (contact 330-672-3391 or visit
\textless{}\url{www.kent.edu/sas}\textgreater{} for more information on
registration procedures). Consult legalistic details for all university
policies at
\textless{}\url{http://www.kent.edu/policyreg/index.cfm}\textgreater{}.

\subsection{Blog Posts}\label{blog-posts}

You will contribute 4 to 5 blog posts of approximately 300 words to the
class blog. A blog post is an opportunity to build on in-class
discussion and to contribute your own observations. While blog posts
need not observe the studied formality of a paper, the blog post should
exhibit proper spelling and punctuation, it should conclude with a
bibliographical entry in a close approximation of MLA style. If you are
assigned to contribute a blog post, please publish the post within 24
hours of the end of class. Blog posts can complement active in-class
participation, and ambitious blog posts can compensate entirely for
minimal oral participation in class. Blog assignments are submitted only
in electronic format.

At the start of the semester, I will allow full credit for blog posts up
to one class session late and will advise in class with reminders on
what a complete blog post requires: visit me during office hours if you
are having trouble. For your second and subsequent posts, I will expect
the posts to appear on time, within 24 hours of end of class. I grade
blog posts either at the end of the week or after the entire set is
complete. For late posts, I deduct a letter grade. Posts more than one
week late earn no credit.

Formal assignments (papers and assignments) are listed on syllabus and
posted on Blackboard or course web site. All formal assignments must be
submitted both as \textbf{print copies} and as \textbf{electronic
copies} on Blackboard. You may submit a word processor document in any
of the following forms: Word DOC or DOCX, Acrobat PDF, GoogleDoc, or
Open Document. If submitting a Google Doc, you may link to KSU Google
Drive so long as you grant me access to the document and do not alter it
after the due date. Presentations may be submitted as PowerPoint
Document or Shows (PPT, PPTX) or as Google Presentations or Prezi
presentations. Paper-style assignments must also be submitted as printed
documents.

\textbf{Formal assignments and papers submitted electronically via other
means (such as email) will not be accepted as on time nor will they be
accepted for credit. Submissions in non-designated proprietary formats
(including Apple Pages) will not be accepted for credit. You may request
permission to submit papers in alternate electronic format: if I have or
can locate non-proprietary free software to read the document, I will
accept them, provided you have requested and received approval before
submission. Corrupt file formats (invalid extensions, etc.) shall be
construed as missed assignments. }

\subsubsection{In-Class Participation}\label{in-class-participation}

Participation during class-wide or small-group discussion is preferred.
A smaller number of high-quality contributions is valued more than
contributions of great quantity. A handout on the art of Sprezzatura
will list suggestions for developing high-quality in-class
contributions, which is also the model for informal blog writing.
Attendance and active in-class participation can complement (but not
substitute for) blog posts and assignments and quizzes.

\subsection{Absences and Disruptions}\label{absences-and-disruptions}

You are permitted to miss the equivalent of up to 2 class sessions on
non-exam days with no formal penalty to your participation grade for the
absences, though you cannot make up in-class grades or activities. For
no more than 2 absences, your ``in-class participation'' portion of your
participation grade will range from A to B+. For 3 or 4 absences, the
``in-class participation'' portion of your participation grade will be
calculated at B- or C+. For 5 or 6 absences, C to D-. For 7 or more, a
``0.''

I will distinguish between excused and unexcused absences. To grant an
excused absence, I expect formal notification (email or voice mail
message) as soon as practicable. In the case of family matters or
serious illness, please send me a brief message as soon as practical.
For scheduled university activities, contact me BEFORE the expected
absence. So that I can arrange make-up exams or quizzes, please provide
a one-week notice prior to the absence. I respect your privacy. You need
not provide documentation in forms of doctor's excuse, death
certificate, etc. An email message in the following form qualifies an
absence as excused: ``I need to attend to a health matter,'' or ``I had
to attend to a family matter.'' Please also notify me when you expect to
return to class. I expect you to contact me at the earliest convenient
time: do not wait until day of your return to class.

\textbf{Extended Absences:} If you suffer an extended health matter or
family crisis---you miss more than a week---you may make up one quiz or
turn in one blog or assignment late without penalty. You will earn full
credit for the make-up. One set of extended absence dates (up to two
weeks) for a serious matter can be worked around during the first 10
weeks of the semester. During late-semester, extended absences for a
qualifying reason (death in family, illness) may qualify you to seek an
incomplete. If I do not receive reasonably regular communication about a
matter that you are attempting to address (at least every two weeks), I
will assume that you intend either to withdraw, request incomplete, or
to face grade consequence for excessive absences. Because of past
experience with students who request extraordinary assistance from
professor in their effort to catch up after missing several classes, I
will only provide catch-up assistance after you have returned to class
and successully completed at least one missed assignment satisfactorily.
Serious health or family catastrophes (more than 3 weeks) may qualify
you to have a semester expunged from your record. For such matters, you
should contact your academic advisor or the student ombuds office, which
will coordinate with your professors.

Keep disruptions to a minimum. Before class begins, silence or turn off
electronic devices (pager, phone, etc.). Conversations unrelated to
class should be held outside of class, and minimize communication (talk,
or text) that distracts you or others from class. Arrive in class on
time, and do not leave early. If you arrive more than ten minutes late
or leave before class is dismissed, expect to be counted absent. To
consult with the instructor, send an email, drop by during office hours,
or schedule an appointment.

\subsection{Maintaining Communication}\label{maintaining-communication}

Regardless of whether absences are excused or unexcused, you are
responsible for checking on Blackboard or class blog and to contact
classmates or group members to identify what you missed. You are also
advised to stop by during office hours or to contact me by email to
confirm what you missed. If you schedule an office visit during my
normal hours but are unable to attend, please notify me at least 4 hours
before the scheduled visit. Office visits during my ``by appointment''
hours should always be scheduled. If you schedule a by-appointment
office visit outside of my usual hours and miss the scheduled
appointment, I will count it as an absence.

In the case of an extended series of absences or an unexplained absence
on a major paper or project or exam date, you are required to initiate a
formal contact with the professor (email, office visit) to reinstate
yourself in the class. Any of the following three events demand that you
contact me: missing more than two classes in a row, missing an exam, or
missing a paper due date and the following class. If you have not
formally dropped and wish to continue in course, an email of explanation
and an office visit are required within two days after returning to
class. If during early semester you miss more than three classes in a
row or if you miss class at a major due date (paper, exam) with no
contact, I will file an ``early alert'' on the campus notification
system. If you miss multiple classes or major due date late in the
semester (weeks 10--15), I will contact you once via email. If you do
not respond promptly (within 48 hours), I will assume that you intend to
drop.

\subsection{Summary}\label{summary}

To maintain yourself in good class standing, it is not acceptable to
skip more than two classes in a row. To restore your standing after
missing three or more classes, I expect to return to class \textbf{and}
set up an office hour appointment. Sporadic good-faith efforts (a paper
in my mail box or on Blackboard, a cryptic email) may demonstrate that
you have a functioning conscience, but possession of a conscience is not
a substitute for class attendance and for submitting assignments and for
taking exams on proper due date.

\section{Papers}\label{papers}

Papers must \emph{always} be submitted in print and electronic form. To
earn full credit, follow all conventions of academic prose and format.
In general I assume the following matters are understood as expectations
for academic papers, but you should review and highlight anything that
departs from your previous practices on papers.

\begin{itemize}
\item
  Papers must have appropriate format for titles (centered, no extra
  space), first-page headings (your name, date, my name, name of class
  and assignment), page numbers, appropriate font (11-pt. Times Roman or
  similar), 1-inch margins, and line spacing. For a sample MLA Style
  paper, see a handbook or the Purdue OWL site. Papers with comically
  exaggerated font size, line space, or margins to lengthen or shorten
  will be returned without credit.
\item
  A paper in standard format, when one allows for difference between one
  or two extended block quotations and all full-length prose lines, has
  about 400 to 425 words per page. Because of heading matter, the first
  page will have fewer words, about 350 to 375: a 4-page paper has 1,550
  to 1,650 words, and a 6-page paper has 2,350 to 2,500 words.
  Generally, I assign papers with flexible length, ``4 to 5'' or ``6 to
  7'' pages. Therefore, based on word count math (5 characters is a
  ``word''), a flexible cushion is built into assignment: ``4 to 5''
  pages may be read as ``1,550 to 2,075'' words. To qualify for full
  credit, an ``A,'' your paper should not depart from these norms by
  more than 10 percent.
\item
  Guidelines on length may seem arbitrary, but you should change your
  way of thinking about that: editors and publishers always have length
  guidelines. The time that you spend revising to ensure your paper
  falls into appropriate length, if you exercise good planning and
  self-discipline so as to demand productive work from yourself, is some
  of the most difficult but important work of writing for a designated
  audience. Tell-tale signs of excessive attention to formatting
  (instead of revision) include the following: fewer than 23 or more
  than 25 lines on a full page; a 0.75 or 1.5-in. page margin; a font at
  a peculiar size like 10.3- or 11.8-pt. or a sans-serif face. Block
  quotes seem especially to invite creativity in the formatting vein, so
  observe following guide: no extra padding of 1-in. left indent, no
  right-margin indent; no 3-line or 8-plus line quotes, and no single
  spacing or extra line space preceding or following. I worked as a
  university press typesetter, typically receive well over 1,000
  manuscript pages per semester, and have access to your electronic
  submissions, so don't waste an hour on formatting cleverness to try to
  sneak something by.
\item
  Use MLA parenthetical references for quotations and paraphrases. At
  end of paper, include works cited list. I do not require a separate
  page for works cited list. If you can save a page, you may print part
  of works cited list on bottom of last text page (I accept that. Some
  professors may not). If the author of a quoted or paraphrased passage
  is unambiguous (i.e., mentioned in sentence, same as previous, primary
  work under discussion), do not repeat author's name in parenthetical
  notation.
\item
  The proposal draft is required. A final draft will only be accepted
  for credit if the proposal draft has been completed.
\item
  You may only submit one paper or proposal late. The late paper
  submissions at any stage (proposal or final) will incur a permanent
  deduction of one letter grade on the overall paper. A second paper or
  proposal submitted late will be assigned a grade of ``0.''
\item
  \textbf{1st and 3rd Person} The judicious use of the first-person
  pronoun ``I'' is acceptable. You can avoid its use in formal writing
  as 3rd-person writing carries with it the assumption that the writer
  holds a critical view or offers an observation. Brief 1st-person
  impressions are permissible in formal writing in my academic
  disciplines (literary and cultural studies), but other professional
  disciplines, such as sciences, vary on attitude toward 1st-person
  remarks. On matter of 3rd-person critical voice, its use is not an
  excuse to bury your source. Statements about text and its cultural
  contexts or history of critical reading should be attributed to
  external sources, even if the source is something the professor said
  in class, is included in anthology introduction, or is posted on
  Wikipedia. In other words, the use of 3rd-person as your well-earned
  voice of critical authority (because you have done research) does not
  relieve responsibility to note sources for facts.
\item
  You are permitted to revise Paper 1 to improve the grade. Paper 2
  cannot be revised, but it is prepared in stages. The deadlines for
  interim stages (proposal, etc.) are actual paper deadlines with a
  consequence for missing the deadline. The purpose is not to be
  punitive but to ensure that you progress in multiple stages, the best
  recipe for ensuring that you write a stronger paper. You are welcome
  to send me a note with questions or to share drafts during office
  hours. But I will not pre-grade multi-page drafts by email, and I am
  only willing to approximate grade if office hour visit to discuss is
  more than 24 hours before the assignment is due. I will answer short
  email queries promptly, but I can offer only one or two comments by
  email on drafts up to 2 pages. If you wish for extended comments at a
  full-draft stage, an office-hour visit is required. The check-up draft
  (when requested) is not ``graded for content'' nor does missing it
  cause a paper grade deduction. It is a participation grade to ensure
  that you continue to make progress on the longer paper.
\end{itemize}

\subsection{Digital Humanities
Projects}\label{digital-humanities-projects}

According to the Modern Language Association Committee on Scholarly
Editions, ``scholarly editions make clear what they promise and keep
their promises.'' There are five criteria according to which scholarly
editions are evaluated: accuracy, adequacy, appropriateness,
consistency, and explicitness. I will aid you in choosing a project that
is assumed to be manageable over the course of the last six weeks of the
semester. By regular communication and with my assistance (during review
of proposals and interim project stages), we will ensure that you
develop reasonable claims about your intended project and are confident
that you can live up to those claims. At the end of the term, when I
review your project systematically, I expect to see a substantial amount
of work, an accounting of procedures that were used to establish
accuracy and consistency in your work, and a clear acknowledgment if
portions of the anticipated or promised work remain undone.

\subsection{Cheating and Plagiarism}\label{cheating-and-plagiarism}

By second week of class, I will post a Blackboard assignment in which
you affirm your familiarity with the university's cheating and
plagiarism policy and in which sanctions for cheating and plagiarism are
described. You must complete the assignment before you can earn credit
for class submissions.

For a violation on a minor assignment---if you cheat or plagiarize on a
quiz, take-home assignment, or blog assignment---you will receive a
permanent zero on the assignment, one which will be calculated with the
final average even if another higher grade is dropped. For cases of
possibly inadvertent misrepresentations (citation omitted, quotation
presented as paraphrase), you will be reminded of the importance by a
deduction of one letter grade to the assignment. A second violation on a
minor assignment will be treated as a serious violation.

The following violations are treated as serious violations. If you cheat
or plagiarize on an exam or paper (proposal or draft or final) or if you
submit falsified information to avoid penalties for late submission, you
must submit a full non-plagiarized version capable of earning a grade of
``B.'' But the grade you receive will be a ``0.'' If you fail to fulfill
the make-up requirement, you will automatically fail the course. For
cases of possibly inadvertent misrepresentations (citation page not
printed, quotation presented as paraphrase), you earn a permanent
deduction of one letter grade for the assignment.

I have generally found that detection of a single incident of plagiarism
requires further investigation of previous assignments. If I detect
plagiarism on one assignment, you will be asked to withdraw previous
assignments (grade changed to ``0'') or resubmit them for review.

For one serious violation or two minor violations, I will forward the
evidence to the department chair, have the charge added to your record
with the college, recommend further judicial sanction, and pursue the
case during the appeals process. As plagiarism accusation procedures
require an opportunity for student to offer defense, plagiarism
accusations on final exam or final paper may result in a grade of
incomplete until procedure can be completed during the following
semester.

\textbf{Note}: The university faculty senate has recommended an option
called plagiarism school for the first incident in which a student is
accused of plagiarism. If this course is your first incident, I will
recommend you to ``plagiarism school,'' which will be required. If you
have previously been accused of plagiarism or have attended plagiarism
school before, you are not eligible for it and shall face consequences
above.

\textbf{How Not to Plagiarize: } Amanda French has offered helpful
advice on impermissible copying, especially actions that constitute
plagiarism and copyright violation.
\textless{}\url{http://digitalpast002.onmason.com/syllabus/}\textgreater{}:

\begin{quote}
If you are copying and pasting text that someone else wrote, you might
be plagiarizing. Pasted or manually retyped text is not plagiarized only
when all of the following three conditions are true: 1) the pasted text
is surrounded by quotation marks or set off as a block quote, and 2) the
pasted text is attributed in your text to its author and its source
(e.g., ``As Jane Smith writes on her blog . . .''), and 3) the pasted
text is cited in a footnote, endnote, and/or a bibliography (e.g.,
``Smith, Jane. Smith Stuff. Blog. Available
\textless{}\url{http://smithstuff.wordpress.com}\textgreater{} Accessed
August 1, 2012.'') Conventions for copying and pasting computer code are
less strict, but even when you copy and paste code, if you can identify
the actual individual who wrote the code, you should give the coder's
name and the source of the code in a code comment. If you find and use
images, audio, or video on the web, you should also cite the creator (if
known) and the source (at the very least) of that media file, usually in
a caption as well as in a footnote, endnote, or bibliography. Note that
reproducing someone else's text, image, audio, or video file in full on
your own public website may constitute copyright infringement, even with
proper attribution.
\end{quote}

That everyone violates formal copyright now or that tech-evangelists or
corporate shills on Twitter or YouTube---or Facebook or Instagram or
Pandora or Google, etc.---endorse a culture of free sharing of
copyrighted content, is not sufficient for you to escape the
consequences of plagiarism within this class. Times and laws change, but
my demand that you hold yourself to a high standard for ethical behavior
is fully within the realm of course policy. I am not qualified to give
legal advice on copyright, but I can advise sensible self-protection.
When you post material on a public web site, due diligence will help you
defend yourself against claims of copyright infringement. To exercise
due diligence, see Cornell University's ``Copyright Term and the Public
Domain in the United States''
\textless{}\url{http://copyright.cornell.edu/resources/publicdomain.cfm}\textgreater{}.
Thoreau, who called for civil disobedience, spent the night in jail. If
your violation of copyright is principled, I assume that failing a
college course assignment is a reasonable opportunity to test whether
you are truly devoted to your principles. If your copyright violations
are clear and in wanton disregard to guidelines, you will be assigned a
failing grade on assignment.

\subsection{Course Material Copyright}\label{course-material-copyright}

The university counsel (attorney's office) has notified professors that
students are selling course materials (presentations, handouts, notes,
exams, etc.) to an internet company. The company re-sells those
materials to subscribers. Selling course materials violates a
professor's copyright: the company is re-selling stolen intellectual
property. Course materials that I create and display or distribute to
students (unless they are owned by someone else and distributed under
fair use guidelines) are my intellectual property. Likewise, were I to
sell your work on a term paper web site, I would be violating your
copyright.

However, my course materials build on the work of other scholars.
Therefore, I claim what is known as an Attribution-NonCommercial License
(CC By-NC). See
\textless{}\url{http://creativecommons.org/licenses/}\textgreater{} for
details. In sum, you have permission to remix, tweak, or build upon my
work (for example, as a school lesson plan), but you must also release
your new remixed work (if it is substantially similar content) in
noncommercial form. If you create a derivative work (that is, you cite
me when creating something new, but yours is a substantially different
work), you do have permission to license your own work on a commercial
basis.

Please note that my course material copyright differs from standard
syllabus boilerplate that the university counsel recommends. Unless
another professor offers materials under a Creative Commons license, the
usual copyright rules apply for material from that professor.

\subsection{Credits}\label{credits}

Credit to other syllabi.

\section{Course Schedule}\label{course-schedule}

\subsection{Week 1: Digital humanities in popular
culture}\label{week-1-digital-humanities-in-popular-culture}

Readings on history of humanities computing and digital humanities, by
scholars and by outside critics: barbarians at the gates narrative and
the destruction of the human.

\subsubsection{Readings (Jan. 14):}\label{readings-jan.-14}

\begin{itemize}
\itemsep1pt\parskip0pt\parsep0pt
\item
  Marche, Stephen, ``Literature is not Data: Against Digital
  Humanities.'' \emph{Los Angeles Review of Books}
  \url{http://lareviewofbooks.org/essay/literature-is-not-data-against-digital-humanities}
\item
  Kirsch, Adam, ``Technology Is Taking Over English Departments.''
  \emph{The New Republic}
  \url{http://www.newrepublic.com/article/117428/limits-digital-humanities-adam-kirsch}
\item
  Susan Hockey, ``The History of Humanities Computing.'' \emph{A
  Companion to Digital Humanities.} 3-19.
\end{itemize}

\subsubsection{Assignment:}\label{assignment}

\begin{itemize}
\itemsep1pt\parskip0pt\parsep0pt
\item
  ``What is Digital Humanities: A Student Discussion``: Build a Blog,
  Print Document, and eBook
\end{itemize}

\subsubsection{In-Class Activities}\label{in-class-activities}

\begin{itemize}
\itemsep1pt\parskip0pt\parsep0pt
\item
  OS Administrator and File Extensions
\item
  Introduce Command Line
\item
  Introduce Plain Text Editors
\end{itemize}

\subsection{Week 2: What does a digital humanist
do?}\label{week-2-what-does-a-digital-humanist-do}

Discussions about what digital humanists do by digital humanists.

\subsubsection{Readings (Jan. 21):}\label{readings-jan.-21}

\begin{itemize}
\itemsep1pt\parskip0pt\parsep0pt
\item
  Kirschenbaum, Matthew. ``What is Digital Humanities and What's It
  Doing in English Departments?''
  \url{http://mkirschenbaum.files.wordpress.com/2011/01/kirschenbaum_ade150.pdf}
\item
  Flanders, Julia. ``The productive unease of 21st-century digital
  scholarship''
  \url{http://www.digitalhumanities.org/dhq/vol/3/3/000055/000055.html}
\item
  Ramsay, Stephen. ``On Building.'' \emph{Stephen Ramsay.}
  \url{http://stephenramsay.us/text/2011/01/11/on-building/}
\item
  Sample, Mark. ``The digital humanities is not about building, it's
  about sharing.'' \url{http://bit.ly/1kLZ8XW}
\end{itemize}

\subsubsection{Assignment:}\label{assignment-1}

\begin{itemize}
\itemsep1pt\parskip0pt\parsep0pt
\item
  Activity 1: Install Pandoc.
\item
  Activity 2: Convert Sample Markdown File with a bibliography to Word
  RTF and HTML
\item
  Activity 3: Write blog post with bibliographical reference on what DH
  is
\end{itemize}

\subsection{Week 3: Form and content}\label{week-3-form-and-content}

How does one distinguish between form (the technological underpinnings
or matters of presentation) and content (the stuff for interpretation)?
If we suppose we can distinguish between form and content, is that a
useful step, or are we likely to deceive ourselves by taking such
distinctions for granted? Could the distinction between form and content
be both useful for some purposes and deceptive for others? We will
explore this question by preparing a document in Markdown format, which
will then be converted to forms suitable for alternate means of
publication: print document, blog, and ebook.

\subsubsection{Readings (Jan. 26):}\label{readings-jan.-26}

\begin{itemize}
\itemsep1pt\parskip0pt\parsep0pt
\item
  ``Single-Source Publishing.'' Wikipedia.
  \url{http://en.wikipedia.org/wiki/Single_source_publishing}
\item
  Sullivan, Ian. ``Innovation in practice.'' Software Freedom Law
  Center.
  \url{https://www.softwarefreedom.org/blog/2014/apr/11/innovation-in-practice/}
\end{itemize}

\subsubsection{Readings (Jan. 28)}\label{readings-jan.-28}

\begin{itemize}
\itemsep1pt\parskip0pt\parsep0pt
\item
  Laue, Andrea. ``How the Computer Works.'' \emph{A Companion to Digital
  Humanities.} 145-160.
\item
  Sperberg-McQueen, C. M. ``Classification and its Structures'' \emph{A
  Companion to Digital Humanities.} 161-176.
\item
  Borges, Jorge Luis. ``Funes, the Memorious.'' Blackboard.
\end{itemize}

\subsubsection{Assignment:}\label{assignment-2}

\begin{itemize}
\itemsep1pt\parskip0pt\parsep0pt
\item
  Activity 1: Install GitHub and DropBox and submit Markdown-format Blog
  Post
\item
  Activity 2: Install Zotero, and export BibLaTeX (*.bib) files
\item
  Activity 3: Install (as necessary) LaTeX or Calibre, or set up Reclaim
  Hosting site with WordPress
\end{itemize}

\subsection{Week 4: Publishing Class Single-Source Discussion of ``What
is
DH''?}\label{week-4-publishing-class-single-source-discussion-of-what-is-dh}

\subsubsection{Assignments:}\label{assignments}

\begin{itemize}
\itemsep1pt\parskip0pt\parsep0pt
\item
  All: Post revised blog Markdown source to GitHub
\item
  Assignment 1 Print Working Group: Submit LaTeX-generated print copy
  and PDF to Team GitHub Repository
\item
  Assignment 1 Public Blog Working Group: Submit link to blog home page
  on Team GitHub Repository
\item
  Assignment 1 eBook Working Group: Submit link to eBook on Team GitHub
  Repository
\item
  All Students: Post Assignment 1 Reflection to class blog, 500 words
\end{itemize}

\subsection{Week 5: Acquiring Texts}\label{week-5-acquiring-texts}

When working with humanities material, we often need to acquire text.
While many documents are available online, in some cases, the text that
we need to acquire are not. For our final project, with archival
letters, we are the ones who plan to make them available to others.
Therefore, we want to be familiar with automated techniques to acquire
text. We shall also gain a greater familiarity with some qualities of
archival materials, which are often viewed as ``faulty'' or
``inadequate'' for machine processing. We continue our concern with
whether there is a useful distinction between the form of archival
documents and their linguistic content.

\subsubsection{Readings (Feb. 2):}\label{readings-feb.-2}

\begin{itemize}
\itemsep1pt\parskip0pt\parsep0pt
\item
  Berman, Ruth. ``\,`Spirituous Consolation': Alcott's Jokes on Drinking
  and Religion.'' \emph{Children's Literature in Education} 39:3 (2008):
  169-185.
  \url{http://search.ebscohost.com/login.aspx?direct=true\&db=aph\&AN=32679769\&site=eds-live}
\item
  Cohen, Daniel J. and Roy Rosenszwig, ``How to Make Text Digital:
  Scanning, OCR, and Typing.'' \emph{Digital History}
  \url{http://chnm.gmu.edu/digitalhistory/digitizing/4.php}
\end{itemize}

\subsubsection{Readings (Feb. 4):}\label{readings-feb.-4}

\begin{itemize}
\itemsep1pt\parskip0pt\parsep0pt
\item
  Eileen Giffort Fenton and Hoyt N. Duggan, ``Effective Methods of
  Producing Machine-Readable Text from Print and Manuscript Sources.''
  \emph{Electronic Textual Editing} 241-253.
\item
  Driscoll, M. J. ``Levels of Transcription.'' \emph{Electronic Textual
  Editing.} 254-261.
\item
  Alcott, Louisa May. \emph{Hospital Sketches. Boston Commonwealth} 22
  May 1863, 29 May 1863, 12 June 1863, and 26 June 1863. (scans posted
  to Blackboard, a ``browsing'' would be more a more accurate
  description than a ``reading'' for this selectoin).
\end{itemize}

\subsubsection{Assignment:}\label{assignment-3}

\begin{itemize}
\itemsep1pt\parskip0pt\parsep0pt
\item
  Acquire text from Internet Archive or Project Gutenberg or University
  of Pennsylvania
\item
  Install Tesseract and OCR selection from \emph{Boston Commonwealth}
  version of \emph{Hospital Sketches}
\item
  Create Twitter account (class or personal) and designate hash tag
\end{itemize}

\subsection{Week 6: Back to the archive
again}\label{week-6-back-to-the-archive-again}

\subsubsection{Readings (Feb. 9):}\label{readings-feb.-9}

\begin{itemize}
\itemsep1pt\parskip0pt\parsep0pt
\item
  Alcott, \emph{Hospital Sketches} (Boston: Redpath, 1863),
  \url{https://archive.org/details/hospitalsketches00alcorich}
\item
  Alcott, \emph{Hospital Sketches and Camp and Fireside Stories}
  (Boston: Roberts Brothers, 1868),
  \url{https://archive.org/details/hospitalsketche00alco}
\item
  Kline, Mary-Jo and Susan Holbrook Perdue, ``{[}Section{]} A:
  Establishing the Editorial Texts''
  \url{http://gde.upress.virginia.edu/06-gde.html}
\end{itemize}

\subsubsection{Readings (Feb. 11):}\label{readings-feb.-11}

\subsubsection{Assignment:}\label{assignment-4}

\begin{itemize}
\itemsep1pt\parskip0pt\parsep0pt
\item
  Transcribe selection from \emph{Hospital Sketches} (Group Project)
\item
  Visit library special collections
\item
  Install Omeka, WordPress and Drupal
\end{itemize}

\subsection{Week 7:}\label{week-7}

\subsubsection{Readings (Feb. 16):}\label{readings-feb.-16}

\begin{itemize}
\itemsep1pt\parskip0pt\parsep0pt
\item
  Duggan, Hoyt N. and Eileen Gifford Fenton, ``Effective methods of
  producing machine-readable text from manuscript and print sources''
  \emph{Electronic Textual Editing} 241--253.
\item
  Turkel, William J. ``Doing OCR Using Command Line in UNIX''
  \url{http://williamjturkel.net/2013/07/06/doing-ocr-using-command-line-tools-in-linux/}
\end{itemize}

\subsubsection{Readings (Feb. 18):}\label{readings-feb.-18}

\begin{itemize}
\itemsep1pt\parskip0pt\parsep0pt
\item
  Daniel J. Cohen and Roy Rosenszweig, ``Digital Images'' \emph{Digital
  History,} \url{http://chnm.gmu.edu/digitalhistory/digitizing/5.php}
\end{itemize}

\subsubsection{Assignment:}\label{assignment-5}

\begin{itemize}
\itemsep1pt\parskip0pt\parsep0pt
\item
  Compare Transcribed and OCR Text in JUXTA\\
\item
  Orally proofread transcribed text against facsimile document
\end{itemize}

\subsection{Week 8: Text and Encoding}\label{week-8-text-and-encoding}

\subsubsection{Readings (Feb. 23):}\label{readings-feb.-23}

\begin{itemize}
\itemsep1pt\parskip0pt\parsep0pt
\item
  Birnbaum, David J. ``What is XML and why should humanists care? An
  even gentler introduction to XML'' \emph{Digital Humanities}
  \url{http://dh.obdurodon.org/what-is-xml.xhtml}
\end{itemize}

\subsubsection{Readings (Feb. 25):}\label{readings-feb.-25}

\begin{itemize}
\itemsep1pt\parskip0pt\parsep0pt
\item
  Ramsay, Stephen, ``Using Regular Expressions,'' \emph{Electronic Text
  Center: University of Virginia} (Blackboard)
\item
  DeRose, S. J., Durand, D. G., Mylonas, E., and Renear A. H. (1990),
  ``What is Text, Really?'' \emph{* Journal of Computing in Higher
  Education} 1.2: 3-26.
\item
  Renear, Alan. ``Text Encoding.'' \emph{A Companion to the Digital
  Humanities}
\end{itemize}

\subsubsection{Assignment:}\label{assignment-6}

\begin{itemize}
\itemsep1pt\parskip0pt\parsep0pt
\item
  Introduce text encoding
\item
  Introduce REGEX
\end{itemize}

\subsection{Week 9: What is text,
really?}\label{week-9-what-is-text-really}

\subsubsection{Readings (Mar. 2):}\label{readings-mar.-2}

\begin{itemize}
\itemsep1pt\parskip0pt\parsep0pt
\item
  Minnesota Digital Library, ``Quick Reference Guide: Digital Imaging
  Best
  Practices''\url{http://www.mndigital.org/digitizing/standards/guide.php}
\item
  University Library, University of Illinois, Urbana-Champagne ``3.0
  Best Practices for Creating Digital Images''
  (http://www.library.illinois.edu/dcc/bestpractices/chapter\_03\_creatingdigitalimages.html\#textdoc){[}http://www.library.illinois.edu/dcc/bestpractices/chapter\_03\_creatingdigitalimages.html\#textdoc{]}
\end{itemize}

\subsubsection{Readings (Mar. 4):}\label{readings-mar.-4}

\begin{itemize}
\itemsep1pt\parskip0pt\parsep0pt
\item
  Daniel J. Cohen and Roy Rosenzweig, ``To Mark Up, Or Not To Mark Up''
  \emph{Becoming Digital}
  (http://chnm.gmu.edu/digitalhistory/digitizing/3.php){[}http://chnm.gmu.edu/digitalhistory/digitizing/3.php{]}
\item
  ``Gentle Introduction to XML''
  (http://www.tei-c.org/release/doc/tei-p5-doc/en/html/SG.html){[}http://www.tei-c.org/release/doc/tei-p5-doc/en/html/SG.html{]}
\end{itemize}

\subsubsection{Assignment:}\label{assignment-7}

\begin{itemize}
\itemsep1pt\parskip0pt\parsep0pt
\item
  Propose Project Plan
\end{itemize}

\subsection{Week 10:}\label{week-10}

\subsubsection{Readings (Mar. 9):}\label{readings-mar.-9}

\begin{itemize}
\itemsep1pt\parskip0pt\parsep0pt
\item
  Pitti, Daniel V. ``Designing Sustainable Projects and Publications,''
  \emph{Companion to Digital Humanities}
  \url{http://nora.lis.uiuc.edu:3030/companion/view?docId=blackwell/9781405103213/9781405103213.xml}
\item
  ``Conversion of Primary Sources,'' \emph{Companion to Digital
  Humanities}
  \url{http://nora.lis.uiuc.edu:3030/companion/view?docId=blackwell/9781405103213/9781405103213.xml}
\end{itemize}

\subsubsection{Readings (Mar. 11):}\label{readings-mar.-11}

\subsubsection{Assignment:}\label{assignment-8}

\begin{itemize}
\itemsep1pt\parskip0pt\parsep0pt
\item
  Encode prose and poetry
\item
  Introduce Scanner
\end{itemize}

\subsection{Week 11: Digital Publication
Options}\label{week-11-digital-publication-options}

\subsubsection{Readings (Mar. 16):}\label{readings-mar.-16}

\begin{itemize}
\itemsep1pt\parskip0pt\parsep0pt
\item
  Pape, Sebastian, Christof Schöch, and Lutz Wegner, ``TEICHI and the
  Tools Paradox'' \url{http://jtei.revues.org/432}
\item
  ``Documentation'' and ``Downloads,'' TEICHI
  \url{http://www.teichi.org/}
\item
  ``Documentation: What Is Omeka'' \url{http://omeka.org/}
\end{itemize}

\subsubsection{Assignment:}\label{assignment-9}

\begin{itemize}
\itemsep1pt\parskip0pt\parsep0pt
\item
  Drupal on Reclaim Hosting, and TEICHI on Drupal
\item
  Omeka on Reclaim Hosting
\item
  Introduce Saxon and XSLT
\end{itemize}

\subsection{Week 12}\label{week-12}

\subsubsection{Readings (Mar. 30):}\label{readings-mar.-30}

\begin{itemize}
\itemsep1pt\parskip0pt\parsep0pt
\item
  Johnson, Peter K. ``Installing and Using Saxon for your XSLT
  Development'' \url{http://www.youtube.com/watch?v=FsDq2-VV0Uo}
\item
  Installing Saxon and Running from Command Line
  \url{http://www.saxonica.com/documentation/using-xsl/commandline.html}
\end{itemize}

\#\#\# Readings (Apr. 1):

\begin{itemize}
\itemsep1pt\parskip0pt\parsep0pt
\item
  ``Introduction'' \emph{TEI by Example}
  \url{http://teibyexample.org/examples/TBED00v00.htm}
\item
  ``Common Structure and Elements'' \emph{TEI by Example}
  \url{http://teibyexample.org/modules/TBED01v00.htm}
\end{itemize}

\subsubsection{Assignment:}\label{assignment-10}

\begin{itemize}
\itemsep1pt\parskip0pt\parsep0pt
\item
  XML and TEILite
\item
  Saxon and XSLT
\end{itemize}

\subsection{Week 13: Who is our
audience?}\label{week-13-who-is-our-audience}

\subsubsection{Readings (Apr. 6):}\label{readings-apr.-6}

\begin{itemize}
\itemsep1pt\parskip0pt\parsep0pt
\item
  ``TEI Header'' \emph{TEI by Example}
  \url{http://teibyexample.org/examples/TBED02v00.htm}
\item
  ``Prose'' \emph{TEI by Example}
  \url{http://teibyexample.org/examples/TBED03v00.htm} -
\end{itemize}

\subsubsection{Assignment:}\label{assignment-11}

\begin{itemize}
\itemsep1pt\parskip0pt\parsep0pt
\item
  Project Work
\end{itemize}

\subsection{Week 14:}\label{week-14}

\subsubsection{Readings:}\label{readings}

\begin{itemize}
\itemsep1pt\parskip0pt\parsep0pt
\item
  TEILite \url{http://www.tei-c.org/Guidelines/Customization/Lite/}
\item
  ``Primary Sources'' \emph{TEI by Example}
  \url{http://teibyexample.org/examples/TBED06v00.htm}
\end{itemize}

\subsubsection{Assignment:}\label{assignment-12}

\begin{itemize}
\itemsep1pt\parskip0pt\parsep0pt
\item
  Project Work
\end{itemize}

\subsection{Week 15: Sharing our wares}\label{week-15-sharing-our-wares}

\subsubsection{Assignment:}\label{assignment-13}

\begin{itemize}
\itemsep1pt\parskip0pt\parsep0pt
\item
  Project Presentation
\end{itemize}

\subsection{Finals Week}\label{finals-week}

\begin{itemize}
\itemsep1pt\parskip0pt\parsep0pt
\item
  Submit Project ASsignment
\end{itemize}

\end{document}
