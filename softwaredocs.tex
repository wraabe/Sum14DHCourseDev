\documentclass[]{article}
\usepackage{lmodern}
\usepackage{amssymb,amsmath}
\usepackage{ifxetex,ifluatex}
\usepackage{fixltx2e} % provides \textsubscript
\ifnum 0\ifxetex 1\fi\ifluatex 1\fi=0 % if pdftex
  \usepackage[T1]{fontenc}
  \usepackage[utf8]{inputenc}
\else % if luatex or xelatex
  \ifxetex
    \usepackage{mathspec}
    \usepackage{xltxtra,xunicode}
  \else
    \usepackage{fontspec}
  \fi
  \defaultfontfeatures{Mapping=tex-text,Scale=MatchLowercase}
  \newcommand{\euro}{€}
\fi
% use upquote if available, for straight quotes in verbatim environments
\IfFileExists{upquote.sty}{\usepackage{upquote}}{}
% use microtype if available
\IfFileExists{microtype.sty}{\usepackage{microtype}}{}
\ifxetex
  \usepackage[setpagesize=false, % page size defined by xetex
              unicode=false, % unicode breaks when used with xetex
              xetex]{hyperref}
\else
  \usepackage[unicode=true]{hyperref}
\fi
\hypersetup{breaklinks=true,
            bookmarks=true,
            pdfauthor={},
            pdftitle={},
            colorlinks=true,
            citecolor=blue,
            urlcolor=blue,
            linkcolor=magenta,
            pdfborder={0 0 0}}
\urlstyle{same}  % don't use monospace font for urls
\setlength{\parindent}{0pt}
\setlength{\parskip}{6pt plus 2pt minus 1pt}
\setlength{\emergencystretch}{3em}  % prevent overfull lines
\setcounter{secnumdepth}{0}


\begin{document}

\section{Software Advice}\label{software-advice}

\subsection{General Tips}\label{general-tips}

\begin{enumerate}
\def\labelenumi{\arabic{enumi}.}
\itemsep1pt\parskip0pt\parsep0pt
\item
  In past 3 years, modern operating system (OS) releases have added
  hurdles if you seek to install software from non-approved vendors. The
  warnings typically invoke viruses or security concerns. Free and
  open-source software, generally maintained by academics, often
  designed originally for servers and UNIX operating system, has been
  around longer, relies on volunteer labor, and is not going to start
  jumping through OS vendor hoops, because the work on software is not
  for-profit and because many people who do the work on software have
  other jobs. Therefore, when installing open source software, your
  Windows or Macintosh OS may issue ominous security warnings. You are
  an adult: if you trust me, you should override the OS settings.\\
\item
  To install software, you must have an administrator account on your
  computer. And you must also be logged in as the administor when
  installing. In Windows, especially, you may be required to install or
  run software in administrator mode.
\item
  Much open-source software will be available through GitHub or
  SourceForge and sometimes on web sites that look dicey to you because
  they still employ frames and fonts out of 1990s. Despite my advice
  that you trust me, SourceForge is a private company that has been
  known to take advantage of newbie users. See
  \href{http://www.gluster.org/2013/08/how-far-the-once-mighty-sourceforge-has-fallen/}{SourceForge
  Warning}. I will try to warn you of potential pitfalls, but I
  encourage you to warn one another if you encounter problems. GitHub is
  generally more widely used today for distribution of open source
  software.
\item
  Likewise, open source software originally designed for UNIX may have
  less support for mouse, may demand a whole arsenal of arcane commands,
  may have minimal availability of handholding tutorials, and may offer
  almost no immediate rewards during first hours of use, first days of
  use, or even first weeks of use. That is, expect 20 hours of struggle
  to reach minimal competence. Ask me about how long it took to achieve
  basic competence with LaTeX, PanDoc, Vim Text Editor, GitHub, etc. Thr
  answers: 4 mos., 4 weeks, 2 mos., 1 week, etc.
\item
  Operating Systems:
\end{enumerate}

\begin{itemize}
\itemsep1pt\parskip0pt\parsep0pt
\item
  Apple Macintosh, you should be on Snow Leopard, Snow Lion, or
  Maverick. You may install and run UNIX-based software on Macintosh,
  which has been a UNIX OS for several years. Often, such as for Vim,
  you may have a choice of a UNIX-style version (shipped with OS) and a
  ported Macintosh version that you can download from a separate web
  site, which will have better support for usual Macintosh interface and
  user behaviors.
\item
  MS Windows, you should be on Windows 7 or later. Windows XP is no
  longer supported. Windows Vista is generally OK, but it is fading in
  use.
\item
  UNIX implementations like Ubuntu, Red Hat for laptop or desktop
  system. Any modern UNIX implementation should work fine. But you're on
  your own, like always.
\item
  Apple or Android Tablets, Google Chrome, etc. These are not supported,
  and I have no idea whether you can do the work using them. I have no
  recommendations for OS-level hacking, emulation, dual boot modes, etc.
  I am not a hacker: I am relatively proficient for an English
  professor.\\
\end{itemize}

\begin{enumerate}
\def\labelenumi{\arabic{enumi}.}
\setcounter{enumi}{4}
\itemsep1pt\parskip0pt\parsep0pt
\item
  I am not a substitute for trained computer troubleshooters or
  technicians. For software, contact \href{http://support.kent.edu}{KSU
  HelpDesk}. For virus and component repair, contact
  \href{http://www2.kent.edu/is/support/thetechspot/index.cfm}{KSU
  TechSpot}.
\end{enumerate}

\subsection{Managing Your Own Web Identity and
Projects}\label{managing-your-own-web-identity-and-projects}

\subsubsection{Reclaim Hosting}\label{reclaim-hosting}

\subsubsection{WordPress}\label{wordpress}

\subsubsection{Omeka}\label{omeka}

\subsection{Plain Text Editors}\label{plain-text-editors}

It is very important to use a plain text editor instead of a word
processor when preparing text-based documents. Word processors like
Microsoft Word and Google Document and Open Office and Apple Pages
create documents that are difficult to read outside of the software and
are often difficult to translate into plain text documents that can be
used by other software. Learning all the vagaries of an individual word
processor is a less valuable skill than learning to encode plain text
documents so that they easily translate from one operating system and
one processing system to another. The latter is a goal of this course.
Another course goal is to reduce or eliminate your dependence on any
particular software vendor. Both are valuable for professions in which
you have to prepare documentation and analyze data---because software
development professionals will recognize and trust you as a fellow
professional. I do not care which of the following you use, but Vi and
emacs are entire worlds of their own, into which considerable immersion
is necessary to achieve basic competence. But they are very powerful and
very widely used.

\subsubsection{Vi Improved (Vim), for UNIX, Macintosh, and
Windows}\label{vi-improved-vim-for-unix-macintosh-and-windows}

\begin{itemize}
\itemsep1pt\parskip0pt\parsep0pt
\item
  Download: \href{http://www.vim.org/}{Vim Installer}. Select the
  appropriate executable installer for Windows or Mac (MacVim), and
  follow instructions for installation.
\item
  Complete the basic online tutorial (browser-based VIM emulation) at
  \href{http://www.openvim.com/tutorial.html}{Open Vim}.
\item
  Within actual VIM, complete the vimtutor. 1) Open VIM, 2) Click . 3)
  Type ``vimtutor'' (exclude quote marks) and press . If that does not
  work, type ``:help vimtutor'' and follow the instructions. Or, type
  ``:echo \$VIMRUNTIME'' to figure out where vimtutor is located.
  Typically, it's located in ``tutor'' folder within the \$VIMRUNTIME
  location. For browser-based emulation of text file, see
  \href{http://www2.geog.ucl.ac.uk/~plewis/teaching/unix/vimtutor}{vimtutor}.
  To ``read'' the tutorial here is not enough. Complete each exercise
  within VIM, about 45 minutes.
\item
  For higher proficiency, see
  \href{http://www.danielmiessler.com/study/vim/}{Daniel Miessler, VIM
  Tutorial and Primer} And see
  \href{http://tnerual.eriogerg.free.fr/vimqrc.pdf}{Laurent Grégoire,
  VIM Quick Reference}
\end{itemize}

\subsubsection{emacs for UNIX, Windows, and
Macintosh}\label{emacs-for-unix-windows-and-macintosh}

\begin{itemize}
\itemsep1pt\parskip0pt\parsep0pt
\item
  Download (Hard Way): {[}emacs Home
  Page{]}(``http://www.gnu.org/software/emacs/emacs.html''
  http://www.gnu.org/software/emacs/emacs.html). Select the ``Nearby GNU
  Mirror'' in ``Obtaining/Downloading GNU emacs.''
\item
  Download (Macintosh, Easy Way): If you are on Macintosh, emacs is
  already installed. But for the latest version, see
  \url{http://emacsformacosx.com/}. For a more Mac-like emacs, see
  \href{http://aquamacs.org/about.shtml}{Aquamacs}
\item
  Download (Windows, no Easy Way): If you are on Windows, select
  ``windows'' folder and then download ``emacs-24.3-bin-i386.zip''.
  Instructions on successful Windows setup are available at {[}Claremont
  McKenna Faculty Page on emacs Windows
  Setup{]}(``http://www.claremontmckenna.edu/pages/faculty/alee/emacs/emacs.html''
  http://www.claremontmckenna.edu/pages/faculty/alee/emacs/emacs.html)
\item
  For introductory tutorial, see
  \href{http://www.jesshamrick.com/2012/09/10/absolute-beginners-guide-to-emacs/}{Jess
  Hamrick's Absolute Beginner's Guide}. For a more amibitious overview,
  within emacs, type C-h t.
\end{itemize}

\subsubsection{Other Plain Text Editors}\label{other-plain-text-editors}

\begin{itemize}
\itemsep1pt\parskip0pt\parsep0pt
\item
  BBEdit (paid) or free TextWrangler (Macintosh):
  \href{http://www.barebones.com/products/textwrangler/}{Bare Bones Text
  Wrangler Download}
\item
  Notepad ++ (Windows, free):
  \href{http://notepad-plus-plus.org/}{Notepad ++}
\item
  Sublime Text, test and then buy (Macintosh or Windows)
  \url{http://www.sublimetext.com/3}
\end{itemize}

\subsection{Working Together}\label{working-together}

Working together is difficult, but you will be working in teams. And you
need to develop routines to ensure that you can work together
efficiently. The only required tool will be GitHub, which you will use
to manage your source text documents, but you may also use other tools
to organize your collaboration with other members of your team.

\end{document}
