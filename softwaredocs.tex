\documentclass[]{article}
\usepackage[T1]{fontenc}
\usepackage{lmodern}
\usepackage{amssymb,amsmath}
\usepackage{ifxetex,ifluatex}
\usepackage{fixltx2e} % provides \textsubscript
% use upquote if available, for straight quotes in verbatim environments
\IfFileExists{upquote.sty}{\usepackage{upquote}}{}
\ifnum 0\ifxetex 1\fi\ifluatex 1\fi=0 % if pdftex
  \usepackage[utf8]{inputenc}
\else % if luatex or xelatex
  \ifxetex
    \usepackage{mathspec}
    \usepackage{xltxtra,xunicode}
  \else
    \usepackage{fontspec}
  \fi
  \defaultfontfeatures{Mapping=tex-text,Scale=MatchLowercase}
  \newcommand{\euro}{€}
\fi
% use microtype if available
\IfFileExists{microtype.sty}{\usepackage{microtype}}{}
\ifxetex
  \usepackage[setpagesize=false, % page size defined by xetex
              unicode=false, % unicode breaks when used with xetex
              xetex]{hyperref}
\else
  \usepackage[unicode=true]{hyperref}
\fi
\hypersetup{breaklinks=true,
            bookmarks=true,
            pdfauthor={},
            pdftitle={},
            colorlinks=true,
            citecolor=blue,
            urlcolor=blue,
            linkcolor=magenta,
            pdfborder={0 0 0}}
\urlstyle{same}  % don't use monospace font for urls
\setlength{\parindent}{0pt}
\setlength{\parskip}{6pt plus 2pt minus 1pt}
\setlength{\emergencystretch}{3em}  % prevent overfull lines
\setcounter{secnumdepth}{0}

\author{}
\date{}

\begin{document}

\section{Software Documentation}\label{software-documentation}

\subsection{General Tips}\label{general-tips}

\begin{enumerate}
\def\labelenumi{\arabic{enumi}.}
\itemsep1pt\parskip0pt\parsep0pt
\item
  In past 3 years, modern OS releases have added hurdles if you seek to
  install software from non-approved vendors. The warnings typically
  invoke viruses or security concerns. Open-source software, generally
  maintained by academics, has been around longer, relies on volunteer
  labor, and is not going to start jumping through vendor hoops because
  the work is not for-profit. Therefore, when installing open source
  software, you may receive ominous OS warnings. You should trust me and
  override the OS settings.\\
\item
  To install software, you must have an administrator account on your
  computer. And you must also be logged in as the administor when
  installing. In Windows, especially, you may be required to install or
  run software in administrator mode.
\item
  Much open-source software will be available through GitHub or
  SourceForge and sometimes on web sites that look dicey to you because
  they still employ frames and fonts out of 1990s. Despite advice that
  you trust me, SourceForge is a private company that has difficulty
  earning a profit. So it has been known to take advantage of newbie
  users.
  \href{http://www.gluster.org/2013/08/how-far-the-once-mighty-sourceforge-has-fallen/}{SourceForge
  Warning} I will try to warn you of potential pitfalls, but I encourage
  you to warn one another if you encounter problems. One risk is that by
  end of course, 90s-style web sites with no dancing graphics may seem
  kind of cool to you. GitHub is generally more widely used today.
\item
  Likewise, open source UNIX-based software may have less support for
  mouse, may demand a whole arsenal of arcane commands, may have minimal
  availability of handholding tutorials, and may offer almost no
  immediate rewards during first hours of use, first days of use, or
  even first weeks of use. That is, expect 20 hours of struggle to reach
  minimal competence. Ask me about how long it took to achieve basic
  competence with LaTeX, PanDoc, Vim Text Editor, GitHub, etc. That
  answers: 4 mos., 4 weeks, 2 mos., 1 week, etc.
\item
  Operating Systems:
\end{enumerate}

\begin{itemize}
\itemsep1pt\parskip0pt\parsep0pt
\item
  Apple Macintosh, you should be on Snow Leopard, Snow Lion, or
  Maverick. You may install UNIX-based software on Macintosh, which has
  been a UNIX OS for several years.
\item
  MS Windows, you should be on Windows 7 or later. Windows XP or Vista
  are not supported.
\item
  UNIX implementations like Ubuntu, Red Hat for laptop or desktop
  system. Any modern UNIX implementation should work fine. But you're on
  your own, like always.
\item
  Apple or Android Tablets, Google Chrome, etc. I have no
  recommendations for OS-level hacking, emulation, dual boot modes, etc.
  In the end, I am still an English professor.\\
\end{itemize}

\begin{enumerate}
\def\labelenumi{\arabic{enumi}.}
\setcounter{enumi}{4}
\itemsep1pt\parskip0pt\parsep0pt
\item
  I will try to help, but I am not a substitute for trained computer
  troubleshooters or technicians. For software, contact
  \href{http://support.kent.edu}{KSU HelpDesk}. For virus and compenent
  repair, contact
  \href{http://www2.kent.edu/is/support/thetechspot/index.cfm}{KSU
  TechSpot}.
\end{enumerate}

\subsection{Text Editors}\label{text-editors}

\begin{enumerate}
\def\labelenumi{\arabic{enumi}.}
\itemsep1pt\parskip0pt\parsep0pt
\item
  Vi Improved (Vim)
\end{enumerate}

\begin{itemize}
\itemsep1pt\parskip0pt\parsep0pt
\item
  Download: \href{http://www.vim.org/}{Vim Installer}. Select
  appropriate executable installer for Windows or Mac (MacVim).
\item
  Follow instructions for installation.
\item
  Complete the basic online tutorial (browser-based VIM emulation) at
  \href{http://www.openvim.com/tutorial.html}{Open Vim}.
\item
  Within actual VIM, complete the vimtutor. 1) Open VIM, 2) Click . 3)
  Type ``vimtutor'' (exclude quote marks) and press . If that does not
  work, type ``:help vimtutor'' and follow the instructions. Or, type
  ``:echo
  $VIMRUNTIME" to figure     out where vimtutor is located. Typically, it's located in "tutor" folder within     the $VIMRUNTIME
  location. For browser-based emulation of text file, see
  \href{http://www2.geog.ucl.ac.uk/~plewis/teaching/unix/vimtutor}{vimtutor}.
  To''read" the tutorial here is not enough. Complete each exercise
  within VIM, about 45 minutes.
\item
  For higher proficiency, see
  \href{http://www.danielmiessler.com/study/vim/}{Daniel Miessler, VIM
  Tutorial and Primer}
\item
  Also see \href{http://tnerual.eriogerg.free.fr/vimqrc.pdf}{Laurent
  Grégoire, VIM Quick Reference}
\end{itemize}

\end{document}
